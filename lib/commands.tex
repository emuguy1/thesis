\newcommandx{\student}[3][]{
	\def\studentName{#1}%
	\def\studentMatnr{#2}%
	\def\studentStudiengang{#3}%
}

\newcommandx{\MyTitelseite}[8][]{
\thispagestyle{empty}
\ifthenelse{\equal{#1}{}}{
	\includegraphics[scale=0.2]{lib/oth-logo.png}
}{
	\includegraphics[scale=0.2]{lib/oth-logo.png}\hfill\includegraphics[scale=0.5]{#1}
}
\begin{center}
\ifthenelse{\equal{#2}{2}}{ % then
	\vspace*{2cm}
	\Large
	\textbf{Ostbayerische Technische Hochschule Regensburg}\\
	\textbf{Fakultät für Informatik und Mathematik}\\
	\vspace*{2cm}
	\Huge
	\textbf{#3}\\[1em]
	\large
	Zur Erlangung des akademischen Grades des\\
	\ifthenelse{\equal{#3}{Bachelorarbeit}}{Bachelor of Science (B.Sc.)}{Master of Science (M.Sc.)}\\
	\vspace*{1cm}
	\Large
	\textbf{#4}\\
}{ % else
	\vspace*{1cm}
	\Large
	\textbf{#4}\\
	\vspace*{2cm}
	\large
	An der Fakultät für Informatik und Mathematik der\\
	Ostbayerischen Technischen Hochschule Regensburg\\
	im Studiengang\\[2em]
	\textbf{\studentStudiengang}\\[2em]
	eingereichte\\
	\vspace*{1cm}
	\Large
	\textbf{#3}\\[2em]
	\large
	zur Erlangung des akademischen Grades des\\
	\ifthenelse{\equal{#3}{Bachelorarbeit}}{Bachelor of Science (B.Sc.)}{Master of Science (M.Sc.)}
	\vspace*{1cm}
	\Large
}
	\vfill
	\normalsize
	%\newcolumntype{x}[1]{>{\raggedleft\arraybackslash\hspace{0pt}}p{#1}}
	\begin{tabular}{rl}%{6cm}p{7.5cm}}
	    \rule{0mm}{1ex}\textbf{Vorgelegt von:} & \studentName \\
		\rule{0mm}{1ex}\textbf{Matrikelnummer:} & \hspace*{-0.5em}\begin{tabular}[t]{r}\studentMatnr\end{tabular} \\ 
		\ifthenelse{\equal{#2}{1}}{~\\}{\rule{0mm}{1ex}\textbf{Studiengang:} & \studentStudiengang \\[2em]}
		\rule{0mm}{1ex}\textbf{Erstgutachter:} & #5 \\ 
		\rule{0mm}{1ex}\textbf{Zweitgutachter:} & #6 \\[2em]
		\rule{0mm}{1ex}\textbf{Abgabedatum:} & #7 \\ 
	\end{tabular} 
\end{center}
\pagebreak
}



% ein paar nützliche Kommandos

\newcommand\defsec[1]{\label{sec:#1}}
\newcommand\refsec[1]{\ref{sec:#1}}

%\newcommand\todo[1]{\textcolor{red}{[{\textbf{TODO}} #1]}}
\newcommand\todo[1]{} % remove todos

%\newcommand\new[1]{\textcolor{new}{{#1}}}
\newcommand\new[1]{#1} % plain text
%\newcommand\old[1]{\textcolor{gray}{{\sout{#1}}}}
\newcommand\old[1]{} % remove old stuff

\definecolor{lgdv}{rgb}{.80,.23,.13}
\definecolor{agreen}{rgb}{.2,.8,.2}
\definecolor{jorange}{rgb}{.9,.6,.0}
\definecolor{new}{rgb}{.8,.4,.4}

\newcommand\NOTE[3]{\textcolor{#1}{[#2: #3]}}

%\newcommand\kai[1]{\NOTE{lgdv}{Kai}{#1}}
\newcommand\kai[1]{} % remove notes
%\newcommand\niko[1]{\NOTE{jorange}{Niko}{#1}}
\newcommand\niko[1]{} % remove notes

\newcommand\To{\ensuremath{\to}}
\newcommand\hi[1]{\textcolor{red}{#1}}

\let\shortcite=\cite
