Mögliche Kriterien:
The comparison investigates several aspects of each language, including program length, programming effort, runtime efficiency, memory consumption, and reliability
siehe:https://ieeexplore.ieee.org/abstract/document/876288

Ease of Coding, Debug/Test, Distribution/Upgrade, Ease of Use: Download, Install and Update, User Interface, Functionalities, Performance Comparison 
siehe:https://ieeexplore.ieee.org/abstract/document/8082717

-----
Vortrag zu Cross Plattform Notizen:

Flutter ist für Ubuntu Standardsprache zur Desktopanwendungsentwicklung

Mittlerweile für fast alle möglichen Plattformen übersetzbar und Gemeinschaft sehr aktiv.

Es gibt eigentlich keine richtigen Gründe mehr nativ zu entwickeln. Mit Flutter ist quasi alles umsetzbar, was man will und wenn es noch nicht das benötigte Plugin gibt, so kann man es selbst entwickeln. Das kann zwar etwas kompliziert werden, aber von der Sache her nicht ummöglich.

Bei Flutter kann man Plugins schreiben. Diese funktionieren, indem man einen Dart Code schreibt und die entsprechenden nativen Codeteile. Dadurch weiß der Dart-Compiler wie er den DartCode in die verschiedenen nativen Anwendungen übersetzt.

Die Grundfunktionalität von Flutter funktioniert, indem ein Canvas genommen wird, auf den die verschiedenen UI-Komponenten "drauf" gemalt werden. Mit einer dahinter liegenden Zuordnung und Tracking wird es dann zu einer funktionalen Nutzeroberfläche.

------------Frage: Ist es überhaupt noch sinnvoll, eine App nativ zu entwickeln? Macht es nicht immer mehr Sinn eine Cross-Plattform-Entwicklung zu tun, falls man einmal doch auf eine weitere Plattform umsteigen will? 
Es könnte ja immer sein, dass man die App erweitern kann. Dann müsste man eine zusätzliche Entwicklung bezahlen. Man müsste gewissermaßen komplett von vorne anfangen. Während wenn man bereits in Flutter etwa entwickelt hat, muss man nur noch die benötigte Plattform kompilieren und eventuell ein paar Sachen anpassen. 

Andererseits hat man einen schnell sich ändernden Technologiemarkt. Flutter wurde erst 2017 auf den Markt gebracht und ist 2021 zum Führenden Framework aufgestiegen. Andererseits ist React Native von einem der viel versprechendsten Frameworks seit dem Erscheinen von Flutter auf dem Abeseigenden Mast und andere haben ganz und gar Ihre Bedeutung verloren.Etwa Apache Flex. Das auch seit 2017 nicht mehr weiter entwickelt wird. So zeigt sich, dass die Wahl auf das richtige Framework und die richtige Entwicklungsstrategie eine sehr wichtige ist und mit viel Vorsicht getroffen werden muss.

-----
Flutter:
Flutter ist ein von Google entwickelte Programmiersprache die mittlerweile das bedeutendste Framework in der Cross-Plattform Entwicklung. Das spannende an dieser Plattform ist, dass nicht nur Google, eines der wichtigsten Technologieunternehmen dahintersteht, sondern ist ebenfalls sehr stark in der Entwicklung mit ständiger Weiterentwicklung. Es entstehen außerdem immer neue Tools und es werden mehr und mehr Plugins angeboten, um die Entwicklung einer Flutter App noch einfacher zu machen und dabei jede mögliche Funktionalität verfügbar anzubieten.

------
Vielleicht ist es gar nicht so sonnvoll feste Kriterien zu finden an denen man die Entwicklung an sich bewerten kann, sondern eher das Endergebnis. Es ist ja auch keine Bewertung sondern eher ein Vergleich.
Kriterien sind:
-Programmlänge
-Programmieraufwand
-Laufzeit effizienz
-Speicherplatzverbrauch
- verfügbarkeit

-Verfügbarkeit von Hilfestellungen und Fragen und Antworten

- Code lesbarkeit[1]
- Einfachheit[1]
- Datentypen[1]
- Syntax[1]
- Designed to make it impossible to making (some) stupid mistakes? [1]
- Programmierfähigkeiten während Entwicklung ohne ständiges neu Compilen und schnelles Testen     von Sachen[1]
- Possibilitys for reusability and dry code[1]
- A language may be wonderful and amazing and increase developer productivity, but if no talented people are out there that know the language, how will you hire the best team?[1]

- Support for internationalization [2]

- Keine Änderung von Grundsätzlichen Strukturen und Funktionalitäten.

- In general, we think you can write good software in Java, C\#, Python, PHP, or any other myriad languages. You can also write really bad software using any of those. To us, the adherence to widely accepted design principles and philosophies is more important than the specific language.[3]



Quellen hierzu:
-Medium Leseliste
-[1] https://cs.lmu.edu/~ray/notes/evaluatingprogramminglanguages/  , Loyola Marymount University website
-[2]https://easyexamnotes.com/p/language-evaluation-criteria-ppl.html
-[3] https://www.27global.com/how-to-choose-a-programming-language/
- [4]https://journals.plos.org/plosone/article?id=10.1371/journal.pone.0088941
- [5]\url{https://books.google.de/books?id=XUqqCAAAQBAJ&lpg=PR6&dq=criteria%20for%20evaluating%20programming%20languages&lr&hl=de&pg=PA35#v=onepage&q&f=false}

------
Framework Kriterien:

Quellen:
- [1] https://symfony.com/ten-criteria
- [2]


- Popularität und Community Größe: Je mehr genutzt und bekannt, umso mehr evolution und besserwertige und mehr Plugins und Lösungen [1]

- Philosophy: Ein von Profis für diesen Zweck entwickeltes Framework passt besser zu eigenem Projekt während Entwicklung für Framework für sehr spezielle Sachen eher unnützlich [1]

- Langlebigkeit: Wenn ein Framework kurz vor Ende ist oder schon lange keine Updates mehr erhalten hat, so wird es eventuell keine Große Hilfe im meistern zukünmftiger Probleme sein [1]

- Technische Details: Frameworks, die sich an die technischen Standards halten, helfen, nicht zu sehr an speziellen Lösungen eines Frameworks hängen zu bleiben und auch offen für neue Programmiersprachen und Frameworks zu bleiben [1]

- Sicherheit: Frameworks müssen bestimmte Sicherheitsrichtlienien erfüllen. [1]

- Gute Dokumentation: Eine gute Dokumentation durch den Entwickler ist entscheidend um verlässliche Informationen zu erhalten [1]

-Lizenzen: Bei Programmiersprachen wie Java gibt es einige Versionen die nur mit gekaufter Lizenz nutzbar sind. Darüber sollte man sich klar machen, bevor man eine Entscheidung trifft. Es ist grundsätzlich kein schlechtes Zeichen, muss allerdings beachtet werden. [1]

- Entwicklerverfügbarkeit [1]

-Ausprobieren:Wenn man nicht mit dem Framework zurecht kommt, sollte man eventuell ein anderes versuchen. Nur durch ausprobieren erfährt man wie gut man damit entwickeln kann.[1]
