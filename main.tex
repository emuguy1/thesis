% add option draft to skip images
\documentclass[12pt,a4paper,twosided,openany]{scrbook}
%\documentclass[12pt,a4paper,twosided,open=right]{scrbook}
% \documentclass[12pt,a4paper,twosided,open=right,draft]{scrbook}

% Entwickelt auf Basis von Carsten Kerns Template.

\usepackage[backend=biber,style=alphabetic,maxalphanames=60,maxbibnames=10]{biblatex}
% \usepackage[backend=bibtex,style=alphabetic]{biblatex}
% \usepackage{natbib}
\usepackage[ngerman]{babel}
\usepackage[utf8]{inputenc}
\usepackage{ifthen}
\usepackage{xargs}
\usepackage{amsmath}
\usepackage{amsfonts}
\usepackage{amssymb}
\usepackage{graphicx}
% \usepackage{fancyhdr}
\usepackage{tabularx}
\usepackage{geometry}
\usepackage{setspace}
\usepackage[right]{eurosym}
\usepackage[printonlyused]{acronym}
\usepackage{subfig}
\usepackage{floatflt}
\usepackage[usenames,dvipsnames]{color}
\usepackage{xcolor}
\usepackage{colortbl}
\usepackage{paralist}
\usepackage{array}
% \usepackage{titlesec}
\usepackage{parskip}
% \usepackage[subfigure,titles]{tocloft}
\usepackage[pdfpagelabels=true,colorlinks=true,linkcolor=black,anchorcolor=black,citecolor=black,filecolor=black,menucolor=black,runcolor=black,urlcolor=black]{hyperref}
\usepackage{booktabs}
\usepackage[lighttt]{lmodern}
\usepackage{mathabx}

\usepackage{listings}
\lstset{basicstyle=\footnotesize, captionpos=b, breaklines=true, showstringspaces=false, tabsize=2, frame=lines, numbers=left, numberstyle=\tiny, xleftmargin=2em, framexleftmargin=2em}

\geometry{a4paper, top=27mm, left=25mm, right=15mm, bottom=27mm, headsep=10mm, footskip=12mm}

% make lstinline be normalsize but keep displayed listings small
\makeatletter
\lstdefinestyle{mystyle}{
  basicstyle=%
    \ttfamily
    \lst@ifdisplaystyle\scriptsize\fi
}
\makeatother

\lstset{
	language=C++,
	% basicstyle=\ttfamily\scriptsize,
	style = mystyle,
	% basicstyle=\ttfamily\scriptsize,
	keywordstyle=\bfseries\ttfamily,
	% commentstyle=\color{LimeGreen}\ttfamily,
	commentstyle=\color{gray}\ttfamily,
	emphstyle={\color{Purple!90!black}},
	tabsize=4,
	morekeywords={nullptr,nullptr_t,vector,string,std,list,map,pair,set,size_t,endl,cout,move},
	emph={},
	showstringspaces=false,
}

\setcapindent{0pt}
\setkomafont{captionlabel}{\bfseries}

\DefineBibliographyStrings{ngerman}{
	andothers = {{et\,al\adddot}},
}


\DeclareLabelalphaTemplate{
  \labelelement{
    \field[final]{shorthand}
    \field{label}
    \field[strwidth=3,strside=left,ifnames=1,names=1]{labelname}
    \field[strwidth=1,strside=left,names=1-3]{labelname}
  }
  \labelelement{
    \field[strwidth=2,strside=right]{year}
  }
}
\renewcommand*\labelalphaothers{$^\Asterisk\mkern-.8mu$}

\newcommandx{\student}[3][]{
	\def\studentName{#1}%
	\def\studentMatnr{#2}%
	\def\studentStudiengang{#3}%
}

\newcommandx{\MyTitelseite}[8][]{
\thispagestyle{empty}
\ifthenelse{\equal{#1}{}}{
	\includegraphics[scale=0.2]{lib/oth-logo.png}
}{
	\includegraphics[scale=0.2]{lib/oth-logo.png}\hfill\includegraphics[scale=0.5]{#1}
}
\begin{center}
\ifthenelse{\equal{#2}{2}}{ % then
	\vspace*{2cm}
	\Large
	\textbf{Ostbayerische Technische Hochschule Regensburg}\\
	\textbf{Fakultät für Informatik und Mathematik}\\
	\vspace*{2cm}
	\Huge
	\textbf{#3}\\[1em]
	\large
	Zur Erlangung des akademischen Grades des\\
	\ifthenelse{\equal{#3}{Bachelorarbeit}}{Bachelor of Science (B.Sc.)}{Master of Science (M.Sc.)}\\
	\vspace*{1cm}
	\Large
	\textbf{#4}\\
}{ % else
	\vspace*{1cm}
	\Large
	\textbf{#4}\\
	\vspace*{2cm}
	\large
	An der Fakultät für Informatik und Mathematik der\\
	Ostbayerischen Technischen Hochschule Regensburg\\
	im Studiengang\\[2em]
	\textbf{\studentStudiengang}\\[2em]
	eingereichte\\
	\vspace*{1cm}
	\Large
	\textbf{#3}\\[2em]
	\large
	zur Erlangung des akademischen Grades des\\
	\ifthenelse{\equal{#3}{Bachelorarbeit}}{Bachelor of Science (B.Sc.)}{Master of Science (M.Sc.)}
	\vspace*{1cm}
	\Large
}
	\vfill
	\normalsize
	%\newcolumntype{x}[1]{>{\raggedleft\arraybackslash\hspace{0pt}}p{#1}}
	\begin{tabular}{rl}%{6cm}p{7.5cm}}
	    \rule{0mm}{1ex}\textbf{Vorgelegt von:} & \studentName \\
		\rule{0mm}{1ex}\textbf{Matrikelnummer:} & \hspace*{-0.5em}\begin{tabular}[t]{r}\studentMatnr\end{tabular} \\ 
		\ifthenelse{\equal{#2}{1}}{~\\}{\rule{0mm}{1ex}\textbf{Studiengang:} & \studentStudiengang \\[2em]}
		\rule{0mm}{1ex}\textbf{Erstgutachter:} & #5 \\ 
		\rule{0mm}{1ex}\textbf{Zweitgutachter:} & #6 \\[2em]
		\rule{0mm}{1ex}\textbf{Abgabedatum:} & #7 \\ 
	\end{tabular} 
\end{center}
\pagebreak
}



% ein paar nützliche Kommandos

\newcommand\defsec[1]{\label{sec:#1}}
\newcommand\refsec[1]{\ref{sec:#1}}

%\newcommand\todo[1]{\textcolor{red}{[{\textbf{TODO}} #1]}}
\newcommand\todo[1]{} % remove todos

%\newcommand\new[1]{\textcolor{new}{{#1}}}
\newcommand\new[1]{#1} % plain text
%\newcommand\old[1]{\textcolor{gray}{{\sout{#1}}}}
\newcommand\old[1]{} % remove old stuff

\definecolor{lgdv}{rgb}{.80,.23,.13}
\definecolor{agreen}{rgb}{.2,.8,.2}
\definecolor{jorange}{rgb}{.9,.6,.0}
\definecolor{new}{rgb}{.8,.4,.4}

\newcommand\NOTE[3]{\textcolor{#1}{[#2: #3]}}

%\newcommand\kai[1]{\NOTE{lgdv}{Kai}{#1}}
\newcommand\kai[1]{} % remove notes
%\newcommand\niko[1]{\NOTE{jorange}{Niko}{#1}}
\newcommand\niko[1]{} % remove notes

\newcommand\To{\ensuremath{\to}}
\newcommand\hi[1]{\textcolor{red}{#1}}

\let\shortcite=\cite

\addbibresource{bib.bib}

\begin{document}

% ----------------------------------------------------------------------------------------------------------
% Titelseite
% ----------------------------------------------------------------------------------------------------------
\newcommand{\stud}{Emanuel Erben}  % Ihr Name
\student{\stud}
{3174817}						% Matrikelnummer
{Informatik}					% Studiengang

\MyTitelseite{}					% Optionales Logo des extern betreuenden Unternehmens
% \MyTitelseite{pics/mathcomm}	% Optionales Logo des extern betreuenden Unternehmens
{1}								% Style der Titelseite (1 oder 2)
{Bachelorarbeit}				% Typ der Abschlussarbeit (\in {Bachelorarbeit, Masterarbeit})
{Vergleich verschiedener Ansätze der Multiplattform Applikationsentwicklung anhand einer Beispiel-Anwendung}				% Thema der Arbeit						
{Prof.\ Dr.-Ing.\ Kai Selgrad}		% Betreuer
{Prof.\ Dr.\ Name des Zweitgutachters}	% Zweitgutachter
{15.09.\the\year}				% Abgabedatum

\thispagestyle{empty}
~\pagebreak

\setcounter{page}{1} 

% ----------------------------------------------------------------------------------------------------------
% Abstract
% ----------------------------------------------------------------------------------------------------------
% \thispagestyle{empty}
\setstretch{1.15} % Zeilenspacing
\chapter*{Abstract}

\bigskip 

In den letzten Jahren nutzen viele Leute nicht mehr ihren Computer sondern vor allem ihre Smartphones und andere mobile Endgeräte, um auf verschiedenen Dienste, Konten und Online-Anwendungen zuzugreifen. Um eine Anwendung auf verschiedenen Plattformen veröffentlichen zu können haben Entwickler unterschiedliche Möglichkeiten. Diese reichen dabei von der Entwicklung einer eigenen Anwendung für jede einzelne Plattform bis hin zu einem Code der für die verschiedenen Plattformen genutzt werden kann. Dadurch ist die Wahl des richtigen Entwicklungsansatzes und der Technologie schwer, da jeder Ansatz seine Vor- und Nachteile hat. 

In dieser Arbeit werden vier verschiedene Applikationen, die mit unterschiedlichen Ansätzen implementiert wurden, beschrieben und analysiert. Danach werden die Ergebnisse der verschiedenen Untersuchungen vorgestellt und verglichen. Dabei wird darauf eingegangen, welche Vor- und Nachteile die einzelnen Implementierungen für die betrachteten Kriterien haben. Der Schwerpunkt der Arbeit liegt dabei auf der Android Plattform und speziell den Programmiersprachen Kotlin und Dart. Die vier Implementierungen sind eine native, hybride, cross-kompilierte und eine cross-kompilierte Applikation mit Webanteil, also einem Mix aus hybridem und cross-kompilierten Ansatz.

\frontmatter

% ----------------------------------------------------------------------------------------------------------
% Inhaltsverzeichnis
% ----------------------------------------------------------------------------------------------------------
\tableofcontents
\vfill
\pagebreak

% % ----------------------------------------------------------------------------------------------------------
% % Abbildungsverzeichnis
% % ----------------------------------------------------------------------------------------------------------
\listoffigures
\vfill
\pagebreak
% 
% % ----------------------------------------------------------------------------------------------------------
% % Tabellenverzeichnis (optional)
% % ----------------------------------------------------------------------------------------------------------
% \listoftables
% \vfill
% \pagebreak

% ----------------------------------------------------------------------------------------------------------
% Listingsverzeichnis (optional; Code nur, wenn wirklich sinnvoll und wichtig)
% ----------------------------------------------------------------------------------------------------------
%\lstlistoflistings
%\vfill
%\pagebreak


% ----------------------------------------------------------------------------------------------------------
% Inhalt
% ----------------------------------------------------------------------------------------------------------
\setstretch{1.15}


\mainmatter

\chapter{Motivation}
Wenn eine Anwendung entwickelt werden soll, ist die Wahl des richtigen Frameworks oder auch der Programmiersprache eine wichtige Entscheidung für das Projekt. Viele Entscheidungen können im Verlauf der Entwicklung noch abgeändert werden, aber um eine derartige Entscheidung zu ändern, müssen Teile der Anwendung oder auch der komplette Quellcode neu geschrieben werden. Neben Unsicherheiten bei der richtigen Wahl erscheinen regelmäßig neue Artikel, die über neue Standards in der Appentwicklung informieren wollen. So ist eine derartig wichtige Entscheidung oft sehr schwierig.

Da Anwendungen meistens nicht nur auf einem Gerätetyp, sondern allen Nutzern auf den verschiedensten Plattformen zur Verfügung stehen sollen, hat diese Entscheidung noch einmal eine größere Bedeutung, da sie eng verbunden mit den Kosten der Entwicklung und dem Nutzungserlebnis ist.

In Zeiten der Heimcomputer und des stetig wachsenden Internets war der einfachste Weg eine Webseite zu entwickeln, die über den Browser der genutzten Geräte aufgerufen werden konnte. Mit der Ära der Smartphones jedoch änderte sich das. So waren anfänglich Webapplikationen entweder nicht für die Nutzung an derartig Geräte mit einer Toucheingabe und kleineren Bildschirmen angepasst oder es musste eine getrennte Version für diese entwickelt und der Nutzer weitergeleitet werden\cite{Bryant2012}.

Die Nutzung von Smartphones ermöglichte allerdings die Nutzung andere Funktionalitäten, wie die Kamera, Bluetooth oder GPS-Daten. Deshalb wurde vermehrt auf die Entwicklung einer speziellen App für Smartphones gesetzt. Sie wurden traditionell nativ mit Objectiv-C, beziehungsweise Swift, für iOS und Java beziehungsweise Kotlin, für Android entwickelt \cite{ELKASSAS2017163},\cite{researchgate_thomas}. Durch die Programmierung mit Hilfe der nativen Programmiersprachen entstehen für die einzelnen Plattformen gut nutzbare Anwendungen \cite{researchgate_thomas}.

Durch den Wunsch nach einer Multi-Plattform Anwendung, also einer Anwendung, die für mehrere Plattformen veröffentlicht werden soll, muss bei der nativen Entwicklung für jede Plattform eine eigene Applikationen in der jeweiligen Programmiersprache geschrieben werden. Dadurch entsteht ein hoher Aufwand und die Kosten multiplizieren sich mit der Zahl der abzudeckenden Plattform.

Deswegen wurden bereits früh sogenannten Cross-Plattform Technologien entwickelt, die es ermöglichen, mit einem geteilten Code so viele Plattformen wie möglich abzudecken. So erschien etwa 2008 PhoneGap. Es war ein Open-Source Framework zur Entwicklung von hybriden mobilen Applikationen, die mit Hilfe von HTML, CSS und JavaScript implementiert wurden. PhoneGap und sein Nachfolger Cordova waren einige Zeit auch sehr beliebt in diesem Bereich und 2019 hatten beide zusammen immerhin einen Marktanteil von etwa 40\% unter den Cross-Plattform Frameworks \cite{statist_CP_Framework}.

Trotz der verringerten Kosten werden dennoch viele Applikationen immer noch nativ entwickelt. So ergab eine interne Untersuchung der Firma ScanBot SDK\footnote{https://scanbot.io/de/blog/native-apps-vs-cross-platform/}, dass 2019, 57\% ihrer Nutzer native Applikationen entwickelten, obwohl ihr Produkt ebenso für viele der gängigen Cross-Plattform Frameworks zur Verfügung steht.

Ein Grund für Unsicherheit in diesem Bereich sind oft auch negative Erfahrungsberichte von Firmen, die einen Umstieg von nativ zu Cross-Plattform-Applikationen versuchten.  So etwa auch Airbnb \cite{Airbnb_react_goals}. Sie nutzten das von Facebook mitentwickelte Framework React Native. 2019 hatte dieses Framework einen Marktanteil von 42\% \cite{statist_CP_Framework}. 

Ihre Ziele waren einfach \cite{statist_CP_Framework}:
\begin{enumerate}%This could be compactenum or inparaenum see paralist dokumentation%
    \item Schnelleres entwickeln.
    \item Die gleiche Codequalität beibehalten.
    \item Nur noch eine Codebasis.
    \item Die Entwicklungsabläufe verbessern.
\end{enumerate}
%%\nointerlineskip
Jedoch traten während der Entwicklung einige technische Probleme auf. So mussten etwa eine eigene Version des genutzten Frameworks erstellt werden, um notwendige Änderungen einzubauen. Dies erschwerte es jedoch Updates des Frameworks zu integrieren \cite{Airbnb_technology}. Die Entwickler von Airbnb erkannten deshalb, dass sie ihre Ziele nicht einhalten konnten und kehrten 2018 wieder zu einem nativen Ansatz zurück.

Durch Beispiele wie dieses, sind App-Entwickler oft skeptisch gegenüber derartigen Lösungen, da dies auch immer mit großen Änderungen, hohen Investitionen und einer  Einarbeitungszeit verbunden sind \cite{medium_Lehtimäki}. Dennoch nimmt die Zahl der Cross-Plattform Entwicklungen in den letzten Jahren zu. So ergab die Untersuchung von ScanBot SDK, dass 2021, gerade einmal zwei Jahre später, 58\% Cross-Plattform Lösungen nutzten, also ganze 15\% mehr. Diese Zahl stützt auch eine Untersuchung von Jetbrains \cite{JetBrains_miscellaneous_2021}. Diese ergab, dass 2021 53\% aller App Entwickler, derartige Technologien nutzten .

\begin{figure}[ht]
  \centering
  \includegraphics[width=15cm,keepaspectratio]{images/cross-platform-mobile-frameworks.png} 
  \caption[Statistik Cross-Plattform-Frameworks]{Cross-Plattform-Frameworks 2019-2021 \cite{statist_CP_Framework}}
  \label{fig:statista_cross_plattform}
\end{figure}

Diese Zahl dürfte in den nächsten Jahren weiter steigen. Einen Grund dafür bieten unter anderem auch neue Frameworks die entstehen. In Abbildung \ref{fig:statista_cross_plattform} ist eine Statistik zu sehen, die die Verteilung von verschiedenen Cross-Plattform-Entwicklungstechnologien zwischen 2019 und 2021 zeigt. Sie verdeutlicht wie schnell sich die Verteilung von Cross-Platform Frameworks ändern kann. Flutter ist dafür ein gutes Beispiel. Es ist ein Framework das erst 2017 auf den Markt gekommen ist und innerhalb von gerade einmal 4 Jahren auf einen Marktanteil von 42\% gekommen ist. Von einigen wird es als der neue Standard angesehen und einige Unternehmen steigen auf Flutter um. 

Jedoch gibt es auch Entwickler, die weiterhin nativ entwickeln. So etwa die App-Agentur Number42, mit der diese Arbeit in Zusammenarbeit entstanden ist. Sie entwickeln Web und mobile Applikationen. Alle der dort entwickelten mobilen Applikationen sind mit den nativen Programmiersprachen geschrieben. Einerseits entwickeln sich auch die nativen Programmiersprachen stetig weiter und erhalten Änderungen, die eine Entwicklung vereinfachen und beschleunigen. Andererseits gibt es auch Projekte, bei denen lediglich eine Plattform benötigt wird, da sie nur auf Geräten des Unternehmens ausgeführt werden sollen und diese beispielsweise nur Smartphones der Firma Apple nutzen. Deswegen ist auch dieser Ansatz nicht von der Hand zu weisen und es kann durchaus sinnvoll sein, neue Apps weiterhin nativ zu entwickeln.

\section{Beitrag dieser Arbeit}
Der Inhalt dieser Arbeit ist die Beschreibung vier verschiedener Implementierungen und deren Vergleich anhand unterschiedlicher Kriterien. Dafür werden Daten aus der Implementierung, Dokumentationen und anderen Quellen gesammelt und verglichen.
Um ein einheitliches Verständnis für die verschiedenen Applikationsklassen zu erhalten, werden diese zunächst vorgestellt, da es viele verschiedene Einordnungen und Klassifizierungen in wissenschaftlichen Arbeiten gibt.
Als Implementierungen werden eine native Android Applikation mit Kotlin, eine hybride Applikation mit Kotlin, eine Cross-kompilierte Applikation mit Flutter und einer gemischte Implementierung mit Flutter und WebView, vorgestellt. 
Weiterhin werden die vorgestellten Implementierungen auf die Kriterien Performance und Entwicklung, Entwicklergemeinschaft, Entwicklungsdauer, Benutzeroberfläche (UI) und Funktionalität untersucht.
Dabei soll analysiert werden, ob anhand der gesammelten Daten Anwendungsfälle identifiziert werden können, bei denen eine bestimmte Implementierung besser geeignet ist als eine andere. Weiterhin soll eine Einordnung und Erklärung gegeben werden, um eine Wahl des passenden Ansatzes zur Entwicklung einer Multi-Plattform-Anwendung zu erleichtern.


\section{Aufbau der Arbeit}
In Kapitel 2 werden zunächst verwandte Arbeiten vorgestellt, während in Kapitel 3 die verschiedenen Applikationsklassen, einige Begriffe und die Implementierungen vorgestellt und die Arbeit abgegrenzt.
Danach werden in Kapitel 4 die verschiedenen Implementierungen und einige Erkenntnisse während der Entwicklung erklärt. Dabei soll auch auf Stärken und Schwächen der einzelnen Implementierungen eingegangen werden, die während der Implementierung und Recherche aufgefallen sind.
In Kapitel 5 soll anschließend eine Auswertung der Implementierungen anhand einiger Kriterien stattfinden und mit weiteren Erklärungen und Vergleichen, eine Einordnung der verschiedenen Ansätze stattfinden. Abschließend soll in Kapitel 6 ein Fazit gezogen werden und ein Ausblick auf künftige Arbeiten gegeben werden.
\chapter{Verwandte Arbeiten}
Es gibt einige verschiedene Arbeiten die sich um das Thema Cross-Plattform beziehungsweise Multi-Plattform Entwicklung drehen. Die meisten Arbeiten sind dabei Veröffentlichungen im Rahmen von Konferenzen oder andere Wissenschaftliche Arbeiten. Im folgenden sollen einige vorgestellt werden und darauf eingegangen werden, was sie von dieser Arbeit unterscheidet.


\begin{figure}[ht]
  \centering
  \includegraphics[width=\textwidth,keepaspectratio]{images/IEEE_Delia_Al.png}
  \caption{Ergebnisstabelle der Performancemessungen von Delia et al \cite{IEEE_development_classes}}
  \label{fig:result_table_IEEE_related_work}
\end{figure}

Die meisten Arbeiten drehen sich um den Faktor Performance. Hier gibt es viele unterschiedliche Untersuchungen. Delia et al \cite{IEEE_development_classes} etwa testete diese anhand von komplexen mathematischen Berechnungen und zeichnet dabei die verstrichene Zeit für iOS und Android auf. Dabei berechneten sie außerdem die Standardabweichung um zu untersuchen, wie gestreut die Ergebnisse sind. In Abbildung \ref{fig:result_table_IEEE_related_work} können diese betrachtet werden. Dabei stellte sie zwar einen Performance Unterschied zwischen Android und iOS fest, dieser ist jedoch ihrer Meinung nach, eher auf die unterschiedliche Hardware der Testgeräte zurückzuführen. Sie konnten aber insgesamt eine gute Performance von Web Applikationen feststellen. Die nativen Applikationen schnitten besonders unter Android deutlich schlechter ab als ein Großteil der restlichen Untersuchten Ansätze, jedoch auf iOS war dieser Ansatz bei ihnen der am besten performende.

In dieser Arbeit werden alle Tests auf einem Gerät durchgeführt um eine Vergleichbarkeit zu erlangen. Außerdem wird die Arbeit lediglich auf Android beschränkt, um mehr Fokus auf die unterschiedlichen Ansätze als auf die Unterschiede zwischen den Plattformen einzugehen. Außerdem werden in der Arbeit neben Performance viele weitere Faktoren untersucht. 
 
Denko et al \cite{Denko_performance} vergleichen nicht nur die verbrauchte Zeit für mathematische Berechnungen sondern analysieren zusätzlich die Auslastung der Geräte während verschiedenen Aufgaben und die Appgröße der kompilierten Apps. Die jeweiligen Parameter vergleichen sie dabei anhand drei Unterschiedlicher Geräte. Dabei konnten sie im am Ende keine eindeutige bessere Methode identifizieren, da oft je nach untersuchten Aspekt, ein anderer Ansatz besser war als ein anderer, dies aber bei einem anderen Faktor wieder vertauscht sein könnte. So war etwa bei der Ausführung von Algorithmen die Native Implementierung die schnellste, während Flutter hier eher als einer der schlechtesten abschnitt. Bei der Nutzung der CPU oder der Generationszeit einer compilierten Anwendung war Flutter dafür wieder besser als die native Android Implementierung. Sie kommen daher auf den Schluss, dass kein Ansatz in Sachen Performance der beste ist und deswegen eher andere Faktoren betrachtet werden müssen um eine fundierte Entscheidung zu treffen.

Wie auch in dieser Arbeit geplant, wurden hier zwar einige weitere Aspekte der Entwicklung in den Vergleich mit einbezogen, jedoch soll in dieser Arbeit die anderen Aspekte eine weit größere Rolle spielen, denn wie dieses Paper zeigt, kann anhand der Performance keine eindeutige Entscheidung getroffen werden.

In der Arbeit von  Andersson \cite{Andersson_2022} vergleicht er die Performanceunterschiede zwischen einer nativ entwickelten Android App und einer Cross-Plattform-App, die mit dem Flutter Framework geschrieben wurde. Dabei untersuchte er 6 verschiedene Funktionalitäten, die in vielen Apps vorkommen, so etwa Animationen, unendliche Listen oder auch Datenbankoperationen. Diese Funktionalität testete er und zeichnete dabei neben der Ausführungszeit auch die CPU und Speicherauslastung des Gerätes auf. Das Ergebnis war, dass auch er keinen eindeutig besseren Ansatz finden konnte, so war in seinem Vergleich Flutter deutlich schneller und Ressourcenschonender bei der Decodierung von Dateien. Dafür zeigte sein Test auch eine deutlich schlechtere Performance bei Animationen im Gegensatz zu der nativen Implementierung. Seine Schlussfolgerung ist dabei, dass Flutter Apps und dementsprechend Cross-Plattform Frameworks keine schlechtere Performance als native haben, es jedoch auf den genauen Anwendungsfall ankommt, welches Framework eine bessere Performance aufweist.

Auch in der Arbeit von Andersson wird wieder eine ähnliche Performance der untersuchten Technologien festgestellt, daher sollen in der folgenden Arbeit, wie bereits erwähnt, mehr Faktoren einbezogen werden. 

\begin{figure}[ht]
  \centering
  \includegraphics[width=10cm,keepaspectratio]{images/Biorn-Hansen_Result_table.jpg}
  \caption{Ergebnisstabelle der Untersuchung von Biørn-Hansen et al \cite{BirnHansen.2020}}
  \label{fig:result_table_Biorn}
\end{figure}

Biørn-Hansen et al \cite{BirnHansen.2020} konzentrieren sich in ihrer Untersuchung vor allem auf die Nutzung Plattformspezifischer Schnittstellen, wie etwa der Geo-Daten oder Kontakte \ac{API}. Die Untersuchten Maße sind dabei neben der Auslastung der Geräte vor allem die Zeit, bis eine Aufgabe abgeschlossen ist. Bei ihrer Auswertung der 5 Kriterien gewichteten sie jeden, der 6 implementierten Ansätzen, anhand einer Zahl von 1 - 6, wobei 6 die Implementierung bekommt, die am besten in diesem Bereich abgeschlossen hat und 1, die die schlechtesten Ergebnisse lieferte. Danach wurden die einzelnen erreichten Punkte zusammengerechnet und die Implementierung mit der höchsten Punktzahl ist die, die in ihrem Vergleich am besten abgeschlossen hat. Wie in Abbildung \ref{fig:result_table_Biorn} zu sehen ist, erreichte Nativ dabei eine Punktzahl von 24 und somit den 1. Platz in ihrer Auswertung. Das Cross-Plattform Framework Flutter hingegen landete auf dem vorletzten Platz mit gerade einmal 15 Punkten. In ihrem Fazit halten sie dementsprechend fest, dass die Nutzung von anderen Technologien zu einer Performanceverschlechterung führen kann, jedoch einige Frameworks in gewissen Bereichen besser abschneiden als die native Implementierung und es somit auf die genauen Anforderungen ankommt.

In ihrer Arbeit bewerten Biørn-Hansen et al die verschiedenen Ansätze. Dies kann dabei zu einer Verzerrung der Ergebnisse führen, da in einigen Kategorien die Unterschiede nur sehr gering waren, dies jedoch nicht in der Bewertung berücksichtigt wird. Deswegen soll in der Arbeit von einer Bewertung abgesehen werden und anhand von Kriterien Empfehlungen beziehungsweise Tendenzen zu gewissen Technologien aufgeführt werden.

Raj und Tolety \cite{IEEE_Rahul_Seshu} gehen in ihrer Arbeit einen etwas anderen Ansatz. Sie stellen in ihrer Arbeit die wichtigsten Grundlagen und stellen eine Einteilung von Applikationen in vier unterschiedliche Kategorien vor. Diese sind Applikationen,
\begin{itemize}
    \item die hauptsächlich eine Anzeige von Server Daten sind.
    \item die durch Nutzereingaben oder Sensoren Daten erhalten und diese verarbeiten,
    \item die ein eigenständiges System sind und keinerlei Verbindung zu einem Server oder anderen benötigen.
    \item die einen hohen Kommunikationsgrad mit einem Server haben.
\end{itemize}
Sie empfehlen dabei, dass Entwickler den Entwicklungsansatz anhand der Appkategorie wählen. So empfehlen sie etwa, dass bei viel Kommunikation mit einem Server, der Web-gestützte Ansatz gewählt wird, da eine Änderung auf dem Server nur an die Geräte geschickt werden muss, ohne die eigentliche Applikation zu aktualisieren.Sie sagen jedoch auch, dass Cross-Plattform Lösungen gut sind, wenn mehrere Plattformen abgedeckt werden sollen, da dadurch Entwicklungszeit und Kosten gespart werden können.

Die meisten ihrer Erklärungen sind nachvollziehbar, jedoch fehlen in ihrer Arbeit Zahlen oder gewisse Faktoren, um ihre Empfehlung zu stützen. Dazu wird in ihrer Auswertung die Klasse der Nativen Applikationen vernachlässigt, dabei stellt diese eine der wichtigsten Klassen im Bereich der Applikationsentwicklung dar. Deshalb soll in dieser Arbeit anhand von konkreten Zahlen und Quellen ein Vergleich zwischen den implementierten Ansätzen gezogen werden, um eine nachvollziehbar Entscheidung zu unterstützen.

Eine weitere Arbeit, die sich nicht nur auf die Messung der Performance beschränkt, ist die Arbeit von Olsson \cite{Olsson_2020}. Sie untersucht neben der Performance vor allem die Oberfläche und deren Benutzung. Dafür implementiert sie eine native Applikation jeweils für iOS und Android und eine Cross-Plattform Flutter App, die auf beiden Plattformen installiert werden kann. Ihre Performance Untersuchung zeigt, wie die anderen bereits erwähnten Arbeiten auch, dass die Cross-Plattform Applikationen eine geringere Performance als die nativen Implementierungen haben, stellt aber auch fest, dass diese nicht sehr groß sind. Als zweiten Teil der Untersuchung befragt sie eine Gruppe an Entwicklern und zeigt ihnen in unterschiedlichen Anwendungsfällen die zwei unterschiedlichen Applikationen auf einem Android Gerät. Danach fragt sie nach Unterschieden und Auffälligkeiten zwischen den zwei Applikationen, ohne dass die Testpersonen wissen, welche Applikation welche ist. Das Ergebnis war, dass die Mehrheit keinen Unterschied zwischen den zwei Anwendungen finden konnten. So antworteten etwa 75\% auf die Frage welche der Applikationen die native ist, dass sie kein Unterschied sehen konnten. Weitere 10\% identifizierten die Flutter App als native und gerade einmal 12,8\% konnten die native Applikation richtig identifizieren. Dabei waren Animationen noch der Faktor, bei dem am meisten Befragte einen Unterschied feststellen konnten. Der dritte Faktor ihrer Arbeit ist die Programmlänge. Bei dieser hatte Flutter die geringste Anzahl an Zeilen. Insgesamt sagt sie, dass Flutter und damit Cross-Plattform Tools eine gute Alternative sein können, dies jedoch abhängig ist von der weiteren Entwicklung der Frameworks.

Die Arbeit von Olsson ist sehr spannend, da sie eine umfassendere Untersuchung zwischen den zwei betrachteten Ansätzen darstellt, als die bisherigen Arbeiten. Jedoch ist die Arbeit lediglich auf zwei Ansätze beschränkt und bietet so nur einen guten Vergleich zwischen den zwei Ansätzen.

Eine letzte Arbeit die hier noch genannt werden soll ist die Arbeit von Khachouch et Al \cite{IEEE_Khackouch_Al}. Das Ziel ihrer 2020 vorgestellten Arbeit war es, einen Entscheidungsgraphen zu erstellen. Durch die Beantwortung einiger Fragen, sollen Entwickler neuer Applikationen, einfach die für ihr Projekt passende Entwicklungsmethodik finden. Sie stellen dabei fest, dass es zwar einige solcher Graphen bereits gibt, diese jedoch oft unausgeglichen sind.
Als erstes suchten sie deswegen nach passenden Fragen, die eine Eingrenzung erlauben und ordneten diesen ein Gewicht zu um die Höhe der Frage in dem Graph festzulegen. Sie stellen am Ende fest, dass wenn es keine Einschränkungen in den Bereichen Entwicklungszeit und Kosten gibt, der native Ansatz aufgrund der besseren Performance und Qualität die beste Lösung bleibt. Durch ihren Graphen wird jedoch auch klar, dass oft andere Ansätze sinnvoller sein können.

Der von Kachouch et al erarbeitete Graph liefert innerhalb von 2 bis 8 Fragen immer eine eindeutige Antwort, welcher Ansatz der Beste ist. In dieser Arbeit hingegen soll lediglich eine Tendenzen aufgezeigt werden, da jedes Projekt durch seine einzelnen Funktionalität sehr komplex ist und nicht mit wenigen Fragen exakt eingeordnet werden kann. Stattdessen soll ein besseres Verständnis erreicht werden, welche Faktoren eine Entscheidung beeinflussen können. 
\chapter{Grundlagen}
Bevor genauer in die Arbeit eingestiegen werden kann, müssen jedoch zunächst erst einmal ein paar Begriffe geklärt, die Ausgangssituation dargestellt und das Thema etwas abgesteckt werden.


\section{Begriffe}
\subsubsection{App/Applikation/Anwendung}
Im Rahmen dieser Arbeit wird oft von einer App oder Applikation geredet. Damit ist eine Anwendung gemeint, die für mobile Endgeräte entwickelt wurden. Die am häufigsten genutzten mobilen Geräte sind dabei Smartphones mit einem Marktanteil von etwa 62,6\%\footnote{https://gs.statcounter.com/os-market-share}.

\subsubsection{Plattform}
Der Begriff Plattform kann auf unterschiedliche Betriebssysteme, Prozessortypen, Kommunikationsprotokolle oder Hardwaresysteme bezogen werden \cite{2014Mulit_plattform_definition}. In dieser Arbeit werden damit die unterschiedlichen Betriebssysteme bezeichnet. Beispiele für diese sind Android, iOS oder auch MacOS.

\subsubsection{Multi-Plattform-Anwendung}
Als Multi-Plattform-Anwendung bezeichnet man jene Anwendung, die auf mehreren Plattformen ausgeführt werden kann und dabei eine gleiche oder ähnliche Funktionalität hat \cite{2014Mulit_plattform_definition}. Eine Anwendung muss dabei nicht für alle Plattformen implementiert sein. Ist eine Anwendung auf mehr als einer Plattform verfügbar so kann diese als Multi-Plattform-Anwendung bezeichnet werden.

\subsubsection{Cross-Plattform-Anwendung}
Cross-Plattform-Anwendungen sind ebenfalls, wie Multi-Plattform Anwendungen, auf mehr als einer Plattform ausführbar. Der Unterschied besteht darin, dass Cross-Plattform-Anwendungen eine gemeinsame Codebasis besitzen, die es ermöglicht, die Applikation nur einmalig implementieren zu müssen, jedoch mehrere Plattformen damit abdecken zu können \cite{2014_Cross_plattform}.

\subsubsection{GraphQL}
\TODO{Eventuell GraphQL erklären, vlt Annotation}

\section{Die verschiedenen Appentwicklungs Framework Klassen}
Für die Programmierung von Multi-Plattform-Applikationen gibt es verschiedene Klassen, in die die Entwicklungsmethoden eingeordnet werden können. Dabei unterscheiden verschiedene Autoren unterschiedliche Klassen. In dieser Arbeit wird daher der von Delia et al \cite{IEEE_development_classes} definierten Einteilung gefolgt, da ihre Definition ein vernünftiger Pfad zwischen zu detailliert und zu generell ist. Im Folgenden sollen diese vorgestellt und auf einige Aspekte der einzelnen Klassen eingegangen werden.

\subsection{Native Applikationen}
Native Applikationen werden entwickelt, um auf einer bestimmten Plattform installiert zu werden. Der Quellcode wird dafür zu ausführbaren Code übersetzt, der spezifisch für die gewählte Plattform ist \cite{IEEE_development_classes}.
Die Programmierung wird in der für die Plattform typischen Programmiersprache geschrieben und ist dadurch nur für eine Plattform nutzbar. Es gibt folglich für jede Plattform einen eigenen Quellcode. Um etwa eine native Android App zu entwickeln, wird diese in Kotlin programmiert und im Anschluss in Kotlin-Bytecode übersetzt. Dieser Bytecode ist daher nur auf Android Geräten ausführbar.

Der Vorteil der nativen Entwicklung ist, dass man die Funktionen der verschiedenen Plattform optimal nutzen kann. So ist etwa eine Nutzung der Kamera, GPS, Beschleunigungssensoren, Kalender und vielem mehr sehr einfach. Es gibt eindeutig definierte Schnittstellen und diese müssen nur aufgerufen werden. Dabei ist die Ausführung nicht nur schnell, sondern kann auch einfach im Hintergrund ausgeführt werden \cite{IEEE_development_classes}. Dazu kommt, dass das Aussehen und die Benutzerschnittstellen ähnlich zu dem Gesamtsystem sind. So entsteht für den Nutzer ein geschlossenes System, das leichter zu bedienen ist, da es keine Unterschiede in Struktur, Design, Aufbau oder auch Benutzung gibt \cite{IEEE_Khackouch_Al}.

Einer der größten Nachteile der nativen Entwicklung jedoch ist der Aufwand und die damit verbundenen Kosten, um für die verschiedenen Plattformen eine Applikation anbieten zu können. Denn die Applikation muss für jede Plattform komplett neu gebaut werden. So können die Kosten für eine Plattform mit der Anzahl der abzudeckenden Plattformen multipliziert werden, um die Gesamtkosten zu erhalten \cite{IEEE_Khackouch_Al}. Doch nicht nur die Programmierung ist ein Kostenfaktor. So nennen Delia et al das Testen, Warten und Verteilen neuer Version als Faktoren, die auf jeder einzelnen unterstützten Plattform auftreten \cite{IEEE_development_classes}. Dazu kommt, dass man für jede Plattform auch Entwickler benötigt, da sich die wenigsten Entwickler auf allen Plattformen auskennen und Anwendungen für die verschiedenen Plattformen oft auch gleichzeitig entwickelt werden sollen. Um große Kosten zu verhindern, werden deswegen häufig nur  eine oder zwei Plattformen ausgewählt, wodurch die Reichweite der Anwendung sinkt.

\subsection{Web-Applikationen}
Web-Applikationen sind Applikationen, die im Netz verfügbar sind. Sie sind darauf ausgelegt, als Webseiten auf einem Server zu laufen und dann über den Browser der Geräte aufgerufen zu werden. Dieser Ansatz ist simpel, da eine Webseite sofort für jeden Nutzer verfügbar ist, sobald sie auf dem Server gestartet wurde. Sie muss auch nur einmal entwickelt werden, da sie auf allen Geräten mit einem Browser und einer Internetverbindung aufgerufen werden kann. So kann man alle Plattformen mit nur einer Entwicklung abdecken \cite{IEEE_development_classes}.

Wie gerade erwähnt, wird ein Code für alle Plattformen geschrieben. Dies ist natürlich ein großer Pluspunkt, falls die Entwicklungskosten stark eingeschränkt sind. Außerdem stehen dem Nutzer Updates direkt nach einem Neustart des Servers zur Verfügung, da es nicht erst an die Geräte verteilt und dann installiert werden muss \cite{IEEE_Khackouch_Al}.

Jedoch hat man hier große Einschränkungen in der Funktionalität, da lediglich die Funktionen des Browser zur Verfügung stehen. So können derartige Anwendungen die nativen Schnittstellen nicht benutzen und sind in ihrer Funktionalität stark beschränkt \cite{Phyo}. Dazu kommt, dass falls keine Internetverbindung vorhanden ist, die Anwendung gar nicht genutzt werden kann und bei einer langsamen Internetverbindung die Performance signifikant sinkt \cite{IEEE_Khackouch_Al}. Außerdem müssen Webanwendungen auch angepasst werden, um gut auf einem mobilen Endgerät genutzt werden zu können. So sind Smartphones etwa höher als sie breit sind, während Computerbildschirme breiter sind als sie hoch sind. Dementsprechend muss etwa eine Menüleiste darauf reagieren können und je nach Seitenverhältnis beziehungsweise Auflösung ein anderes Design anbieten. Dadurch muss Zeit investiert werden, um Webseiten zu entwickeln, die auf allen Geräten gut nutzbar sind.

Web-Applikationen sind zwar gerade in den letzten Jahren durch die Verfügbarkeit von schnellen Internetverbindungen in fast allen Gebieten der Welt, immer einfacher nutzbar geworden.
Im Schnitt stieg die mobile Bandbreite zwischen 2017 und 2021 um 59,5\% beziehungsweise von knapp 20 \ac{Mbps} auf 55\ac{Mbps} \footnote{https://www.ookla.com/articles/global-index-2019-internet-report}.
Jedoch ergab eine Studie von Yoram Wurmser \cite{report_webusage}, dass während der Smartphonebenutzung etwa 89\% der Zeit in Applikationen und gerade einmal 11\% in einem Browser verbracht wird.
Da also ein deutlich höherer Fokus auf tatsächlich auf dem Gerät installierbaren Anwendungen liegt, wird diese Klasse in der Arbeit nicht genauer betrachtet.

\subsection{Hybride Applikationen}
Eine Klasse die viel mit der vorherigen Klasse zu tun hat und in dieser Arbeit genauer betrachtet werden soll, sind die hybriden Apps. Sie nutzen Web-Technologien, also HTML, Javascript und CSS, werden jedoch nicht im Browser des Geräts aufgerufen, sondern in einem nativen Container ausgeführt \cite{IEEE_development_classes}. die zu einem Teil aus nativem Code bestehen und zu einem anderen Teil aus einem Web-Container in dem Inhalte entweder direkt aus dem Netz angezeigt werden, oder lokal gespeicherte Seiten gezeigt werden. Sie benutzen also Web-Technologien, benutzen dabei aber nicht den Browser, sondern einen eigenen Web-Container. Unter Umständen kann dieser mit Hilfe von bestimmten API-Schnittstellen auf gerätespezifische Funktionen zugreifen \cite{IEEE_development_classes}.
\TODO{Vielleicht noch ein Bild zur Architektur einbauen siehe: https://fh-bielefeld-mif-sw-engineerin.gitbooks.io/klausurthemen/content/fullstack-development/Hybride\_AppEntwicklung/WebApps\_NativeApps\_HybrideApps.html}

 

Grundsätzlich hat diese Klasse die gleichen Vor- und Nachteile wie die vorherige, da sie zu einem großen Teil aus einer Web-Applikation bestehen, die in einer nativen App angezeigt werden. Es gelten hier jedoch einige Ausnahmen, auf die auch noch einmal genauer in dem Kapitel mit der Implementierung zu dieser Klasse eingegangen wird. Denn je nach Art von hybrider App werden einzelne Probleme durch spezielle Implementierungen gelöst.Was jedoch immer ist, ist dass in den meisten Benutzeroberflächen weiterhin keine nativen Komponenten genutzt werden können und durch das Laden von Container und Website die Performance sinkt \cite{IEEE_development_classes}. Des weiteren kann die Anwendung auch weiterhin nicht im vollen Umfang genutzt werden, wenn keine Internetverbindung hat. Man hat allerdings auch eine extrem schnellere Entwicklungszeit bei bestehender Web-Anwendung, abhängig vom Umfang der nativen Integration.

\subsubsection{Cross-Plattform Applikationen}

Die letzte Klasse sind Multi-Plattform Applikationen. Sie zeichnen sich dadurch aus, dass jeweils nur ein Code geschrieben wird und am Ende eine Applikation für mehrere Plattformen entsteht.\TODO{Das hier entfernen} 

Sie haben den großen Vorteil, dass sie in der Entwicklung deutlich billiger sind als wenn man eine Anwendung für jede Plattform eine eigene Schreiben würde. Auch die Entwicklungszeit wird dadurch oft deutlich verkürzt. Durch die Devise, dass nur ein Code geschrieben wird, treten hier auch seltener Ungereimtheiten bei der Definition und Benutzung der Schnittstellen zu etwa einem Server. Auch Logikfehler treten wenn dann bei allen Anwendungen auf und nicht nur bei bestimmten. Dies klingt zwar im ersten Moment erstmal negativ, ist aber insofern gut, da Fehler dadurch leichter reproduzierbar sind und auch bereits beim testen sofort auffallen sollten, da man lediglich einen Code testen muss.

Nachteile die für diese Entwicklungsform sind oft Frameworkspezifisch. So ist etwa ein Problem, dass manchmal nicht alle Plattformfunktionen ohne Einschränkungen genutzt werden können. Des weiteren ist bei manchen ein Kritikpunkt, dass die UI nicht Geräte-typisch aussieht. Diese Probleme sind jedoch auch Abhängig von der Art, wie die Anwendung übersetzt wurde. Im folgenden soll deswegen in zwei Gruppen unterschieden werden.

\subsection{Interpretierte Cross-Plattform Anwendungen}
Bei Interpretieren Cross-Plattform Anwendungen schreibt der Entwickler einen Code, der wird dann mithilfe von nativen Code während der Laufzeit in ausführbaren Code übersetzt. Das bedeutet, dass die auf dem Gerät installierte Anwendung einen nativen Teil, oftmals Frameworkcode zum Übersetzen, und einem Cross-Plattform Teil, der Anwendungslogik, besteht \cite{IEEE_development_classes}.

Wie man bereits vermuten kann ist ein Problem dieser Klasse, dass bei der Ausführung eine Zwischenschicht vorhanden ist, die für zusätzliche Latenz der Ausführung führt.

\subsection{Compilierte Cross-Plattform Anwendungen}
Das Problem der zusätzlichen Latenz kann durch diese Klasse verhindert werden, weswegen auch nur diese in der Arbeit untersucht wurde. Denn bei kompilierten Cross-Plattform Anwendungen wird zwar während der Entwicklung nur ein Quellcode geschrieben, jedoch wird nach der Entwicklung eine Anwendung für jede Plattform gebaut, die nur aus nativen Teilen besteht \cite{IEEE_development_classes}.


\section{Themenabgrenzung}
In dieser Arbeit sollen nicht die verschiedenen Programmiersprachen oder Frameworks innerhalb der einzelnen Ansätze verglichen werden, sondern viel mehr die Ansätze gegenseitig. Daher wird für eine Cross-Plattform Implementierung das Framework Flutter\footnote{https://flutter.dev/} genutzt, da dies das aktuell am meisten genutzte Framework\cite{statist_CP_Framework} ist und somit einen guten Repräsentanten für diese Gruppe darstellt. Für die Android Implementierungen wurde als Programmiersprache Kotlin\footnote{https://kotlinlang.org/} genutzt. 

Die nativen Implementierungen werden in dieser Arbeit auf eine Android Implementierung beschränkt. 
Einerseits sind iOS und Android vergleichbar, da beide nativ auf die vollständigen Funktionalitäten der Betriebssysteme zugreifen.
Dabei sind es nicht die verschiedenen Programmiersprachen und ihre Kleinigkeiten die entscheidend sind, um native Apps zu entwickeln, sondern die Konzepte.
So sagen Goadrich et al \cite{iOSvsAndroid} in einer Untersuchung, welche Plattform für einen Universitätskurs passend wäre, dass beide Umgebungen die für den Kurs benötigten Funktionalität haben.
Weiter sagen sie, dass Studenten dadurch praktische Erfahrung in der Applikationsentwicklung bekommen würden. Sie kommen am Ende auf den Schluss, dass es kein Unterschied macht welche Plattform dabei behandelt werden würde und beide Plattformen helfen, eine Grundlage zur Lehre der Grundideen für mobile Endgeräte Programmierung bilden würden.
Auch in Sachen Performance gibt es zwischen den Plattformen keine größere Unterschiede. So zeigt eine Untersuchung von Győrödi et al \cite{Android_IOS_Performance_comparison}, dass es zwar kleinere Unterschiede in der Performance gibt, es aber keine Plattform gibt, die insgesamt gesehen besser ist, als die andere.
Desweiteren stellt Android die deutlich größere Nutzergruppe dar. So besagen aktuelle Statistiken, dass Android aktuell einen Marktanteil von etwa 70\% hat, während iOS nur auf 30\% kommt \cite{statist_OS_worldwide}.  

Die betrachteten Implementierungen umfassen ebenfalls keine Spieleimplementierung. Diese stellen zwar einen signifikanten Teil der in den Appstores vorhandenen Anwendungen dar, jedoch sind dies keine typischen Apps die von Appagenturen oder privaten Entwicklern produziert werden, sondern eher von Unternehmen mit Grafik- oder Spieleentwicklungshintergrund. Außerdem werden die meisten Apps nicht in Kotlin oder Swift direkt entwickelt, sondern mit Game- und Grafikprogrammen wie Unity oder ähnlichem gebaut. 



Der Schwerpunkt der Arbeit liegt außerdem auf Smartphone-Applikationen. Zwar sind in der Klasse der mobilen Endgeräte auch Laptops mit den Betriebssystemen Windows, MacOS oder die verschiedenen Linux distributionen, dennoch werden Applikationen primär für den Smartphone Markt entwickelt, während für mobile Rechner dann eher ein Programm schreibt.
\TODO{Warum auf Smartphone bei Untersuchung beschränkt?}


\section{Projektbeschreibung}
Als Basis für diese Arbeit wird eine bereits bestehende Elixir-Web-Anwendung genutzt. Hierbei handelt  es sich um eine Plattform, die das Ziel hat, das Verleihen und Leihen innerhalb von Bekanntschaftskreisen zu vereinfachen/ ermöglichen. Hierfür kann jeder Nutzer seine eigenen, verleihbaren Gegenstände auf der Plattform eintragen. Zusätzlich können Nutzer sogenannte Kreise erstellen und zusammen mit Freunden bzw. Familie beitreten. Jeder kann dann die Gegenstände sehen, die in den verschiedenen Kreisen vorhanden sind, in denen er Mitglied ist. Zur Kontaktaufnahme gibt es ein Chatsystem, bei dem sich Leute Nachrichten hin und her schicken können, um den Austausch zu organisieren. Zur Kommunikation mit einer Applikation besitzt das System eine GraphQL-Schnittstelle, über die alle benötigten Daten abgefragt werden können. Auch kann der Nutzer über die Schnittstelle authentifiziert werden und neue Nachrichten versendet.

\section{Funktionsumfang der Beispiel Anwendung}
Grundsätzlich wurde beim Entwurf des Funktionsumfang versucht, die typischen Funktionalitäten von Applikationen abzubilden. Eine Befragung mobiler Anwendungsentwickler durch JetBrains ergab, dass die wichtigsten Funktionen Datenspeicherung, Kommunikation über Netzwerk, Medienanzeige, Status und Navigationsmanagment, Datensynchronisierung, Dateien lesen/schreiben, Sicherheit, Bezahlung, Berechnungen und Machine Learning sind\cite{JetBrains_miscellaneous_2021}. Natürlich sind gerade Machine Learning oder die Bezahlung eine sehr Anwendungsfallspezifische Sache, jedoch gibt diese Liste einen guten Eindruck, was eine typische App an Funktionalität benötigt.
Um den Arbeitsaufwand realistisch zu halten und trotzdem aber einige der oben genannten Parameter abzudecken, ist die Implementierung wie im Folgenden beschrieben eingeschränkt.

Allgemein soll eine App gebaut werden, die sich an einer bestehende Webanwendung orientiert und einen Teil der Funktionalität durch die Programmierung abbilden soll. Des weiteren wird eine GraphQL-Schnittstelle genutzt um die Daten der Webanwendung zu nutzen. 

\subsection{Nativer und Cross-kompilierter Ansatz}
Bei der Nativen und der Cross-kompilierten Applikation wird der in Abbildung \ref{fig:pageflow} dargestellten Ablauf abgebildet. Dabei sind diese komplett durch in der Applikation implementierten Seiten dargestellt. Diese sind eine Start-, Login- , Profil- , Kommunikations- und eine Chatseite. Dadurch werden die oben genannte Aspekte bis auf Bezahlung und Machine Learningaben abgedeckt. Denn durch den Login etwa erfüllt die Applikation teilweise die Bereiche Sicherheit, Statusmanagment, Daten lesen/schreiben und Kommunikation über Netzwerk. 

\begin{figure}[ht]
  \centering
  \includegraphics[height=7cm,keepaspectratio]{images/Pageflow_native_flutter.png} 
  \caption[Seitenablauf der implemierten nativen und Cross-kompilierten Applikation]{Verbindungen zwischen den Seiten der implementierten Applikation des nativen und des Cross-kompilierten Ansatzes}
  \label{fig:pageflow}
\end{figure}

Der genaue Ablauf der App ist dabei, dass der Nutzer die App öffnet. Wenn er bereits eingeloggt ist, wird er automatisch auf die Profil Seite übergeleitet, auf der eine über die Schnittstelle von einem Server abgefragte Liste angezeigt wird. Ist er jedoch nicht angemeldet, so landet er auf einer Startseite/ Willkommensseite. Über ein Menü erreicht der Nutzer die Loginseite, von der er nach erfolgreicher Anmeldung zu der bereits erwähnten Profilseite kommt. Über einen Knopf in der Menüleiste der App kann der eingeloggte Nutzer auf die Konversationsseite wechseln. Hier wird im eine Liste an Konversationen angezeigt, an denen er beteiligt ist. Durch die Auswahl einer Konversation gelangt er auf eine Unterhaltungsseite, auf der er die bereits bestehenden Nachrichten lesen und neue Nachrichten abschicken kann. Während der Nutzer angemeldet ist, kann er sich außerdem abmelden und damit wieder auf die Startseite weitergeleitet werden. Dabei werden die Zugangs- und andere Personendaten gelöscht.

\subsection{Hybrider und gemischter Ansatz}
Bei dem hybriden Ansatz wird lediglich die bereits bestehende Website in eine native programmierte Applikation eingebunden. Sie stellt dabei die Aufgabe nach, die bei diesem Ansatz eigentlich ein eigenes Framework übernehmen würde. Sie ist jedoch durch die spezifische Implementierung der Webseite stark in der Funktionalität eingeschränkt. Beim letzten Ansatz, dem gemischten Ansatz geht dies deutlich weiter. Hier wird zwischen der Webseite und mit einem Cross-kompilierten Framework erstellten Seiten hin und her geschaltet. Dabei wird insbesondere der Login die Profilseite und das Chat System ersetzt, während der Rest der Applikation aus der Webseite besteht. Außerdem wird in dieser Implementierung die Navigation der Webseite durch eine neu programmierte Navigation ersetzt. Daher ist dieser Ansatz eine Mischung aus dem hybriden und dem Cross-kompilierten Ansatz und somit der letzten Applikationsklasse zuzuordnen.



\chapter{Entwicklung}
Im 4. Kapitel dieser Arbeit werden die vier verschiedene Implementierungen erklärt, Dabei wird zu jedem Ansatz einige Grundlagen, genutzte externe Bibliotheken, Herausforderungen und ein Fazit zu der Implementierung vorgestellt. Dadurch sollen die Implementierungen reproduzierbar werden und Erkenntnisse während der Entwicklung geteilt werden. Wie in Kapitel \label{cha:3_3abgrenzung} bereits erklärt, wird dabei lediglich  auf eine Android Implementierung eingegangen, wenn nicht automatisch andere Plattformen unterstützt werden. 

\section{Entwicklung Nativer Android Applikation}
Die native Entwicklung ist die ursprünglichste von allen Arten. Android etwa wurde 2008 vorgestellt. 
Damals wurden die Apps für Android noch in Java entwickelt, eine Sprache die in der Anwendungsentwicklung damals und heute noch sehr gut bekannt ist und auch oft noch als Programmiersprachen an den Universitäten gelehrt wird. 
Jedoch änderte Google die offiziell bevorzugte Programmiersprache 2019 zu Kotlin\footnote{https://developer.android.com/kotlin}. 
Kotlin wurde von Jetbrains entwickelt, um einen Ersatz für Java zu finden. 
Sie entwickelten eine Sprache die alle benötigten Funktionen für eine effektive Appentwicklung hat, jedoch genauso schnell kompiliert werden kann. 
Eine ähnliche Entwicklung fand auch bei Apple statt, bei der von Objectiv-C zu Swift gewechselt wurde. 
Durch die Entwicklung eigener Programmiersprachen, haben die Plattformentwickler die Kontrolle, welche Funktionalitäten hinzugefügt oder entfernt werden und können diese perfekt für ihre Bedürfnisse anpassen \cite{medium_Swift_Kotlin}.

\subsection{Grundlagen}
Die Entwicklung einer nativen Android App besteht aus zwei separaten Teilen, 

Der erste Teil ist das Layout. Dabei wird das Design für die UI und eventuelle Elemente mit Hilfe der XML-Notation erstellt.
In der XML Datei wird ähnlich zu einer HTML Datei, die Oberfläche aufgebaut. Als Wurzelelement hat eine Seite dabei ein Layoutelement. Neben einem linearem Layout oder einem Tabellenlayout, gibt es hier vor allem das so genannte Constraint-Layout\footnote{https://developer.android.com/guide/topics/large-screens/support-different-screen-sizes}. 
Es ist ein wichtiges Layout in Android, da, wie der Name bereits andeutet, die Positionen der Elemente abhängig von anderen Elementen definiert ist. Durch diese Eigenschaft, eignet es sich, um Designs für unterschiedliche Bildschirmgrößen zu erstellen \cite{ConstraintLayout_Android}. Es bietet dabei nicht nur die Möglichkeit, die Position, sondern auch die Größe abhängig von anderen Elementen zu definieren.
Die eigentliche Benutzeroberfläche wird durch Verschachtelung und Anordnung von UI-Elementen innerhalb des Grundlayouts gebaut. Diese können dabei sowohl von Android vordefiniert oder von externen Bibliotheken stammen. Die erstellten Dateien beinhaltet lediglich das Layout und keinerlei Funktionalität.

Der zweite Teil der Implementierung ist der Logik und Funktionalitätsteil der Anwendung. Dieser kann wiederum in drei Klassen unterteilt werden. 
Die erste Klasse, die sogenannte Activity\footnote{https://developer.android.com/reference/kotlin/android/app/Activity} ist die Hauptklasse für eine Seite. In ihr wird das Layout aufgerufen und der aktuellen Seite hinzugefügt, den Elemente der UI ihre Funktionalität zugeordnet, in der Oberfläche die benötigten Daten ergänzt und vorhandenen Listen ein Controller hinzugefügt \cite{sarkar_android}.
Diese Controller, genannt ListAdapter\footnote{https://developer.android.com/reference/androidx/recyclerview/widget/ListAdapter} sind die zweite Klasse. Sie werden bei dynamischen Listen in Android benötigt, da diese so designt sind, dass Listenobjekte die außerhalb der sichtbaren Bereiches sind, wiederverwendet werden \cite{recyclerview_android}. Dafür benötigt jede Liste einen eigenen Adapter, der das benötigte Design, Daten und Funktionalität dem Listenelement hinzufügt.
Die dritte Klasse sind ViewModels\footnote{https://developer.android.com/topic/libraries/architecture/viewmodel}. Diese sind die Schnittstelle zwischen der Datenhaltung und der Activity und verringern damit die Koppelung von Benutzeroberfläche und der Datenhaltung \cite{viewModel_android}. Neben dem asynchronen Sammeln von Daten aus ihren Quellen, kann ein ViewModel außerdem Daten speichern, die eine Konfigurationsänderung überstehen sollen. Knofigurationsänderungen sind dabei Änderungen, die einen Neustart der Activity nych sich ziehen\footnote{https://developer.android.com/guide/topics/resources/runtime-changes}. Dazu zählt etwa das Ändern der Bildschirmausrichtung. ViewModels überstehen eine solche Änderung und können nach dem erneuten Bauen der Aktivität dieser wieder hinzugefügt werden. Sie werden erst beendet, wenn die dazugehörige Activity geschlossen wird. Dadurch können sowohl Daten zwischengespeichert werden, als auch verhindert werden, dass Daten unnötig oft abgefragt werden.


\subsection{Genutzte Bibliotheken}
Zur Kommunikation mit der GraphQL-Schnittstelle der genutzten Webanwendung, wird Apollo Kotlin\footnote{https://www.apollographql.com/docs/kotlin/} genutzt. 
Es ist ein GraphQL Client, der geschriebenen GraphQL Queries in Kotlin und Java Modelle umwandelt, die daraufhin wie normale Objekte in der Implementierung aufgerufen werden können.
Anfänglich stellt die Bibliothek während der Entwicklung eine Verbindung mit der GraphQL-API her, um die Schnittstellendefinition abzufragen und zu speichern. Dadurch kann der Client die Antworten der Schnittstelle, die in einem JSON Format geliefert werden, automatisch den jeweiligen Datentypen zuordnen. Daher ist der Zugriff auf die Antwort des Servers besonders sicher möglich, da Schreibfehler in der Abfrage der Daten verhindert werden können.
Damit der Client in der kompletten Applikation genutzt und wenn nötig abgeändert werden kann, wird er der Applikation als globales Attribut hinzugefügt.

Die zweite wichtige Bibliothek die genutzt wird, ist Room\footnote{https://developer.android.com/training/data-storage/room}. Sie ist Teil von Android Jetpack\footnote{https://developer.android.com/jetpack}, einm Set an Bibliotheken, dass Entwickler helfen soll, Apps für verschiedene Android-Versionen problemlos zu bauen. Room selber ist dabei der Teil, der es ermöglichen soll einen möglichst effizienten Datenbank Zugriff zu haben.
Die Besonderheit ist, dass Room eine Zwischenebene darstellt, die die geschriebenen SQL Befehle überprüft und validiert. Des Weiteren wird durch die Nutzung von Annotationen redundanter Code verhindert und eine Asynchronität der Datenbankoperationen vereinfacht \cite{Room_docs}.

Zum Speichern des Authentifizierungsstrings wurden sogenannte Shared-Preferences\footnote{https://developer.android.com/reference/android/content/SharedPreferences} genutzt. Sie sind ein weiterer Teil von Android Jetpack. Sie werden genutzt, um Key-Value Werte ohne großen Aufwand innerhalb der Applikation zu speichern. Sie können von innerhalb der ganzen Applikation abgefragt hinzugefügt, geändert oder gelöscht werden. Hierfür muss keine eigene Entität in der Datenbank angelegt werden und der Zugriff ist schneller. Sie sind jedoch nur für einfache Daten geeignet, da lediglich primitive Datentypen genutzt werden können. Daher werden sie vor allem für Werte wie Authentifizierung oder bestimmte Flags geeignet um den Status der App zu bestimmen.


\subsection{Fazit Android Nativ}
Der native Ansatz ist sehr gut dokumentiert. Für fast jedes Probleme, wie etwa der GraphQl Kommunikation, kann eine erprobte und gut funktionierende Lösung gefunden werden. Dazu kommt, dass es viele Bibliotheken gibt, die Google beziehungsweise JetBrains selber anbieten, um die Entwicklung von Apps zu vereinfachen und zu standardisieren.

Ein Vorteil der Layoutentwicklung in diesem Bereich ist, dass das Layout in einer sofortigen Anzeige zusammengebaut werden kann und somit das Aussehen sofort ersichtlich ist. Dies ist dabei sowohl bei der nativen Android als auch der nativen iOS Entwicklung möglich. Jedoch ist diese Anzeige nur auf die Oberfläche ohne Funktionalität oder von der Activity hinzugefügte Daten beschränkt. Wenn also Änderungen betrachtet oder getestet werden sollen, so muss die App dennoch neu gebaut werden. Neben der Zeit, die es dauert die Änderungen zu laden, startet die App dabei wieder vom Startbildschirm neu, weshalb wieder zu der entwickelten stelle navigiert werden muss, um Änderungen zu testen. Dadurch wird der Entwicklungsprozess verlangsamt und erschwert,


\section{Entwicklung einer hybriden Android Applikation mittels eines WebView-Containers}
Der zweite realisierte Ansatz ist eine hybride Applikation, die mit Kotlin implementiert wurde. 
Da einerseits die Implementierung der Web-Komponente durch viele verschiedene Ansätze umgesetzt werden kann und andererseits in diesem Fall bereits eine Webseite existiert, wird folglich nur auf die Implementierung der Schnittstelle eingegangen. 
Um ein besseres Verständnis über die Funktionalität des hybriden Ansatzes zu erhalten, wird auf die Nutzung eines Frameworks verzichtet und die Schnittstelle zwischen der eingebundenen Webseite und der Plattform nativ entwickelt.
Dabei wird die grundlegende Funktionalität ermöglicht. Eine umfänglichere Nutzung von Plattformfunktionalität wird in einem Exkurs vorgestellt.

\subsection{Grundlagen}
Um eine Minimalversion zu erhalten, wird eine native Applikation benötigt, die eine WebView-Komponente\footnote{https://developer.android.com/reference/android/webkit/WebView} enthält. Um die Funktionalität der Webseite zu ermöglichen, müssen unter Anderem die folgenden Aspekte beachtet werden.

So benutzen rund 98\% aller Webseiten JavaScript in ihrem Code, um zum Beispiel ein Menü ein oder auszublenden, ohne die Seite neu laden zu müssen \footnote{https://w3techs.com/technologies/details/cp-javascript}. 
Dazu muss jedoch JavaScript in der WebView aktiviert werden. 
Jedoch kann die Aktivierung von JavaScript ein Sicherheitsrisiko darstellen. So können etwa fremde Webseiten ebenfalls versuchen, auf das Gerät zuzugreifen \cite{webview_javascript_security}. 
Daher sollte die Implementierung externe Links erkennen und diese in einem externen Browser-Fenster öffnen.
Die Links können dabei entweder in dem normalen Browser des Gerätes geöffnet werden oder in einem CustomTabsIntent\footnote{https://developer.android.com/reference/androidx/browser/customtabs/CustomTabsIntent}.
Dadurch entsteht ein Browsertab in der eigenen Applikation, dass etwa in der Akzentfarbe der Applikation angepasst werden kann um eine Zugehörigkeit erkennbar zu machen, jedoch dennoch einen externen Browser zu erhalten. Es ist jedoch durch die angezeigte Funktionsleiste mit angezeigter URL am oberen Bildschirmrand dennoch als externer Inhalt erkennbar.

Um dies zu ermöglichen muss jedoch die aufgerufene URL abgefangen werden und entschieden werden, wie mit der aufgerufenen URL verfahren werden soll. So ist eine einfache Grundkonfiguration, dass alle URLs die zur angezeigten Webapplikation gehören zugelassen werden, während externe Links weitergeleitet werden.

Eine letzte Konfiguration, die benötigt wird um etwa das Login mit Google zu ermöglichen ist das Setzen eines UserAgentStrings. Er ist eine Text Flagge, die in einem HTTP-Paket zu finden ist. Sie beinhaltet Informationen zu dem benutzten Client oder der genutzten Hardware. Außerdem wird sie auch genutzt, um verdächtige Pakete heraus zu filtern\cite{UserAgentString}. So filtert Google LogIn-Anfragen herraus, bei denen der UserAgent nicht gesetzt ist. Dieser kann durch das Setzen der Variable auf dem Webcontainer definiert werden.

Mit diesen Konfigurationen ist eine grundsätzliche Funktionalität einer Webapplikation sichergestellt. So kann an diesem Punkt theoretisch eine erste Version veröffentlicht werden. Jedoch ist an diesem Punkt dem Nutzer immer noch sichtbar, dass es sich bei dem Inhalt um eine Webseite handelt. Um dies zu verhindern können über die Layoutdatei native Elemente über oder um das WebView anlegen. So kann etwa ein Home Button dem Layout hinzugefügt werden, um der Anzeige native Elemente hinzuzufügen.

Neben der Konfiguration des Aussehens kann die WebApplikation noch in der benutzung von Gerätefunktionalität angepasst werden. Auf eine derartige mögliche Erweiterung soll deswegen im Folgenden Exkurs etwas eingegangen werden.

\subsection{Exkurs: Nutzung von Plattformfunktionalität}
Wie in Kapitel \ref{cha:3_hybrid} erwähnt, haben hybride Applikationen die Möglichkeit auf Plattformfunktionalität zuzugreifen. Dies ist der Teil, den Frameworks typischer Weise bereits umgesetzt haben und um die ein Entwickler sich normalerweise keine Gedanken machen muss. Jedoch wird in dieser Arbeit kein Framework genutzt, um einen Möglichen Ansatz zu zeigen, dies in die eigene Applikation zu integrieren. So kann ein von der App initialisiertes und der WebView hinzugefügtes WebAppInterface\footnote{https://developer.android.com/guide/webapps/webview\#BindingJavaScript} direkt aus dem JavaScript der Webseite aufgerufen werden \cite{webview_javascript_security}. Dafür werden Objekte und ihre öffentlichen Methoden dem Webcontainer unter einem definierten Namen zur Verfügung gestellt. Es können dabei sowohl Daten gesendet als auch empfangen werden.

Auch kann aus der Applikation JavaScript Code aufgerufen beziehungsweise ausgeführt werden. Dafür wird mit der loadUrl Methode, die in der WebView normalerweise einen Link aufruft, JavaScript-Code mit dem Präfix \verb|"javascript:"| der WebAnwendung hinzugefügt \cite{webview_javascript_security}. Dies kann sowohl selbst geschriebener Code als auch ein Aufruf bereits definierter Methoden sein. Dies zeigt jedoch, dass auch innerhalb der Webanwendung ein besonderes Augenmerk auf die Sicherheit gesetzt werden muss, da unter Umständen neben der Applikation, auch die Webseite angreifbar ist. So könnten bösartige Applikationen versuchen, über die JavaScript Schnittstelle eigenes JavaScript auszuführen, um so etwa den aufgerufenen Webserver zu attackieren. Dies ist möglich, da JavaScript-Code, der über loadUrl hinzugefügt wurde, im aktuell angezeigten Kontext gleichwertig wie von von der Webanwendung stammendes JavaScript ausgeführt wird \cite{webview_javascript_security},

\subsection{Fazit hybride Implementierung}
Ein Vorteil dieser Implementierung ist die Entwicklungsgeschwindigkeit, die erreicht werden kann, wenn bereits eine Webanwendung existiert. Anstatt die Anwendung komplett neu zu schreiben und somit multiple Implementierungen der Anwendungslogik zu erstellen, kann mit diesem Ansatz die Webanwendung in eine native Applikation eingebettet werden.
Ein weiterer Vorteil ist, dass durch die Nutzung der extern gehosteten Webanwendungen, Änderungen am Inhalt der angezeigten Anwendung nicht erst auf den Geräten der Nutzer installiert werden müssen, sondern direkt nach einem Neuladen der Webseite bei allen Nutzern verfügbar ist. 
Dazu kommt, dass an der eigentlichen Applikation nur sehr selten Änderungen durchgeführt werden müssen, da hier in den meisten Fällen nicht viel Logik verbaut ist.

Ein Nachteil ist, dass in diesem Fall die App nur nutzbar ist, wenn eine aktive Internetverbindung besteht, da der Großteil der angezeigten App, aus der Webseite bestehen. Dazu kommt, dass der Programmier- und Wartungsaufwand steigt, wenn eine umfangreiche Nutzung von Gerätefunktionalität geplant ist. Des Weiteren wird bei dieser Implementierungsart wieder eine eigene Implementierung des nativen Containers für jede Plattform benötigt., dies kann jedoch durch die Nutzung eines Frameworks verhindert werden. Ein Problem für diese Art der Entwicklung sind jedoch die Review Guidelines des Apple AppStores\footnote{https://developer.apple.com/app-store/review/guidelines/\#minimum-functionality}. Diese legen fest, dass die App nicht einfach nur ein Container für eine Webseite sein darf, sondern zu großen Teilen aus für den Nutzer nützlichen Funktionalitäten bestehen muss. Dadurch ist es wahrscheinlich, dass eine Implementierung im Umfang der hier dargestellten Applikation von Apple nicht veröffentlicht werden würde, wenn nicht noch zusätzliche Funktionalität hinzufügt wird.

Einige der Vor und Nachteile sind jedoch stark von der hier vorgestellten Implementierung abhängig. So wird etwa durch die Nutzung einer lokal gespeicherten Implementierung eine offline Funktionalität erreicht, jedoch müssen dadurch Änderungen über den normalen Updateprozess der einzelnen Plattformen verteilt werden.
Auch kann die gesamte Implementierung durch die Nutzung von Frameworks erledigt werden, jedoch muss die Webseite dann an dieses Framework angepasst werden und die nutzbare Funktionalität durch die implementierten Lösungen eingeschränkt sein. 


\section{Entwicklung Cross-Plattform Applikation mit Flutter}
Der dritte Ansatz der hier erklärt werden soll, ist eine Cross-Plattform-Implementierung mit Hilfe des Flutter Frameworks.

\subsection{Flutter Grundlagen}
Flutter ist ein 2017 von Google veröffentlichtes Framework das mittlerweile zum bedeutendsten Cross-Plattform-Framework geworden ist\cite{statist_CP_Framework}. Neben viel Aufmerksamkeit von Entwicklern haben auch Firmen großes Interesse an Flutter. So hat etwa die Firma hinter Ubuntu, Canonical, angekündigt, dass alle von ihnen entwickelten Anwendungen, in Flutter programmiert werden\cite{Ubuntu_Flutter}. Als Programmiersprache wird dabei Dart benutzt, die je nach Plattform, mit unterschiedlichen Compilern übersetzt wird,

Eine Grundlage von Flutter ist \verb|Composition over inheritance|. Dieses Konzept ist besonders gut sichtbar beim Aufbau der Nutzeroberfläche. Alles ist ein Widget. Bis auf die Geschäftslogik ist alles, wie etwa Gestenerkennung, Layouthelfer und tatsächlich Angezeigte Elemente ein Widget\cite{Thiele_2018}. Dabei werden Widgets baumartig verschachtelt, wobei die meisten Widgets wieder ein oder mehrere Kinderelemente haben, die ein Widget sind. In Abbildung \ref{fig:flutter_layout_tree} etwa ist der Widget-Baum einer Menüleiste zu sehen. Dabei sind Container Elemente, die um Widgets gebaut werden, um weitere Sachen wie Abstände oder anderes hinzufügen. Nur bei Widgets wie in diesem Fall etwa Text oder Icon, stoppt der Baum, da diese keine weiteren Kinder haben.

\begin{figure}[ht]
  \centering
  \includegraphics[height=7cm,keepaspectratio]{images/sample-flutter-layout.png} 
  \caption{Hierachie einer Menüleiste Quelle\protect\footnotemark}
  \label{fig:flutter_layout_tree}
\end{figure}

\footnotetext{https://docs.flutter.dev/development/ui/layout}

Anders als bei der nativen Implementierung wird in Flutter das Layoutfile mit der Funktionalität dahinter gemischt. Das bedeutet, dass die Funktionalität direkt beim definieren eines Knopfes zugeordnet wird. Dafür muss lediglich das Widget in den Baum eingefügt werden und danach das entsprechende Attribut gesetzt werden. Dadurch ist alles an einem Ort.

Da Flutter einen großen wert auf Perfomance legt, ist eine Änderung auf der \ac{UI} so gebaut, dass nur Elemente neu gebaut werden, die ihren Inhalt geändert haben. Dafür hat Flutter für \verb|Stateful Widgets|, die eine extra State Klasse besitzen. In diesem werden Daten gespeichert und sobald er sich ändert, registriert Flutter die Änderung und baut daraufhin das geänderte Widget und die in der Hierachie folgenden neu\cite{9623025}. Dadurch kann häufiges und vor allem vollständiges Neubauen des Widgetsbaums verhindert werden. Jedoch benötigt nicht jedes Element einen State. Wenn etwa bereits zum Zeitpunkt des Erzeugens des Widgets, alle Daten feststehen und auch nicht mehr geändert werden, dann wird kein State benötigt und es reicht ein \verb|Stateless Widget| für die Implementierung\cite[Kapitel~4]{Flutter_Recipes}. Dadurch wird einerseits die Implementierung eines States ersparrt und bei einem Neubau des Baums, kann das Widget-Element wieder genutzt werden und muss lediglich neu gerendert werden, was bei Flutter jedoch sehr performant ist.

Um sich wiederholenden Code zu sparen, kann in Flutter einfach ein eigene Widget geschrieben werden, dass im gesamten Projekt wie jedes andere Element auch, in den Widget-Tree eingefügt werden kann. Das Erstellen eines Widgets ist dabei der selbe Prozess wie das erstellen einer Seite. Es wird ein Knoten Element gewählt und dann mit anderen Elementen kombiniert, um die gewünschte Oberfläche zu erhalten. Außerdem muss entschieden werden, ob es ein Stateful oder Stateless Widget ist. Je nachdem muss noch ein zusätzlicher State implementiert werden.


\subsection{benutzte Plugins}
Das erste Plugin dass zu erwähnen ist, ist \verb|graphql\_flutter|\footnote{https://pub.dev/packages/graphql\_flutter}. Es ist ein Packet, dass ähnlich zu dem bereits erwähnten Apollo GraphQL einen Client zur Kommunikation mit einer GraphQL Schnittstelle. Damit der Client in allen Pages und Widgets verfügbar ist, muss die Konfiguration als Wurzel der Applikation gesetzt werden. Danach kann entweder über ein Widget oder über eine programmierte Funktion Die verschiedenen GraphQL Funktionen ausgeführt werden. Es ist ein Open Source von einer Community geschriebenes Plugin. Dies ist auch in der Nutzung spürbar. So sind einige Funktionalitäten nicht so sehr ausgereift wie bei der Apollo Implementierung. So muss etwa auf die einzelnen Felder der Antwort mit der JSON Zugriff \verb|Antwort.data[\"Feldname\"]|. Die Gefahr ist dabei hoch, dass durch ein Tippfehler der Zugriff schief läuft. Durch die Entwicklung von einer Community ist es allerdings einfacher Hilfe zu bekommen. So wurde während der Entwicklung ein paar Fragen auf dem dafür eingerichteten Discord-Server eingestellt, die professionell innerhalb einiger Stunden beantwortet wurden.

Das zweite wichtige Plugin ist Hive\footnote{https://pub.dev/packages/hive}. Es ist ein einfacher und schneller Key-Value Speicher, der Daten auhc über einen Neustart der App hinaus, in einer Datei innerhalb des Applikationsordner speichert. Er wird dafür genutzt, um den Identifikationsstring für den Server zu speichern. Es ist vergleichbar mit den \verb|SharedPreferences| der nativen Entwicklung vergleichbar, ist jedoch umfangreicher, da auch Objekte mit der richtigen Konfiguration gespeichert werden können und ist beim lesen vergleichbar beziehungsweise beim Schreiben schneller.

Ein drittes Plugin das hier noch erwähnt werden sollte ist \verb|simple\_gradient\_text|. Es ist ein Paket um ein Text mit Farbverlauf in der Schrift zu haben. Es ist also lediglich ein Paket, das zur Erstellung der Oberfläche genutzt wurde und ist dementsprechend hier nicht so wichtig. Jedoch zeigt sich hier wieder der Vorteil an einer aktiven Community. Bei der Installieren des Paketes gab es anfänglich einige Probleme, da es auf Android funktionierte aber auf anderen Plattformen nicht. Nach einer Unterhaltung mit dem Entwickler über ein erstelltes Problem auf GitHub ergab sich, dass das Plugin eine höhere Minimalversion von Dart benötigte, als in meiner Konfiguration eingestellt. Dadurch konnte bei mir der Fehler einfach behoben werden. Daraufhin wurde die Dokumentation des Paketes sowohl auf GitHub und dem zentralen Plugin Verzeichnis aktualisiert und um die entdeckte Anforderung erweitert.

Als letzte wichtige Bibliothek, wurde \verb|sqflite\_common\_ffi|\footnote{https://pub.dev/packages/sqflite\_common\_ffi} für die Implementierung der Datenbank genutzt. Es ist eine Bibliothek, dass auf Basis von \verb|sqlite3| eine Datenbankimplementierung für alle Plattformen anbietet.
Diese wird beim Start der Anwendung geöffnet und ist danach in der kompletten Anwendung verfügbar.
Mit ihre können alle gängigen \ac{CRUD}-Operationen durchgeführt werden. Die Implementierung ist dabei denkbar einfach.
Zum erstellen der Tabelle muss lediglich der SQL Befehl zum erstellen ausgeführt werden und danach kann über dedizierte \verb|insert| und \verb|query| Methoden die Daten gespeichert beziehungsweise abgefragt werden.
Nebenbei kann jede Art von SQL-Befehl ausgeführt werden, um Befehle auszuführen, die über die normale Implementierung hinaus gehen.


\subsection{Exkurs: Platform spezifische Funktionalität entwickeln}
Im Verlaufe der Entwicklung kann es vorkommen, dass eine gewisse Funktionalität, die essentiell für die Anwendung ist, bisher nicht implementiert wurde oder die Verfügbaren Bibliotheken nicht die gewünschte Funktionalität hat oder gewisse Plattformen nicht unterstützt werden. Für diesen Fall kann eigener Plattformcode geschrieben werden. Dafür wird einerseits der benötigte Dart Code geschrieben und dann der jeweilig notwendige Plattformcode. Zur Kommunikation zwischen den Verschiedenen Teilen werden dabei Plattform-Channels genutzt.\cite[Kapitel~12.3]{Flutter_Recipes}

Durch die Nutzung eines Kanals zur Kommunikation erlaubt es die Ausführung des Plattform Codes auf einem separaten Thread. Dadurch kann dieser asynchron ausgeführt werden und eine Blockade des Threads auf dem die \ac{UI} ausgeführt wird, verhindert werden\cite{plattform_code_flutter}.

Um weiteren Entwicklern die entwickelte Erweiterung zur Verfügung zu stellen kann außerdem das Plugin in das offizielle Repository von Flutter hochladen.

Es ist also möglich benötigte Funktionalität hinzuzufügen, solange sie auf den einzelnen Plattformen möglich ist. Jedoch werden für eine derartige Entwicklung das nötige Wissen benötigt, um den Plattformcode zu schreiben.

\subsection{Exkurs: Firebase}
Ein weiterer Aspekt, warum Flutter gerade bei kleineren App-Projekten gern genutzt wird, ist die umfangreiche Integrationsmöglichkeit von Firebase. Eine aktuelle Untersuchung ergab, dass in Android Applikationen die eine Analyse Software integriert haben, 92\% der weltweiten Applikationen, Firebase nutzt \cite{statist_analytics_SDK}.

Firebase ist eine Backend-as-a-service Lösung, die dabei helfen soll, schnell und einfach Anwendungen zu entwickeln und zu betreiben\footnote{https://firebase.google.com/}. Es ist also eine Sammlung von verschiedensten Tools und Plugins, die einem Entwickler die Möglichkeit geben soll, sich auf Design und Funktionalität der App konzentrieren zu können. Es umfasst Tools wie eben Analyse Software um Nutzungsdaten zu sammeln und zu analysieren, ein fertiges Chatsystem oder auch eine Cloud gestützte Datenbank Lösung. So schreiben Guzzi et Al, dass entweder ein Team an Entwickler angestellt werden müssten, das in monatelanger Arbeit ein Backend System programmieren würde, um dann eine Schnittstelle zu einem Backend zu bauen. Andererseits kann auch ein bereits bestehendes System gennommen werden. Mit Firebase werden tausende Zeilen Code eingespart und erhält dabei die Möglichkeit, asynchrone Aufrufe und nebenläufige Prozesse für eine reaktive App zu nutzen \cite[p.~608]{Flutter_Apprentice}.

Um eine Datenbank für seine App zu erstellen sind es wenige Schritte. So kann auf der Webseite von Firebase eine neue Datenbank angelegt werden. Danach wird die erzeugte Konfigurationsdateien für die jeweiligen Plattformen heruntergeladen. Diese werden den jeweiligen Implementierungen hinzugefügt. Danach müssen die Datenstrukturen lokal in der App entworfen werden und eine Verbindung hergestellt. Dafür werden die Daten-klasse mit JSON Konvertierungen, ein \ac{DAO} mit einer Methode zum speichern und holen der Daten und zuletzt noch einen Provider erzeugt. Außerdem besteht die Möglichkeit, neben einer online Datenbank ebenfalls eine offline Version hinzuzufügen, die synchronisiert wird, sobald eine Internetverbindung hergestellt wird \cite{Flutter_Apprentice}.

Ein weiterer Pluspunkt ist die Integration von Firebase-Authentifizierung. Es bietet eine fertige Lösung zur Authentifizierung von Nutzern. Es besteht außerdem die Möglichkeit, Benutzer zu kategorisieren beziehungsweise die Nutzung einzuschränken. So kann über Regeln in der Firebase-Website, präzise definiert werden, welcher Nutzer auf welche Daten zugreifen kann. Weiter können Nutzer auch von der Nutzung des Dienstes ausgeschlossen werden oder nur bestimmte Emailadressen zugelassen. Es bietet also ein vielseitig und umfangreiches Nutzerverwaltungstool \cite{Flutter_Apprentice}.

Firebase ist natürlich nicht nur für Flutter verfügbar, sondern genauso für die verschiedensten Programmiersprachen der Plattformen Android, iOS oder auch Web. Dennoch ist Flutter und Firebase ein interessantes Gesamtsystem. Denn durch Flutter muss lediglich einmal der Code zum Zugriff auf die Datenbak oder die anderen Dienste geschrieben werden. So wird nicht nur weiter Entwicklungszeit um sehr ähnlichen Code zu schreiben, sondern verhindert gleichzeitig eine unterschiedliche Definition der Entitäten oder anderen Elementen. Außerdem ist dadurch ein problemloser Wechsel von einem System zu einem anderen Möglich, da alle Anwendungen auf egal welcher Plattform gleich sind.

\subsection{Conclusion Flutter}
Durch die hier beschriebene Entwicklung kann mit Hilfe von einem Code, eine Anwendung geschrieben werden, die sowohl auf Android, iOS, Windows, Mac und Linux läuft. Lediglich die Web Implementierung funktioniert nicht, da der GraphQL Client hier keine richtige Verbindung erstellen kann. Für alle anderen Plattformen muss lediglich die Unterstützung deklariert, der Code in Plattform spezifischen Code compiliert und am Ende ausgeführt werden.

Das Tempo mit dem eine Flutter Anwendung entwickelt werden kann ist ähnlich wie eine einzelne native Implementierung. Dabei entsteht aber, wie bereits erwähnt, nicht nur die Implementierung für eine Plattform, sondern 5. Selbst wenn der Fokus der Entwicklung zuerst nur auf einer Plattform liegt, kann es sinnvoll sein Flutter zu nutzen um in Zukunft weitere Plattformen hinzuzufügen. Denn alle Plattformen können nachträglich noch exportiert werden.

Jedoch ist dies natürlich nicht immer möglich. Denn etwa der GraphQL Client ist nicht mit einer Web-Version kompatibel. Daher sollte bei der Wahl der Pakete darauf geachtet werden, welche Packete ausgewählt werden, um eine Inkompatibilität mit den gewünschten Plattformen zu verhindern. Wie bereits ausgeführt, kann versucht werden, solche Probleme durch eine eigene Implementierung zu lösen, dies ist jedoch mit einem erhöhten Aufwand und nötigen Wissen verbunden. Dabei ist die eigentliche Implementierung an sich nicht das eigentliche Problem, da auch hier dann native externe Bibliotheken genutzt werden können, jedoch ist die Konfiguration der Kommunikationsschnittstellen mitunter recht kompliziert und müssen sinnvoll implementiert sein, um die Performance nicht erheblich zu schwächen.

Flutter ist außerdem noch recht neu. So werden bei regelmäßigen Updates auch Änderungen eingeführt, die die Qualität und Entwicklung verbessern, jedoch dabei auch umfangreiche Änderungen in allen Applikationen notwendig macht. So wurde etwa ein Update veröffentlicht, dass das Nullsafety verbessern sollte. In Folge dessen musste jede Bibliothek angepasst und auch große Teile von Applikationen angepasst werden. Dadurch sind auch einige Anleitungen für Flutter veraltet und funktionieren in der aktuellsten Version nicht ohne Anpassungen.

Ein letzter Punkt ist, dass es keine Liste der verfügbaren Widgets innerhalb der Entwicklungsumgebung gibt. So muss alles, was nicht bereits bekannt ist gegoogelt werden. Hier wäre eine Übersicht über die typischen Elemente hilfreich um gerade für Anfänger den Einstieg zu erleichtern.

\section{Entwicklung einer gemischten Applikation mit WebView und Flutter}
Die vierte Implementierung ist dem gemischten Ansatz zuzuordnen. Sie besteht aus der Kombination einer hybriden Applikation und einer cross-kompilierten Applikation.
Hierbei sind die einzelnen Ansichten entweder eine mit Flutter implementierte Seite oder eine mittels WebView integrierte Ansicht der Webseite.
Durch diese Kombination soll die bereits bestehende Webseite wiederverwendet werden, während gleichzeitig bestimmte Teile des Systems für die mobile Nutzung angepasst werden und durch eigens programmierter Flutter Seiten ersetzt werden.
Dadurch kann ein Teil der Neuimplementierung vermieden werden, während die Nutzererfahrung für die neuen Plattformen optimiert werden kann.
Dabei soll weiterhin die Möglichkeit bestehen, die individuellen Applikationen aus einer gemeinsamen Code-Basis zu kompilieren.

\subsection{Grundlagen}
Da es sich wiederum um eine Flutter Applikation handelt gelten die selben Grundlagen wie in Kapitel \ref{cha:4_3_1}.
Zusätzlich wird nun eine WebView-Komponente hinzugefügt und zwischen Web-Oberflächen und Flutter-Seiten hin und her gewechselt.
Da dies kein weit verbreiteter Ansatz ist und es folglich keine offizielle Dokumentation oder Anleitungen für diesen Ansatz gibt, mussten hierzu einige Herausforderungen überwunden werden, die im Folgenden erläutert werden sollen.

Die Integration der Webseite ist ähnlich zum Vorgehen bei der hybriden Applikation. So werden aufgerufenen URLs abgefangen und es wird sich für eine von drei Vorgehensweisen entschieden. So werden Links zu externen Webseiten im Browser des Gerätes geöffnet. URL´s der eigenen Web-Anwendung, bei der die Web-Ansicht genutzt werden sollen, werden in der WebView geladen. Letztlich werden URL's, die zu Seiten gehören, die durch eine Flutter Seite angezeigt werden sollen, verworfen und die entsprechende Flutter Seite mit eventuell vorhandenen Parametern aus der URL geladen. 
Dementsprechend muss die Analyse und darauf folgende Unterscheidung der URLs deutlich feingradiger stattfinden.

Flutter beendet standardmäßig alle nicht mehr angezeigten Seiten und löscht dabei auch mögliche gespeicherte Informationen. So etwa auch die Navigationshistorie der WebView.
Da diese allerdings benötigt wird, um innerhalb der Webanwendung zurück zur vorherigen Seite zu navigieren, wird folglich eine Möglichkeit benötigt, um die Daten der WebView zu speichern. Dafür wurde das von Flutter bereitgestellte \verb|AutomaticKeepAliveMixin|\footnote{\url{https://api.flutter.dev/flutter/widgets/AutomaticKeepAliveClientMixin-mixin.html}} verwendet. 
Wie der Name beschreibt, werden damit die States von Widgets markiert, um zu bewirken, dass die Garbage Collection diesen nicht löscht. 
Bei einem erneuten Aufruf der Seite wird dann der noch vorhandene State geladen. Dadurch wird ein Zurücknavigieren in der Webseite auch nach dem vorherigen Verlassen der Web-Ansicht weiterhin möglich.

Es existiert außerdem bei diesem Ansatz eine zusätzliche Herausforderung bei der Navigation.
Dabei ist das Navigieren zu einer neuen Seite ohne Probleme durch die Navigation der WebView möglich. Die aufgerufene URLs werden abgefangen und ,wie bereits erklärt, weiterverarbeitet. Aus einer Flutter-Ansicht heraus wird dann entweder zu einer anderen Flutter-Ansicht navigiert oder die neue URL in der WebView aufgerufen. Bei der Rückwärtsnavigation ergibt sich jedoch folgendes Problem. Zwar kann die Flutter Navigation ohne Probleme von einer Flutter-Seite in die WebView zurückkehren, wo dann die Navigation der WebView übernimmt. Wurde jedoch in vorherigen Verlauf zu einer Flutter Webseite navigiert, so ist diese im Verlauf der Webseite enthalten. Da die Methode, welche über das weitere Verfahren der URL entscheidet, lediglich bei vorwärts navigieren aufgerufen wird, würde fälschlicher Weise die Webseite angezeigt werden und nicht die angepasste Flutter-Seite.
Um diese Herausforderung zu überwinden, wurden zwei mögliche Lösungen erarbeitet:
\begin{enumerate}
    \item Eine eigene Navigation programieren, in der genau definiert werden kann, wann eine URL als Webseite geladen wird und wann eine Flutter Seite aufgerufen wird. 
    \item Jedes mal, wenn von einer Flutter Seite zurück in eine WebView gewechselt wird, eine neue WebView erzeugen und somit den bereits vorhandenen Entscheidungsprozess für URLs nutzen. 
\end{enumerate}

Beide Lösungen haben Vor und Nachteile. So benötigt der erste Ansatz weniger Ressourcen, da lediglich eine WebView genutzt werden kann. Jedoch wird hier ein hoher Implementierungsaufwand benötigt, um eine angepasste Navigation zu schreiben. 
Beim zweiten Ansatz ist die Performance zwar schlechter, jedoch kann die von Flutter bereitgestellte Navigation genutzt werden. Somit können Fehler bei der Implementierung und ein zusätzlicher Programmieraufwand vermindert oder vermieden werden.
Wenn zusätzlich sinnvolle Punkte in der App definiert werden, an welchen die Navigationshistorie der Flutter-Navigation gelöscht wird, können folglich alte WebView Konfigurationen von der Garbage Collection gelöscht werden und folglich die vom Ansatz benötigten Ressourcen reduziert werden. Daher wurde für die Implementierung die zweite Lösung gewählt. 

Eine letzte Herausforderung war die Kombination der verschiedenen Technologien. Beispielsweise wurde bei der Web-Implementierung eine sogenannte Live-Komponente benutzt. Diese sorgt dafür, dass bei einem Wechsel von einer Seite zu einer anderen, keine neue URL geladen wird, sondern lediglich der Inhalt dynamisch nachgeladen wird. Dies lag zum Beispiel bei der Umsetzung des Chats vor. So war die Überssichtsseite der Konversationen mit der Anzeige der einzelnen Chats über diese Technologie verbunden. Deshalb konnte nicht ,wie ursprünglich geplant, lediglich die Chat-Seite durch Flutter ersetzt werden, sondern es musste auch die davorliegende Übersichtsseite umgesetzt werden.

\subsection{Benutzte Packages}
Es wurden die selben Erweiterungen wie in der Flutter Implementierung benutzt. Zusätzlich wurde ein WebView-Package benötigt, um die hybride Implementierung zu ermöglichen. Hierfür wird das von Flutter veröffentliche \verb|webview_flutter|\footnote{\url{https://pub.dev/packages/webview\_flutter}} Package genutzt. Anzumerken ist, dass die Wahl der richtigen Erweiterung kompliziert war, da keine der Optionen alle Plattformen unterstützte. Letztendlich wurde die offizielle Erweiterung gewählt, welche lediglich Android und iOS unterstützt.

\subsection{Fazit gemischte Implementierung}
Im Vergleich zur hybriden Implementierung ist mehr Implementierungsaufwand nötig, da Teile der Webseite durch eine eigene Implementierung ersetzt werden. Jedoch konnte die Benutzbarkeit der Applikation verbessert werden, da durch das gezielte Ersetzen der Webanwendung eine für die Plattform speziell angepasste Funktionalität oder Aussehen erreicht werden kann. Dabei kann Implementierungszeit im Vergleich zur nativen beziehungsweise cross-kompilierten Applikation gespart werden, wenn die Teile der Webanwendung, die von einem derartigen Umbau nicht profitieren, weiterhin als Webseite angezeigt werden.
Außerdem ermöglicht dieser Ansatz, dass ein Umbau von Webanwendung zu Applikation iterativ möglich ist und somit eine erste Version früher veröffentlicht werden kann.

Jedoch entstanden durch die Nutzung des gemischten Ansatzes einige Herausforderungen, für die es wenig Hilfestellungen in Foren oder Dokumentationen gibt. Daher ist dieser Ansatz mit einem erhöhten Aufwand verbunden und es besteht die Gefahr, dass etwa eine reine cross-kompilierte Implementierung schneller entwickelt werden könnte.
\chapter{Auswertung}
In dem folgenden Kapitel sollen die Ergebnisse der Implementierungen ausgewertet und analysiert werden. Dafür wurden einige Kriterien ausgewählt und die Implementierungen und sonstige Quellen daraufhin untersucht und verglichen.

\section{Performance und Entwicklung}
Bei einigen Implementierung geht es auch stark um die Performance der entwickelten Anwendungen. Die Performance und einige andere Faktoren der Entwicklung sollen im folgenden etwas genauer unter anderem durch gemessene Werte analysiert werden.

\subsubsection{Dauer typischer Entwicklungsoperationen}
Wenn es um Entwicklungzeit und Fortschritt geht, dann ist oft auch ein Faktor, wie lange die App braucht um gebaut zu werden und wie lange es dauert um Änderungen in der App anzuzeigen. Oft muss bei der Entwicklung von Oberflächen einige Kleinigkeiten geändert und dann überprüft werden, ob die Änderung den gewünschten Effekt hatte. Deshalb sind vor allem kurze ladezeiten von Änderungen besonders wichtig.

Für diesen Test wurde die Dauer des erstmaligen Compilieren, die Dauer eines Neubauens auf Basis eines bestehenden Caches, die Dauer bis Layoutänderungen und bis Anwendungslogik geladen wurde, im Debug Modus durchgeführt. Es wurde außerdem die Compilierzeit im Release Modus aufgezeichnet.
Die Tests hierfür wurden jeweils fünf mal ausgeführt und dann am Ende ein Durchschnitt gebildet. Als Hardware wurde ein PC mit 32GB RAM und einer Ryzen 5 2600 6-Kern CPU genutzt. 
Die genauen Zahlen sind von der genutzten Hardware und der Projektgröße abhängig, geben aber einen Einblick ob hier einzelne Ansätze einen Vorteil haben. 

\begin{table}
\centering
\caption{Dauer typischer Entwicklungsoperationen in Sekunden}
\begin{tabular}{ |p{4cm}||p{3cm}|p{2cm}|p{2cm}|p{3cm}|p{3cm}| }
 \hline
 Funktion & Flutter-Hybrid & Flutter & Kotlin-Nativ & Kotlin-WebView \\
 \hline
 Build Zeit Release       &   59,2&   54,22& 40,92& 22,42\\
  \hline
 Build Zeit Debug  & 33,78& 28,76& 20,2& 20,19\\
  \hline
 Erneuter Build mit Cache & 8,8& 8,36& 3,36& 3,21\\
  \hline
 Neuladen nach Layoutänderg & 0,59& 0,6& 2,59& 2,64\\
  \hline
 Neuladen nach Logikänderung & 0,626& 0,622& 3,18& 2,63\\
  \hline
\end{tabular}
\label{tab:evaluations_build_time}
\end{table}

Bei der Analyse der Daten fällt auf, dass grundsätzlich bei der Build Zeit die Flutter Apps schlechter abschließen. Sowohl beim Erstellen einer Release oder auch Debug \ac{APK} sind die Kotlin Applikationen schneller. Bei der Debug \ac{APK} ist der Unterschied jedoch deutlich geringer und selbst die Kotlin-WebView App benötigt in etwa gleich lang, wie die native Kotlin Implementierung. Beim Compilieren einer Release App scheint es daher tatsächlich mehr um den Umfang der Applikation zu gehen, als bei der Debug Version.

Der auffälligste Unterschied ist jedoch die Neuladezeit nach einer Änderung im Quellcode. Denn Flutter schneidet dank des HotReload Features hier deutlich besser ab und benötigt somit gerade mal 1/5 der Zeit die die mit Koltin implementierten Applikationen benötigen. Dazu kommt, dass beim Neustarten bei Android, die Applikation vom Startbildschirm neustartet, während Flutter nur die Änderungen in die aktuelle Seite lädt und diese neustartet. Dadurch muss nicht erst wieder zu der aktuell entwickelten Stelle zurückgekehrt werden. So wrid etwa die Farbe eines Textes im Fluttercode geändert und innerhalb von 600ms wird die Änderung angezeigt. So wird nicht nur Zeit gesparrt, sondern es kann sich auch besser darauf konzentriert werden, eine ordentliche Nutzeroberfläche zu entwickeln.

\subsubsection{Performance}

\begin{table}
\centering
\caption{Performancemessung der verschiedenen Applikationen }
\begin{tabular}{ |p{4cm}||p{3cm}|p{2cm}|p{2cm}|p{2cm}|p{2cm}| }
 \hline
 Parameter & Flutter-Hybrid & Flutter & Kotlin-Nativ & Kotlin-WebView \\
 \hline
 Durchschnittliche CPU-Auslastung       &   2,54\%&   1,96\%& 0,9\%& 1,8\%\\
  \hline
 Maximale CPU- Auslastung  & 9,8\%& 6,4\%& 3,6\%& 7,4\%\\
  \hline
 Durchschnittliche RAM-Auslastung & 215,38 MB& 150,68MB& 86,74MB& 107,68MB\\
  \hline
 Maximale RAM- Auslastung & 238MB& 175,46MB& 100,64MB& 117,06MB\\
  \hline
 App-Größe & 8,6MB& 8,4MB& 5,2MB& 4,4MB\\
  \hline
 Maximale Startzeit & 532ms& 452ms& 263ms& 486,6ms\\
 \hline
 Durchschnittliche Renderzeit &8,68ms& 5,12ms& 9,04& 21,88ms\\
 \hline
\end{tabular}
\label{tab:evaluations_performance}
\end{table}

In Tabelle \ref{tab:evaluations_performance} sind die Ergebnisse der Performance-Messungen zu sehen, die an den beschriebenen Implementierungen durchgeführt wurden.
Die Messungen wurden mit dem Programm Apptim durchgeführt. Dies zeichnet die Performance von Apps während der Nutzung einer App auf. Die Tests wurden auf einem Google Pixel 5 durchgeführt mit Android Level 12 oder auch API Level 31. Es hat dabei 8GB DDR4-Ram und einer 8-Kern-CPU, die mit 6x1,8GHz, 1x2,2GHz und 1x2,4GHz getaktet sind.
Der Test wurde dabei für jede Implementierung 5 mal wiederholt und aus den Ergebnissen dann ein Durchschnittswert gebildet. Die Apps wurden nach jeder Nutzung zurückgesetzt und alle anderen Apps wurden während der Tests beendet. Die Apps waren dafür jeweils als Release-Version installiert, so dass die Apps bei bester Leistung wie auch bei einem Endbenutzer laufen würden.
Renderzeit entspricht der Zeit, die gebraucht werden, bis die Änderungen geziechnet wurden und als Bild angezeigt werden können

Hierbei ist erkennbar, dass die native Android App insgesamt die schnellste ist, während die Hybride Flutter App die am schlechtesten performende. Die schnellste Renderzeit hatten die Flutter-Applikationen, während die Auslastung des Rams bei den Android Applikationen deutlich geringer war. Überraschend zu sehen ist, dass die WebView Applikation mehr Ressourcen verbraucht als die nativ implementierte, obwohl sie lediglich einen WebContainer bauen muss. was außerdem auffällt, ist dass Flutter Applikationen etwa zwei mal so groß sind als die nativen. Eine weitere Sache ist, dass die die Android Apps und die reine Flutter App sehr vergleichbar sind, während die Hybride Flutter App in fast jeder Kategorie doppelt so viele Ressourcen verbraucht hat, wie die native Android Anwendungen.

Was festhalten werden kann, ist, dass eine Flutter App zwar in der Performanz etwas den nativen geschriebenen Applikationen hinterherhängt, jedoch der Unterschied nicht derart groß ist, als dass er eine große Rolle spielen würde. Was jedoch auffält ist das die hybride Flutter App Performance technisch deutlich schlechter abschneidet als die native.  

Biorn Hansen et al\cite{BirnHansen.2020} sagen in ihrer Auswertung, dass sie einen deutlichen Anstieg bei Datenbankbenutzung sehen konnten. Deshalb wurde bei der Flutter App und der nativen Android App  jeweils eine Datenbankimplementierung nach der Dokumentation der einzelnen Plattformen eingefügt. Daraufhin wurden beide Apps wieder wie oben beschrieben getestet und am Ende der Anstieg in den einzelnen eine Applikationen ausgewertet und in Tabelle \ref{tab:evaluations_performance_Overhead_database} zusammengetragen.

Auffällig ist, dass die CPU auslastung bei beiden gleichmäßig ansteigt, während bei der Nativen die RAM Nutzung deutlich mehr ansteigt, als bei der Flutter. Wenn Tabelle \ref{tab:evaluations_performance} miteinbezogen wird, nutzt die native App damit immer noch weniger Ram als die normale Flutter, da diese bereits ohne Datenbank eine recht hohe RAM Nutzung hat. Bei der App Startzeit ist die Flutter App, die nochmal 60 ms mehr länger braucht als die Android und somit nun 36\% langsamer ist als die Native. Bei der Renderzeit zeigt sich weiterhin die stärke von Flutter, da hier ein deutlich geringerer Anstieg zu verzeichnen ist.

\begin{table}
\centering
\caption{Unterschied bei Implementierung mit zusätzlicher Datenbankimplementierung}
\begin{tabular}{ |p{7cm}||wc{3.5cm}|wc{3.5cm}|}
 \hline
 Parameter & Flutter &  Kotlin-Nativ \\
 \hline
 Durchschnittliche CPU-Auslastung       &  0,44\%&   0,26\%\\
  \hline
 Maximale CPU-Auslastung  & 2,2\%& 2,6\%\\
  \hline
 Durchschnittliche RAM-Auslastung & 5,64 MB& 18,04MB\\
  \hline
 Maximale RAM-Auslastung & 4,98MB& 39,64MB\\
  \hline
 App-Größe & 0,1MB& 0,1MB\\
  \hline
 Maximale Startzeit & 221ms& 162,6ms\\
 \hline
 Durchschnittliche Renderzeit &0,82ms& 4,68ms\\
 \hline
\end{tabular}
\label{tab:evaluations_performance_Overhead_database}
\end{table}

Grundsätzlich kann festgehalten werden, dass native Entwicklungen meistens die beste Performance haben werden, Jedoch ist mit Flutter ein Framework auf den Markt gekommen, das dies in Frage stellt. Denn Flutter ist wie bereits erwähnt ein Framework, dass den geschriebenen Dart-Code in Nativen Code umwandelt. Es hat sich außerdem zum Ziel gesetzt, möglichst performant zu sein. Eine Untersuchung von Biørn-Hansen et al hat ergeben, das Nativ zwar immer noch am performantesten ist, Flutter in einigen Kategorien auch noch etwas hinten an ist, jedoch dieser Unterschied nicht besonders groß ist  \cite{BirnHansen.2020}.

Eine andere Untersuchung zeigte, dass  es darauf ankommt, was man macht\TODO{Hier fehlt noch was}

\subsubsection{Debugging/Testing}
Bei Debugging und Testing stehen zumindest in unseren Testfällen die Apps beim Applikationscode in keinem Teil nach. Bei allen kann man sowohl Breakpoints als auch Konsolenausgaben veranlassen. Es können auch für alle Arten Tests geschrieben werden.

Bei den hybriden Ansätzen ist die Debug und Test Möglichkeiten  durch die Web-Implementierung allerdings teils stark eingeschränkt. Wenn etwa wie in diesem Fall die Web-Implementierung nicht innerhalb der Applikation sonder auf einem externen Server läuft, müssen die beiden Teile getrennt betrachtet werden. Dies ist auch mit allen Frameworks möglich, bedeutet jedoch einen erhöhten Aufwand und erschwert die Fehlerfindung mitunter.

\section{Community}
Die Entwickler Community ist ein wichtiger Faktor für die Wahl eines Frameworks. Wenn es keine Community gibt, wird es schwer Entwickler oder auch Lösungen für Probleme zu finden, ohne sie selber zu entwickeln. Aber nicht nur Bibliotheken helfen dabei, schneller zu entwickeln sondern auch Frage-Antwort Communities wie Stackoverflow\footnote{https://stackoverflow.com/}.  Die genauen Zahlen der Community-Größe oder die Populärität sind dabei nicht genau bestimmbar. Sie können jedoch anhand einiger Faktoren annähernd bestimmt werden, bzw. verglichen werden.

\subsubsection{Star-History}

Ein erster Anhaltspunkt ist die Sogenannte Star-Anzahl von Github-Repositorys. Es gibt an, wie viele Leute sich die Repositories mit einem Stern markiert haben, um über Neuerungen informiert zu werden. Es gibt also keine absolute Zahl wie viele Leute mit einem Framework oder Programmiersprache entwickeln, es zeigt jedoch ganz gut wie viel Interesse Leute an der Entwicklung dieser haben.

\begin{figure}[ht]
  \centering
  \includegraphics[height=7cm,keepaspectratio]{images/star-history_programming languages.png} 
  \caption[Zeitlicher Verlauf von Stars der Github-Repositorys von Swift, Kotlin und Flutter]{Zeitlicher Verlauf von Stars der Github-Repositorys von Swift, Kotlin und Flutter\protect\footnotemark }
  \label{fig:star_history}
\end{figure}
\footnotetext{https://star-history.com/\#flutter/flutter\&JetBrains/kotlin\&apple/swift\&Date}


Abbildung \ref{fig:star_history} zeigt ein mit Hilfe der GitHub-Api erstelltes Diagramm der Anzahl der Stars der Swift, Kotlin und Flutter Repositories im zeitlichen Verlauf. 
Besonders gut zu erkennen ist, wie stark das Interesse an Flutter ist. Nach der ersten Änkündigung 2018 und der darauf ersten veröffentlichten Version ist die Zahl der Interessierten innerhalb von 4 Jahren auf 140 000 angestiegen. Das Repository für Swift und Kotlin haben im gleichen Zeitraum gerade mal 20 000 Stars dazu erhalten. Dabei haben Swift und Flutter beide innerhalb des ersten Jahres nach Vorstellung etwa 30 000 Stars gemacht. Jedoch ist das Interesse an Swift danach abgeflacht und verläuft aktuell parallel zu Kotlin. Es zeigt also doch, dass das Interesse sehr hoch ist.

\subsubsection{Stackoverflow}
Ein weiterer Anhaltspunkt ist die Anzahl an gestellten Fragen auf Stackoverflow\footnote{https://stackoverflow.com/}. Stackoverflow ist für viele Entwickler ein guter Ort um Lösungen oder Code zu finden, um die aufgetretenen Probleme zu lösen. Stackoverflow kann durch Filterung etwa die Anzahl der Fragen ausgeben. Hierfür wurden die Anzahl der Fragen für Kotlin, Swift und Flutter der letzten 2 Jahre gesammelt. Dabei wurden auch eventuelle Überschneidungen herausgefiltert und es musste mindestens eine Antwort geben.

Für Kotlin sind es 105 082\footnote{Filter: [kotlin] or [android][kotlin] or [android]-[flutter]-[java] lastactive:2y.. is:question answers:1..}, für Swift 71 749\footnote{Filter: [swift] or [ios][swift] or [ios]-[flutter]-[objectivc] lastactive:2y.. is:question answers:1..} und bei Flutter 77,568\footnote{Filter:[flutter] or [dart] -[ubuntu] lastactive:2y.. is:question answers:1..}Fragen.

Flutter hat also scheinbar ebenfalls eine aktive Community die durchaus mit der von Kotlin und Apple mithalten kann. Hier muss jedoch eingeschrenkt werden, dass manche Fragen schon vor dem untersuchten Zeitraum gestellt wurden und deswegen eine Frage nicht in dem untersuchten Zeitraum erneut auftaucht. Da sowohl Kotlin als auch Swift bereits länger auf dem Markt sind, dürfte die Anzahl der Fragen die deshalb nicht gestellt wurden, höher sein als bei Flutter, jedoch kann trotzdem festgestelt werden, dass es hier dennoch genügend Leute vorhanden sind, um Hilfe bei Problemen zu erhalten.

\subsubsection{Dokumentation}
Ein weiterer wichtiger Faktor ist Dokumentation. Hierbei, haben sowohl Flutter als auch Kotlin eine umfangreiche und von den Entwicklern dauerhaft aktualisierte und erweiterte Dokumentationsseite. Wo jedoch Flutter einen Vorteil zu Kotlin hat, ist ein zentraler Ort\footnote{https://pub.dev/} wo man nach weiteren Plugins, Bibliotheken, Widgets und vielem mehr suchen kann. Hier sind nicht nur offizielle Flutter Repositorys verlinkt, sondern jeder aus der Community kann seine Plugins hier verlinken. Neben einem Punktesystem, dass diese bewertet, wie viele Richtlinien es einhält, wird außerdem angezeigt, für welche Plattformen das angezeigte Projekt funktioniert. So kann einfach nach den passenden Erweiterungen gesucht werden, ohne sich durch GitHub-Repositorys zu hangeln, um nach einer passenden Bibliothek zu suchen.

Die Dokumentation für die verschiedenen Ansätze ist allerdings etwas anders. So gelten alle bisherig getroffenen Aussagen vor allem für die native und Cross-Plattform Lösung, da hier die Frameworks und ihre Programmiersprachen dafür eingesetzt werden, wofür sie entwickelt wurden. Für die hybride beziehungsweise die Mix-Implementierung ist dies anders. So kann zwar die in der Regel gute Dokumentation für die gewählte Grundlagen und die Web Implementierung genutzt werden, jedoch ist eine Dokumentation für den genauen Ansatz selten oder nur begrenzt vorhanden. So gibt es etwa einige Tutorials und Seiten Dokumentation für den implementierten hybriden Ansatz, für den hybriden Flutter Ansatz hingegen gab es nur Tutorials für Teile der Implementierung, jedoch nicht für den kompletten Ansatz. 

\subsubsection{Entwickler}
Ein letzter Faktor der zu dieser Kategorie analysiert werden soll, ist der Blick auf verfügbare Entwickler. Denn die Entwicklung einer Applikation in speziellen Technologien funktioniert nur, wenn auch Entwickler gefunden werden können. Eine aktuelle Statistik die aus einer Befragung von 71,547 Entwicklern auf Stackoverflow gebildet wurde, besagt, dass etwa 9\% Kotlin, 7\% Dart und 5\% Swift beherrschen \cite{statist_used_programming_languages}.

Bei den Programmiersprachen für Web, gaben  65\% an, JavaScript zu kennen und 55\% kennen sich in  HTML/CSS aus \cite{statist_used_programming_languages}. Diese Zahlen lassen den Entschluss zu, dass die Ansätze mit einem hohen Anteil an Web Technologie einen Vorteil haben, da es hier noch einfacher sein dürfte, passende Entwickler zu finden. Jedoch wird auch hier mindestens ein Entwickler benötigt, der sich mit den einzelnen Plattformen genauer auskennt.

Zusammenfassend lässt sich sagen, dass es für alle Ansätze genügend Entwickler geben wird, jedoch haben die hybriden und die Flutter-Web Implementierung dahingegen einen Vorteil, dass ein wesentlicher Teil aus einer Webseite oder Webtechnologie besteht und es hier eine deutlich höhere Anzahl an Entwicklern gibt.

\section{Sonstiges}
\subsubsection{Aussehen der Applikationen}
Grundsätzlich ist das Aussehen der Applikation vor allem davon abhängig, wie viel Zeit hierfür investiert wird. Jedoch können anhand der Applikationen dennoch einige Beobachtungen angestellt werden.

\begin{figure}[ht]
  \centering
  \includegraphics[height=7cm,keepaspectratio]{images/Startbildschirm_vergleich.png} 
  \caption[Vergleich des Startbildschirms der Implementierungen]{Vergleich des Startbildschirms der Implementierungen. Von links nach rechts: Hybride-Applikation, native Kotlin Applikation, Flutter Cross-Plattform-Applikation, Flutter Hybride Applikation}
  \label{fig:startscreen}
\end{figure}

In Abbildung \ref{fig:startscreen} sind die Startbildschirme der verschiedenen Anwendungen zu sehen. Für alle wurde eine etwa gleich lange Maximalzeit verwendet, um das Ergebnis vergleichen zu können. Der erste Bildschirm ist von der Webversion. Für diese wurde am wenigsten Zeit genutzt, da an ihr nicht viel geändert werden konnte. Sie ist also die gleiche Darstellung wie wenn die Webseite im Browser aufgerufen werden würde.  Sie ist, wie es auffällt bereits stark für mobile Geräte optimiert und sieht dementsprechend gut nutzbar aus. Was bei ihrer jedoch negativ auffällt, sind die nicht nativen Elemente durch die reine Nutzung von JavaScript.
So ist in Abbildung \ref{fig:sidemenu} die Seitenmenüanzeigen anhand von Flutter und durch die Webimplementierung gezeigt. Hier merkt man bei der Nutzung deutlich, dass die Web-Implementierung für PC Nutzer ausgelegt ist, mit dem Ziel gut für Smartphone Nutzer bedient zu werden, während die Flutter Implementierung deutlich besser für die Nutzung an einem Smartphone ausgelegt ist. Dies ist besonders gut spürbar, da das Menü der Webseite nur über einen Knopf und manuelles drücken nutzbar ist, während in der Flutter Implementierung von der Seite gewischt werden kann um die Vorteile eines Touchscreens auszunutzen.

\begin{figure}[ht]
  \centering
  \includegraphics[height=7cm,keepaspectratio]{images/Seitenmenü_vergleich.png} 
  \caption[Vergleich des Seitenmenüs bei nativer Implementierung und JavaScript Implementierung]{Vergleich des Seitenmenüs bei nativer Implementierung (links) und JavaScript Implementierung (rechts).}
  \label{fig:sidemenu}
\end{figure}

Während auf der Webseite einiges an Zeit und ein extra Designpakete genutzt wurden, um das insgesamte Aussehen der Anwendung zu verändern, wurde bei den getroffenen Implementierungen lediglich die Farben angepasst. Bei der Flutter Implementierung wirken die Elemente dabei deutlich moderner und angepasst für eine Smartphone-Applikation. Das wird besonders deutlich, wenn die Login-Screens verglichen werden.

\begin{figure}[ht]
  \centering
  \includegraphics[height=7cm,keepaspectratio]{images/Login_vergleich.png} 
  \caption[Vergleich des Login-Bildschirms von Kotlin und Flutter Implementierung.]{Vergleich des Login-Bildschirms von Kotlin (links) und Flutter (rechts) Implementierung.}
  \label{fig:loginscreen}
\end{figure}

In Abbildung \ref{fig:loginscreen} ist der Loginscreen der nativen und des Cross-compilierten Ansatzes zu sehen. Hier wurde bis auf Farbanpassungen keine größeren Anpassungen vorgenommen, um einen Eindruck für das Standardaussehen der einzelnen Komponenten zu erhalten. Wie bereits erwähnt hat Flutter ein deutlich besseres Out-of-the-Box Design Paket. Die Input Felder bei der Koltin Implementierung wirken dabei recht unscheinbar und unauffällig, so dass diese für ein fertiges Produkt noch einmal angepasst werden müssten, um Nutzerfreundlich zu sein.

Das Aussehen der einzelnen Apps ist aber durch Erweiterungen und sonstigen stark anpassbar, sodass mit genügend Zeit und Wille alle Applikationen egal mit welchen Ansatz sehr ähnlich aussehen könnten. Wie aber ersichtlich geworden ist, bietet Flutter die am modernsten und passend wirkende Standardkonfigutration für die Benutzeroberflächen. So ist die Entwicklung einer \ac{UI} in Flutter deutlich schneller und mit weniger Konfigurationsaufwand verbunden, als mit jeder anderen betrachteten Implementierung.

\subsubsection{Konsistenz über Plattformen hinweg}
Ein weiterer Aspekt bei der Entwicklung der Nutzeroberfläche ist die Konsistenz über Plattformen hinweg. Das Ziel ist es dabei, dass der Nutzer keinen Unterschied bemerkt, wenn er von einer Plattform zu einer anderen wechselt. 
Dies bedeutet, dass wo immer unterschiedliche Implementierungen für die Plattformen geschrieben werden, die Gefahr für Inkonsistenz am höchsten ist.
Dementsprechend sind Implementierungen die für mehrere Plattformen wieder verwendet werden können ein positiver Faktor um die Konsistenz zu fördern.
Auf die in dieser Arbeit vorgestellten Implementierungen bezogen, bedeutet dies, dass die reine native Implementierung eine höhere Gefahr von Inkonsistenz besitzt, als etwa die hybride oder Flutter Applikation.
Dabei ist allerdings zu erwähnen, dass auch die nativen Applikationen einen hohen Grad an Konsistenz erreichen können. Dies bedeutet jedoch häufig einen höheren Aufwand bei der Implementierung und eine enge Zusammenarbeit der Entwickler der verschiedenen Plattformen.

\subsubsection{Plattformabdeckung \& Wiederverwendbarkeit}
Wie an mehreren Stellen der Arbeit bereits erwähnt sind nativ entwickelte Applikationen immer nur für eine Plattform entwickelt. Sie haben dementsprechend auch den am wenigsten wiederverwendbaren Code, da lediglich die Logik geteilt werden kann, jedoch die genaue Implementierung für jede Plattform unterschiedlich ist. Jedoch kann für jede Plattform eine native Apllikation geschrieben werden.

Hybride Apps haben hier bereits einen deutlich höheren Wiederverwenbarkeitsgrad, da bei ihnen ja lediglich die Anzeigelogik ausgetauscht werden muss, aber die iegentliche Implementierung in einer Webtechnologie getan wurde, wodurch sie auf fast jeden Gerät genutzt werden kann. Jedoch können einige Technologien auf ein paar Plattformen beschränkt sein. Desweiteren kann es bei dieser Klasse durchaus Probleme geben, wenn sie zu einem großen Anteil aus der Ansicht der Webseite besteht. So wurde ja bereits die Richtlinien von Apple erwähnt, die diese Klasse zu einem gewissen Grad aus dem App-Store verbannt. Da hybride Applikationen außerdem einen nativen Code-Teil besitzen, sind sie wieder nur auf einer Plattform installierbar. 

Cross-Plattform-Applikationen werden mit einer Technologie gebaut, um mehr als eine Plattform mit einem Code abzudecken. Dabei ist dies zwar von der genau genutzten Technologie abhängig, aber in dem vorgestellten Fall sind 5 Plattformen damit abdeckbar. Mit Flutter und der Wahl der richtigen Erweiterungen sind nach aktuellem Stand sogar 6 Plattformen möglich. Außerdem kann durch die Wahl von Flutter der gleiche Code für alle Plattformen genutzt werden. Eine Wiederverwendbarkeit ist so gesehen also nicht bewertbar, da es theoretisch nicht nötig ist den Code für eine andere Implementierung wiederzuverwenden. Wenn allerdings der Code für eine andere Implementierung genutzt werden sollte ist es wie im nativen Fall, dass nur die Applikationslogik nutzbar ist, da die Anwendung in ihrer eigenen Programmiersprache geschrieben ist.

Die Flutter-Web Implementierung ist in diesem Aspekt ähnlich zu der Flutter Implementierung. So gelten grundsätzlich die gleichen Aussagen, jedoch mit Einschränkung der Anzahl der abgedeckten mobilen Endgeräte. So ist diese Implementierung lediglich auf den beiden Plattformen Android und iOS nutzbar. Wie bereits erwähnt ist dies Abhängig von der Web-Container Implementierung und kann mit der Wahl eines passenderen Containers oder Eigenentwicklung eines eigenen Browser-Plugins auf die restlichen Plattformen erweitert werden.

\subsubsection{Programmieraufwand}
Der genaue Programmieraufwand ist immer stark abhängig von den genauen Vorkenntnissen, der genutzten Technologie und externen Bibliotheken und natürlich dem Umfang der Anwendung. Ein Anhaltspunkt sind die \ac{LOC} die für die einzelnen Implementierungen geschrieben wurden, um dies ein wenig vergleichen zu können. Dafür wurden die einzelnen Daten mit Hilfe des Statistics\footnote{https://plugins.jetbrains.com/plugin/4509-statistic} Plug in gesammelt. Dabei wurden die gesammelten Daten in die drei Typen Konfiguration, Oberfläche/UI und Logik unterteilt. Bei den zwei Implementierungen die außerdem noch Teile der Webseite mit anzeigen, wurden außerdem die in der Anwendung abgedeckten Programmteile erfasst, da diese für eine Implementierung auf dem Stand genauso notwendig ist. Zusätzlich wurde der benötigte Code untersucht um die Liste der Gegenstände anzuzeigen. Also sowohl die Layoutdatein, als auch die benötigte Anwendungslogik.\TODO{Weiß nicht ob Logik hier der richtige Begriff ist, da dies ja auch das hinzufügen der Oberflächenfunktionalität ist.}
Bei den Implementierungen, bei denen Flutter genutzt wurde, ist die Oberflächen und Logikcode Zeile zusammengefasst, da dies bei Flutter in einer Datei geschrieben wird. 

\begin{table}
\centering
\caption{Programmlänge der verschiedenen Implementierungen in \ac{LOC}}
\begin{tabular}{ |p{4.5cm}||p{3cm}|p{2cm}|p{2cm}|p{3cm}|p{3cm}| }
 \hline
 Programmteile in Lines of Code & Flutter-Web & Flutter & Kotlin-Nativ & Kotlin-Hybrid \\
 \hline
 Gesamte Anwendung       &   2391(App) + 2033(Web) &   2100 & 3138 & 449(App) + 2814(Web)\\
  \hline
 Konfigurationscode  & 42 + 268& 32& 229& 16 + 357\\
  \hline
 Oberflächencode &\multirow{2}{*}{2349 + 1765}  &\multirow{2}{*}{2068}  & 1958& 334 + 1768\\
  \cline{1-1}
  \cline{4 -5}
 Logikcode & & & 951& 99 + 689\\
  \hline
 Beispeil: Liste an Gegenständen & 65 & 65 & 178 & 71(Web)\\
  \hline
\end{tabular}
\label{tab:lines_of_code}
\end{table}

In Tabelle \ref{tab:lines_of_code} sind die \ac{LOC}s der einzelnen Implementierungen auf verschiedenen Programmteile aufgeschlüsselt zu sehen. Dabei wurde auch der Code der genutzten WebImplementierung dazugerechnet, da diese ein Teil der jeweiligen Anwendung ist. Die wenigsten Zeilen hat die reine Flutter Implementierung und die meisten die Flutter-Web Implementierung.  Die Flutter-Web Implementierung besteht dabei in etwa zu einer Hälfte aus der Flutter Implementierung und zu einer Hälfte aus der Web Implementierung. Die hybride Implementierung hat zwar den wenigsten Code für die Applikation an sich, hat jedoch durch die Webseitenimplementierung mehr Zeilen als die native Applikation, jedoch weniger als die Flutter-Web Implementierung. 

Bei dem untersuchten Beispiel einer Liste für Gegenstände ist wieder die Flutter Implementierung die mit dem geringsten Code. Dabei entfallen 55 Zeilen auf die Anzeige eines Gegenstandes und gerade einmal 10 Zeilen Code um die Liste einzubauen und zu konfigurieren. Ähnlich ist dies bei der Webseite. Einzig die native Kotlin App hat hier ein erhöhten Aufwand. Dabei entfallen 75 Zeilen auf das Design und weitere 83 Zeilen auf die Steuerung der Liste und dem hinzufügen der Daten. Ein Großteil davon ist der Adapter der benötigt wird um die Liste zu steuern. Durch solche Konstrukte benötigt Kotlin mehr \ac{LOC}, als etwa die Flutter Implementierung.

Bei den Implementierungen in der die Webseite eingebaut sind, fügt die Webseite einen hohen Teil der Zeilen hinzu. Dies ist aber zu einem gewissen Teil dem Aufbau des Webcodes geschuldet. So ist bei dem Vergleich des benötigten Codes für die Liste, die Webseite gerade mal 6 Zeilen hinter der Flutter Implementierung und damit immer noch gerade mal halb so viel wie die Kotlin Entwicklung benötigt.

\subsubsection{Zeit bis zum Release}
Durch den Programmieraufwand und der Anzahl der zu unterstützenden Plattformen wird die Dauer der Entwicklungszeit und die Zeit bis zu einem möglichen Release stark beeinflusst. Genaue Werte sind dabei auch stark von Funktionsumfang, Anzahl der Entwickler und den Vorraussetzungen abhängig. Jedoch lassen sich dennoch einige Aussagen treffen.

So ist etwa ein Release der Nativen Anwendung am aufwendigsten. Nicht nur dass wie bereits gesehen, mit der meiste Code geschrieben werden muss, muss dies auch noch für jede Plattform passieren. Am schnellsten ist dies bei der Cross-Plattform Implementierung mit Flutter. Hier muss lediglich einmal die Anwendung durchprogrammiert werden und kann im Anschluss für die einzelnen Plattformen einfach nur noch exportiert werden. Nach der reinen Flutter Implementierung ist die Flutter Implementierung mit Webanteil. Diese ist dann gefolgt vom hybriden Ansatz. 

Wenn jedoch die Webseite bereits existiert und dementsprechend nicht bei der Implementierzeit beachtet werden muss, ändert sich hier die Reihenfolge sehr wahscheinlich je nach umfänglichkeit der Zusatz Implementierung des Flutter Web Ansatzes.

Der Flutter-Web Mixansatz kann außerdem noch von hohem Vorteil sein, wenn eine Webseite in eine App verwandelt werden soll. Denn hier kann wie bei der hybriden Implementierung eine anfängliche Version veröffentlicht werden, die lediglich die Webseite öffnet und danach Stück für Stück die Webseite durch Flutter Seiten ersetzen. So kann der Nutzer jederzeit die komplette Anwendung nutzen und bekommt Stück für Stück eine angepasste Appversion.

\subsubsection{Offline Funktionalität}
Es gibt einige Fälle in denen der Nutzer unter Umständen eine Applikation weiter nutzen will, auch wenn er aktuell keine aktive Internetverbindung hat. In diesem Fall benötigt die Applikation eine offline Funktionalität.

Wie bereits an mehreren Stellen in dieser Arbeit erwähnt haben die Implementierungen, die die Webseite als Teil ihrer Implementierung haben, hier oft keine Funktionalität. Im Falle der Flutter-Web Applikation kann die Funktionalität zumindest teilweise erreicht werden, indem andere Seiten angezeigt werden die nativ implementiert werden. Die native und Cross-Compilierte Lösung haben hier eindeutig den besseren Standpunkt. Ihre Anzeige ist unabhängig von einer aktiven Internetverbindung nutzbar. Hier kommt es lediglich darauf an, ob die Daten, die in der App angezeigt werden auf dem Gerät gespeichert werden oder ob sie immer von einem Server abgefragt werden müssen. 

Hier ist allerdings zu erwähnen, dass andere hybride Ansätze zwar auch Webtechnologie nutzen, jedoch die Daten alle lokal auf dem Gerät gespeichert werden. In diesem Fall hätte auch der hybride Ansatz offline funktionieren. Der in diese Arbeit gewählte Ansatz jedoch nicht.


\subsubsection{Lesbarkeit und Übersichtlichkeit der Implementierung}
Bei der Übersichtlichkeit ist es wieder eine Frage nach der Art der Implementierung. 

\section{Fragekompass um eine Auswahl zu treffen}
1. Existiert bereits eine Webseite, die in eine App umgewandelt werden soll?
    Wenn eine Webseite bereits existiert, sind der hybride und der Web-Flutter Ansatz eine gute Option. Denn durch die Integration kann sowohl ein großer Teil der Implementierung vermieden werden, selbst wenn letztlich nicht die Webseite in der App angezeigt wird, sondern die echten HTML Datein. Denn diese können im Zweifelsfall kopiert und falls benötigt umgearbeitet werden.
    
    Es kann einen fließenden Übergang von Webseite zu komplett Implementierter Applikation geben soll. So kann bei dem entwickelten gemixten Ansatz eine erste Version schnell entwickelt werden, die ausschließlich aus der Webseite besteht und dann Stück f
2. Ist Performance Priorität Nummer 1 ?

3. Sollen mehr als nur eine Plattform abgedeckt werden ?

4. Existiert bereits eine Applikation die nur etwas geändert werden soll?

5. 


\chapter{Zusammenfassung und Ausblick}
Im Folgenden werden die Erkenntnisse der Arbeit noch einmal zusammengefasst und einige Erkenntnisse erwähnt werden. Im Anschluss soll ein Ausblick auf weitere Forschung und mögliche Verbesserungen gegeben werden.

\section{Zusammenfassung}
In dieser Arbeit wurden verschiedene Methoden genauer vorgestellt, um eine Multi-Plattform Anwendung für mobile Endgeräte zu Entwickeln. Dabei wurde die Implementierung auf eine Smartphone Android Implementierung beschränkt und vier verschiedene Ansätze programmiert. Dabei wurde einerseits eine native Android Implementierung mit Kotlin, eine hybride Applikation mit Kotlin, eine Cross-Compilierte Flutter Applikation und ein gemischter Ansatz einer hybriden Cross-Compilierten Flutter Applikation. Nach der Implementierung dieser Ansätze wurden diese in verschiedenen Kriterien untersucht. So etwa die Performance. Dabei konnte festgestellt werden, dass in dem betrachteten Fall die native Implementierung die am schnellsten startende und am Ressourcenschonendste Implementierung ist, während die Mix Implementierung hierbei am schlechtesten abschnitt. Lediglich beim Rendern der Oberfläche war nativ hier nicht der schnellste Ansatz. Dafür ist er im Bereich der Entwicklergemeinschaft durch die längere Existenz und größere Verbreitung im Vorteil. Dabei gab es mehr beantwortete Fragen und teilweise besser ausgearbeitete Lösungen für Probleme wie der Serverkommunikation über GraphQL. Dabei zeigte sich jedoch auch das große Interesse das an Flutter existiert und der diesen Unterschied über die Zeit verringern könnte.

Die hybride Implementierung konnte dafür in den Bereichen der Wiederverwendbarkeit und der Zeit bis zum Release überzeugen. Da

---
Wie durch die verschiedenen untersuchten Kriterien klar geworden ist, konnte der hier vorgestellte mischte Ansatz aus hybrider und Cross-Compilierter Applikation in den meisten Kriterien nicht überzeugen. So hatte er eine spürbar schlechtere Performance, höheren Programmieraufwand, eine deutliche Einschränkung in der Plattformabdeckung oder auch in der Dokumentation. Jedoch kann dieser Ansatz von Vorteil sein, wenn man von einer bestehenden Webseite eine Cross-Compilierte Flutter App erstellen will.

Beim Thema Funktionalität konnte kein Unterschied zwischen den Flutter und der nativen Applikation festgestellt werden, vor allem da Flutter jedgliche native Funktionalität durch ein Erweiterung, die auch selbst entwickelt werden kann, bei Bedarf hinzugefügt werden kann. Lediglich die hybride Applikation fällt hier heraus, da sie wegen ihres Aufbaus auf die Funktionalität von Webseiten beschränkt ist.

Ein wichtiger Punkt jedoch bei dem die native Entwicklung aber hinten an ist, sind die Kosten der Implementierung. Denn bei einer Veröffentlichung einer App auf den Plattformen Android, iOS, Windows, Linux und MacOS würde hier in etwa die 5 fachen Kosten entstehen wie eine Flutter Implementierung, da eben 5 getrennte Applikationen geschrieben werden müssten. Dazu kommt, dass auch die Wartung und Weiterentwicklung jedes mal die 5 fache Arbeit bedeuten würde, während bei Flutter lediglich ein Code entwicklet werden muss. Der hybride Ansatz bietet hier einen Mittelweg, da ein großer Teil seines Code s auf den unterschiedlichen Plattformen wiederverwendet werden kann und nur einige Teile angepasst werden müssen. 

Durch die Nutzung von einem Code kann sowohl bei den Flutter Lösungen als auch der hybriden Webapplikation ein über alle Plattformen hinaus gleiches System geschaffen werden, das keinerlei Unterschiede zwischen den Plattformen aufweist. 

Wie im laufe der Arbeit offensichtlich geworden ist, gibt es viele verschiedene Ansätze um eine Applikation für die verschiedenen Plattformen zu entwickeln. Innerhalb dieser Ansätze gibt es auch nochmal viele verschiedene Frameworks und auch Programmiersprachen die genutzt werden können. Dazu kommt, dass täglich neue Technologien entstehen, die die Ergebnisse der Arbeit wieder komplett verändern könnten.

\section{Ausblick}
Natürlich deckt diese Arbeit nicht alle Arten der möglichen Entwicklung ab. So wurde etwa für die Klasse der hybriden Applikation eine Entwicklung mit Hilfe eines WebContainers genutzt, auf dem lediglich eine im Internet gehostete Webseite angezeigt wird. Jedoch gibt es hier viele andere und mit sicher auch größere Frameworks. Cordova, PhoneGap, Ionic und viele mehr, nur mal um ein paar Namen zu nennen. Jedoch wurde in dieser Arbeit lediglich der erwähnte Ansatz genutzt, da dieser unter den Bedingungen dieser Arbeit der sinnvollste war. Außerdem steht der gewählte Ansatz etwa in Sachen Geräteauslastung keinem der anderen Frameworks nach, sondern dürfte hier sogar besser sein, da der Großteil der Anwendung auf einem externen Server läuft und dadurch nicht auf dem Gerät ausgeführt werden müssen. Es sollte außerdem in dieser Arbeit nicht gesagt werden, welcher Ansatz in den einzelnen Klassen ist, sondern sollte eine Übersicht über die verschiedenen Entwicklungsklassen geben und einen einfachen Orientierungskompass geben um den Entscheidungsweg zum richtigen Ansatz zu unterstützen.

Die Evaluation von Cross-Plattform Frameworks wurde in diesem Fall mit dem aktuell beliebtesten Framework Flutter gemacht. In einer zukünftigen Arbeit wäre eine Untersuchung weiterer Technologien sehr interessant. Dazu kommt, dass die untersuchten Technologien weiterentwickeln. So ist allein in der Zeit der Erstellung der Arbeit bei Flutter ein großer und 3 kleine Release herausgekommen. Dabei sind unter anderem Performance Verbesserungen oder auch neue Plattformen hinzugefügt wurden. Deswegen ist eine wiederholte Betrachtung nach einiger Zeit durchaus interessant zu sehen, wie die Ergebnisse dann aussehen würden.




\vfill
\pagebreak

\appendix

% ----------------------------------------------------------------------------------------------------------
% Abkürzungsverzeichnis (optional, bitte nur wenn sinnvoll)
% ----------------------------------------------------------------------------------------------------------
%\listoftables
\addchap{Abkürzungsverzeichnis}
\begin{acronym}[KDE]
\acro{CRUD}[CRUD]{Create, Read, Update, Delete, die vier Operationen, die auf ein Objekt ausgeführt werden können}
\acro{UI}[UI]{User Interface}
\acro{BA}[BA]{Bachelorarbeit}
\acro{App}[App]{Applikation}
\acroplural{App}[Apps]{Applikationen}
\acro{API}[API]{Application Programming Interface}
\acro{JSON}[JSON]{JavaScript Object Notation}
\acro{DAO}[DAO]{Data-Access-Object}
\acro{APK}[APK]{Android Package Kit, dem Dateiformat von Android Anwendungen}
\acro{LOC}[LOC]{Lines of Code}
\end{acronym}
\vfill
\pagebreak


% ----------------------------------------------------------------------------------------------------------
% Filter fuer Literatur und Quellen definieren
% ----------------------------------------------------------------------------------------------------------

\defbibheading{Literatur}{\addchap{Literaturverzeichnis}} 
\defbibheading{Quellen}{\addchap{Internetquellenverzeichnis}} 
  
\defbibfilter{Literatur}{\not\keyword{online}} 
\defbibfilter{Quellen}{\keyword{online}} 


% ----------------------------------------------------------------------------------------------------------
% Literatur
% ----------------------------------------------------------------------------------------------------------

\printbibliography[heading=Literatur,filter=Literatur] 
\vfill

\pagebreak


% ---------------------------------------------------------------------------------------------------------- 
% Internetquellen 
% ---------------------------------------------------------------------------------------------------------- 

\printbibliography[title = {Quellenverzeichnis}, heading=Quellen,filter=Quellen] 

\pagebreak 

% ----------------------------------------------------------------------------------------------------------
% Anhang
% ----------------------------------------------------------------------------------------------------------
\appendix


%\chapter{Anhang}

% \input{inhalt/suppl}


\pagebreak


% % ----------------------------------------------------------------------------------------------------------
% % Eigenschtändigkeitserklaerung
% % ----------------------------------------------------------------------------------------------------------
\thispagestyle{empty}
\chapter*{Erklärung zur Bachelorarbeit}

\bigskip
\bigskip 
\bigskip 

\textbf{1.}\\[1ex]
    Mir ist bekannt, dass dieses Exemplar der Abschlussarbeit als Prüfungsleistung in das Eigentum der Ostbayerischen Technischen Hochschule Regensburg übergeht.

\textbf{2.}\\[1ex]
    Ich erkläre hiermit, dass ich diese Abschlussarbeit selbständig verfasst, noch nicht anderweitig für Prüfungszwecke vorgelegt, keine anderen als die angegebenen Quellen und Hilfsmittel benutzt sowie wörtliche und sinngemäße Zitate als solche gekennzeichnet habe.

\bigskip 
\bigskip 
\bigskip 
~\hfill\begin{minipage}{.5\textwidth}

Regensburg, den 12. August 2022

\bigskip 
\bigskip

\line(1,0){200}
\newline
\stud

\end{minipage}




\end{document}