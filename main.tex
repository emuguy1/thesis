% add option draft to skip images
\documentclass[12pt,a4paper,twosided,openany]{scrbook}
%\documentclass[12pt,a4paper,twosided,open=right]{scrbook}
% \documentclass[12pt,a4paper,twosided,open=right,draft]{scrbook}

% Entwickelt auf Basis von Carsten Kerns Template.

\usepackage[backend=biber,style=alphabetic,maxalphanames=60,maxbibnames=10]{biblatex}
% \usepackage[backend=bibtex,style=alphabetic]{biblatex}
% \usepackage{natbib}
\usepackage[ngerman]{babel}
\usepackage[utf8]{inputenc}
\usepackage{ifthen}
\usepackage{xargs}
\usepackage{amsmath}
\usepackage{amsfonts}
\usepackage{amssymb}
\usepackage{graphicx}
% \usepackage{fancyhdr}
\usepackage{tabularx}
\usepackage{geometry}
\usepackage{setspace}
\usepackage[right]{eurosym}
\usepackage[printonlyused]{acronym}
\usepackage{subfig}
\usepackage{floatflt}
\usepackage[usenames,dvipsnames]{color}
\usepackage{xcolor}
\usepackage{colortbl}
\usepackage{paralist}
\usepackage{array}
% \usepackage{titlesec}
\usepackage{parskip}
% \usepackage[subfigure,titles]{tocloft}
\usepackage[pdfpagelabels=true,colorlinks=true,linkcolor=black,anchorcolor=black,citecolor=black,filecolor=black,menucolor=black,runcolor=black,urlcolor=black]{hyperref}
\usepackage{booktabs}
\usepackage[lighttt]{lmodern}
\usepackage{mathabx}

\usepackage{listings}
\lstset{basicstyle=\footnotesize, captionpos=b, breaklines=true, showstringspaces=false, tabsize=2, frame=lines, numbers=left, numberstyle=\tiny, xleftmargin=2em, framexleftmargin=2em}

\geometry{a4paper, top=27mm, left=25mm, right=15mm, bottom=27mm, headsep=10mm, footskip=12mm}

% make lstinline be normalsize but keep displayed listings small
\makeatletter
\lstdefinestyle{mystyle}{
  basicstyle=%
    \ttfamily
    \lst@ifdisplaystyle\scriptsize\fi
}
\makeatother

\lstset{
	language=C++,
	% basicstyle=\ttfamily\scriptsize,
	style = mystyle,
	% basicstyle=\ttfamily\scriptsize,
	keywordstyle=\bfseries\ttfamily,
	% commentstyle=\color{LimeGreen}\ttfamily,
	commentstyle=\color{gray}\ttfamily,
	emphstyle={\color{Purple!90!black}},
	tabsize=4,
	morekeywords={nullptr,nullptr_t,vector,string,std,list,map,pair,set,size_t,endl,cout,move},
	emph={},
	showstringspaces=false,
}

\setcapindent{0pt}
\setkomafont{captionlabel}{\bfseries}

\DefineBibliographyStrings{ngerman}{
	andothers = {{et\,al\adddot}},
}


\DeclareLabelalphaTemplate{
  \labelelement{
    \field[final]{shorthand}
    \field{label}
    \field[strwidth=3,strside=left,ifnames=1,names=1]{labelname}
    \field[strwidth=1,strside=left,names=1-3]{labelname}
  }
  \labelelement{
    \field[strwidth=2,strside=right]{year}
  }
}
\renewcommand*\labelalphaothers{$^\Asterisk\mkern-.8mu$}

\newcommandx{\student}[3][]{
	\def\studentName{#1}%
	\def\studentMatnr{#2}%
	\def\studentStudiengang{#3}%
}

\newcommandx{\MyTitelseite}[8][]{
\thispagestyle{empty}
\ifthenelse{\equal{#1}{}}{
	\includegraphics[scale=0.2]{lib/oth-logo.png}
}{
	\includegraphics[scale=0.2]{lib/oth-logo.png}\hfill\includegraphics[scale=0.5]{#1}
}
\begin{center}
\ifthenelse{\equal{#2}{2}}{ % then
	\vspace*{2cm}
	\Large
	\textbf{Ostbayerische Technische Hochschule Regensburg}\\
	\textbf{Fakultät für Informatik und Mathematik}\\
	\vspace*{2cm}
	\Huge
	\textbf{#3}\\[1em]
	\large
	Zur Erlangung des akademischen Grades des\\
	\ifthenelse{\equal{#3}{Bachelorarbeit}}{Bachelor of Science (B.Sc.)}{Master of Science (M.Sc.)}\\
	\vspace*{1cm}
	\Large
	\textbf{#4}\\
}{ % else
	\vspace*{1cm}
	\Large
	\textbf{#4}\\
	\vspace*{2cm}
	\large
	An der Fakultät für Informatik und Mathematik der\\
	Ostbayerischen Technischen Hochschule Regensburg\\
	im Studiengang\\[2em]
	\textbf{\studentStudiengang}\\[2em]
	eingereichte\\
	\vspace*{1cm}
	\Large
	\textbf{#3}\\[2em]
	\large
	zur Erlangung des akademischen Grades des\\
	\ifthenelse{\equal{#3}{Bachelorarbeit}}{Bachelor of Science (B.Sc.)}{Master of Science (M.Sc.)}
	\vspace*{1cm}
	\Large
}
	\vfill
	\normalsize
	%\newcolumntype{x}[1]{>{\raggedleft\arraybackslash\hspace{0pt}}p{#1}}
	\begin{tabular}{rl}%{6cm}p{7.5cm}}
	    \rule{0mm}{1ex}\textbf{Vorgelegt von:} & \studentName \\
		\rule{0mm}{1ex}\textbf{Matrikelnummer:} & \hspace*{-0.5em}\begin{tabular}[t]{r}\studentMatnr\end{tabular} \\ 
		\ifthenelse{\equal{#2}{1}}{~\\}{\rule{0mm}{1ex}\textbf{Studiengang:} & \studentStudiengang \\[2em]}
		\rule{0mm}{1ex}\textbf{Erstgutachter:} & #5 \\ 
		\rule{0mm}{1ex}\textbf{Zweitgutachter:} & #6 \\[2em]
		\rule{0mm}{1ex}\textbf{Abgabedatum:} & #7 \\ 
	\end{tabular} 
\end{center}
\pagebreak
}



% ein paar nützliche Kommandos

\newcommand\defsec[1]{\label{sec:#1}}
\newcommand\refsec[1]{\ref{sec:#1}}

%\newcommand\todo[1]{\textcolor{red}{[{\textbf{TODO}} #1]}}
\newcommand\todo[1]{} % remove todos

%\newcommand\new[1]{\textcolor{new}{{#1}}}
\newcommand\new[1]{#1} % plain text
%\newcommand\old[1]{\textcolor{gray}{{\sout{#1}}}}
\newcommand\old[1]{} % remove old stuff

\definecolor{lgdv}{rgb}{.80,.23,.13}
\definecolor{agreen}{rgb}{.2,.8,.2}
\definecolor{jorange}{rgb}{.9,.6,.0}
\definecolor{new}{rgb}{.8,.4,.4}

\newcommand\NOTE[3]{\textcolor{#1}{[#2: #3]}}

%\newcommand\kai[1]{\NOTE{lgdv}{Kai}{#1}}
\newcommand\kai[1]{} % remove notes
%\newcommand\niko[1]{\NOTE{jorange}{Niko}{#1}}
\newcommand\niko[1]{} % remove notes

\newcommand\To{\ensuremath{\to}}
\newcommand\hi[1]{\textcolor{red}{#1}}

\let\shortcite=\cite

\addbibresource{bib.bib}

\begin{document}

% ----------------------------------------------------------------------------------------------------------
% Titelseite
% ----------------------------------------------------------------------------------------------------------
\newcommand{\stud}{Emanuel Erben}  % Ihr Name
\student{\stud}
{3174817}						% Matrikelnummer
{Informatik}					% Studiengang

\MyTitelseite{}					% Optionales Logo des extern betreuenden Unternehmens
% \MyTitelseite{pics/mathcomm}	% Optionales Logo des extern betreuenden Unternehmens
{1}								% Style der Titelseite (1 oder 2)
{Bachelorarbeit}				% Typ der Abschlussarbeit (\in {Bachelorarbeit, Masterarbeit})
{Vergleich verschiedener Ansätze der Multiplattform Applikationsentwicklung anhand einer Beispiel-Anwendung}				% Thema der Arbeit						
{Prof.\ Dr.-Ing.\ Kai Selgrad}		% Betreuer
{Prof.\ Dr.\ Name des Zweitgutachters}	% Zweitgutachter
{15.09.\the\year}				% Abgabedatum

\thispagestyle{empty}
~\pagebreak

\setcounter{page}{1} 


4.3 Flutter
ausmisten
ordentlich neu schreiben
4.4 hybrid flutter

5.Auswertung
Überarbeiten

6.Fazit
überarbeiten


% ----------------------------------------------------------------------------------------------------------
% Abstract
% ----------------------------------------------------------------------------------------------------------
% \thispagestyle{empty}
\setstretch{1.15} % Zeilenspacing
\chapter*{Abstract}

\bigskip 


In dieser Arbeit $\ldots$

Siehe auch \url{https://de.wikipedia.org/wiki/Abstract}


\bigskip 

Das vorliegende Beispiel ist eine mehr oder weniger direkte Übersetzung des Papers `Hierarchical Multi-Layer Screen-Space Ray Tracing'~\cite{MaiThiNguyenKim.}, es dient den \LaTeX-Style der Arbeit zu testen und zu illustrieren, es wurde aber nicht darauf geachtet, dass das konvertierte Paper (z.B. von Platzierung und Größe der Bilder) passend für eine Arbeit in diesem Format ist.



\frontmatter

% ----------------------------------------------------------------------------------------------------------
% Inhaltsverzeichnis
% ----------------------------------------------------------------------------------------------------------
\tableofcontents
\vfill
\pagebreak

% % ----------------------------------------------------------------------------------------------------------
% % Abbildungsverzeichnis
% % ----------------------------------------------------------------------------------------------------------
\listoffigures
\vfill
\pagebreak
% 
% % ----------------------------------------------------------------------------------------------------------
% % Tabellenverzeichnis (optional)
% % ----------------------------------------------------------------------------------------------------------
% \listoftables
% \vfill
% \pagebreak

% ----------------------------------------------------------------------------------------------------------
% Listingsverzeichnis (optional; Code nur, wenn wirklich sinnvoll und wichtig)
% ----------------------------------------------------------------------------------------------------------
%\lstlistoflistings
%\vfill
%\pagebreak


% ----------------------------------------------------------------------------------------------------------
% Inhalt
% ----------------------------------------------------------------------------------------------------------
\setstretch{1.15}


\mainmatter

\chapter{Motivation}
Wenn man eine eigene Applikation schreiben will, steht man erstmal vor der Frage wie und wo man anfängt.
Fragen nach der richtigen Programmiersprache bzw. dem richtigen Framework kommen dann auf.

Langezeit gab es hier sehr konkrete Antworten. Da die meisten Leute oft nur einen Computer besaßen, entwickelte man meist nur PC-Anwendungen oder eben Webapplikationen, die jeder von überall aus aufrufen konnte, ohne die Applikation auf seinem eigenen Computer installieren zu müssen.

Mit der Ära der Smartphones jedoch änderte sich das etwas, denn auch hier kann man natürlich Webapplikationen nutzen, jedoch sind diese oft nicht angepasst gewesen.

Daraus entstand eine Zeit wo mobile Applikationen speziell für das eigene Smartphone entwickelt wurden. Die nutzung von Smartphones öffnete auch die Tür, andere Funktionalitäten wie die Kamera zu nutzen. Deshalb wurden dann Applikationen mit Objectiv-C/Swift für iOS und Java bzw. mittlerweile Kotlin für Android endwickelt. Mit deren Hilfe wurden native Anwendungen für das Smartphone entwickelt, die diese neuen mobilen Plattformen optimal ausnutzen konnten.

Wenn man jedoch nun auf mehreren Plattformen seine Applikationen veröffentlichen wollte, so musste man für jede Plattform eine eigene Applikationen in der jeweiligen Programmiersprachen schreiben. Dadurch entstand ein hoher Aufwand. Deswegen wurde bereits früh mit der Entwicklung von sogenannten Cross-Platform Frameworks gestartet, die es ermöglichen sollten, mit nur einem Code möglichst viele Plattformen abzudecken. So kam bereits 2008 etwa PhoneGap heraus. Es war ein Open-Source Framework zur Entwicklung von hybriden mobilen Applikationen. PhoneGap und sein Nachfolger Cordova zusammen waren einige Zeit auch sehr beliebt in diesem Bereich. 2019 hatten beide zusammen einen Marktanteil von kanpp 40\% unter den Cross-platform Frameworks.\cite{statist_CP_Framework}

Trotzdem war die Entwicklung von nativen Apps bisher der Standard. Auch konnte man bereits den ein oder anderen Blogpost von größeren Unternehmen lesen, in dem sie erklären, wie sie versuchten eine  Cross-Plattform-Entwicklungen einzuführen, jedoch aus verschiedenen Gründen einstellten. So etwa auch Airbnb. Sie nutzten das von Facebook mitentwickelte Framework React Native. 2019 hatte dieses Framework einen Marktanteil von 42\%. 
\break
Die Ziele waren einfach:
\begin{enumerate}
    \item Schnelleres entwickeln
    \item Die gleiche Codequalität beibehalten
    \item Nur noch eine Codebasis
    \item Die Entwicklererfahrung verbessern\cite{Airbnb_react_goals}
\end{enumerate}

Jedoch traten während der Entwicklung Probleme auf, die dazu führten, dass sie 2018 wieder zu einem nativen Ansatz zurückkehrten.
\TODO{Hier eventuell noch etwas mehr dazu warum, aber nciht zwingend}

Durch Beispiele wie dieses, waren App-Entwickler lange Zeit skeptisch gegenüber derartigen Lösungen, da dies auch immer mit großen Änderungen und hohen Investitationen verbunden waren.
\TODO{Schauen mit Abbildungsverzeichnis}

\begin{figure}[ht]
  \centering
  \includegraphics[height=7cm,keepaspectratio]{images/cross-platform-mobile-frameworks.png} 
  \caption{Cross-Plattform-Frameworks 2019-2021 \cite{statist_CP_Framework}}
  \label{fig:statista_cross_plattform}
\end{figure}
Abbildung \ref{fig:statista_cross_plattform} zeigt eine Statistik von JetBrains, die die Verteilung von verschiedenen Cross-Plattform-Entwicklungen zeigt. Sie zeigt eindrucksvoll wie schnell sich die Verteilung von Cross-Platform Frameworks ändern kann.
Ein Framework, das hier besonders allerdings besonders heraus sticht ist Flutter. Es ist ein Framework das erst 2017 auf den Markt gekommen ist und innerhalb von gerade einmal 4 Jahren auf einen Marktanteil von 42\% gekomm ist. Von einigen wird es schon als der neue Standard angesehen und Unternehmen wie msg. entwickeln neue Applikationen, wenn nicht offiziell anders gewünscht nur noch in Flutter.
\TODO{Fragen wie man Aussagen aus Vortrag benutzen kann}

Wegen den oben genannten Unsicherheiten und einigen anderen Gründen gibt es allerdings auch viele Entickler die immer noch nativ entwickeln. So entwickelt die Number42 alle ihre betreuten mobilen Applikationen mit den nativen Programmiersprachen. Auch die nativen Programmiersprachen entwickeln sich stetig weiter und bekommen Änderungen, die eine Entwicklung vereinfachen und beschleunigen.



Deswegen soll im folgenden verschiedene Ansätze zur Entwicklung von mobilen Anwendungen untersucht werden und dabei darauf eingegangen werdem, was die Vor- und Nachteile der verschiedenen Ansätze sind und anhand von verschiedenen Kriterien eine Einordnung liefern
\TODO{umschreiben}
Deswegen soll in dieser Arbeit drei unterschiedliche Ansätze mit beispielhaften Frameworks und Programmiersprachen betrachtet werden um am Ende vlt eine bessere Einschätzung geben zu können, wie eine solche Entscheidung ausfallen könnte und Gründe für und gegen bestimmte Ansätze geben.
\TODO{umschreiben}
Daher ist der Fokus dieser Arbeit einmal an einigen konkreten Beispielen zu untersuchen, wie der aktuelle Stand der Technologie hier ist und zu erforschen, welche Einschränkungen die Frameworks besitzen um hier auch eine gewisse Bewertung zu Benutzbarkeit als Applagentur zu untersuchen.

\chapter{Definitionen und Erklärungen}
\chapter{Begriffsklärung}

Bevor genauer in die Arbeit eingestiegenw erden kann, müssen jedoch erstmal ein paar Begriffe geklärt und das Thema etwas abgesteckt werden.

\section{Begriffe}
Unter einer Multiplattform Applikation versteht 

\section{Themenabgrenzung}
\chapter{Entwicklung Nativer Android Application}
Wie bereits erwähnt wird im nativen Bereich lediglich die Android Seite implementiert. Die zugrundeliegenden oft verwendeten Technologien sind in den meisten Fällen auf beiden Seiten vorhanden. Sie haben ihre einzelnen Feinheiten, können aber im Grunde soweit als gleich angesehen werden.

Wenn man eine Android App entwickelt, so muss man wie bei jeder Technologie erstmal ein paar Entscheidungen treffen. Hier gibt es natürlich erstmal die größte Entscheidung der Programmiersprache. Denn selbst wenn man native Applikationen entwickelt so ist dennoch nicht klar definiert bzw. vorgegeben mit welcher Programmiersprache dies geschehen muss. Für iOS sind diese Swift bzw. Objectiv-C. Bei Android gibt es da die Unterscheidung zwischen Java und Kotlin. Diese Entscheidung ist jedoch eher eine kleine. Denn von der Sache her sollte man Kotlin und Swift nutzen, da diese die neuen Sprachen sind die von Google bzw. Apple extra hierfür entwickelt und veröffentlicht wurden. Man kann immer noch die alten Programmiersprachen nutzen und diese sind auch immer noch aktiv in Nutzung und haben viele Forumseinträge und Anleitung für die verschiedensten Problemstellungen. Außerdem bauen die neuen Programmiersprachen grundsätzlich auf den alten auf. So ist Kotlin eine auf Java basierende Programmiersprache die bidirektional übersetzt werden kann. So kann man alten Java Code in die Kotlin App einbinden und automatisch übersetzen lassen. Außerdem wird der Kotlin Code zum bauen der App in Java übersetzt und dann in Java ausgeführt.
Nachdem dies aus dem Weg geschafft ist und wir uns also für Kotlin in diesem Fall entschieden haben, gibt es nun weitere Entscheidungen zu treffen. Bei Android Apps kann man grundsätzlich zwischen zwei verschiedenen Arten des Seitenaufbaus und der Grundarchitektur. 
Diese nennen sich Activitys bzw. Fragments. Sie sind die Entscheidung für eine Art Grundstruktur. Bei Activitys sind die einzelnen Seiten unterschiedliche Klassen und eigenständige Systeme. Jede Activity hat ihren eigenen Context und wird selbständig auf dem Bisldschirm aufgebaut. Danach wird neben ein paar ausnahmen nur zwischen den eigenen Activitys hin und her navigiert um den Ablauf der App nachzubilden.
Bei Fragments haben diese einen geteilten Kontext. Sie werden alle gleichzeitig gebaut und werden dann nur darübergelegt bzw. vom Bildschirm entfernt. Ein großer Vorteil dieser Methode ist, dass man wenn man genügend Platz hat, zwei Bildschirme nebeneinander angezeigt werden können und diese beide normal funktionieren. Bei kleinen Bildschirmen oder unter Umständen kann auch dann nur ein Fragment angezeigt werden. Somit zeigt sich, dass vorallem für Anwendungen die sowohl auf Tablet und Handy Problemlos genutzt werden sollen, Fragments sich gut eignen.
Beide Ansätze haben ihre Vor- und Nachteile.
Für diese Arbeit wurde entschieden auf die Activitys Architektur zurück zu greifen, da sie anfänglich einfacher ist und Schwierigkeiten mit dem Kontextmanagment und dem App Backstack zum navigieren durch die History reibungsloser funktioniert. Außerdem sind Activities aktuell besser dokumentiert und viele Problemlösungen sind für Activities beschrieben.

Nach dem man sich nun für eine Grundarchitektur entschieden hat, müssen ein paar Entscheidungen anhand der Projektarchitektur getroffen werden. Ersteinmal muss man entscheiden ob man eine lokale Offline Datenbank braucht oder ob man eine Onlinedatenhaltung mit eventuell angebundener Serverlogik hat. Natürlich kann auch beides gemacht werden. Jedoch in diesem Fall werden wir nur eine Onlinedatenhaltung benutzen die über eine API angebunden ist. 
\chapter{Entwicklung Hybrider Android Application mit WebView}
\input{inhalt/hybride}
\chapter{Entwicklung Cross-Plattform Application mit Flutter}
Notizen zur Entwicklung mit Flutter:
\TODO{https://www.youtube.com/watch?v=wE7khGHVkYY}
Composition: Composing different existing widgets to a new one

Durch voreinstellung und Erstellen eines Basic screen schon nach wenigen Minuten eine erste laufende "Version" zu sehen.

Hot Reload zeigt sofort sichtbare Änderungen, so dass man gut UI debuggen umbauen und anpassen kann.

Schwer herauszufinden welche Elemente es gibt, wie man sachen konfigurieren kann. Am Anfang nicht sehr intuitiv. Man muss sich auf jeden Fall gut in die Doku einlesen und vlt. auch ein kleines Tutorial machen, bzw. die genauen Sachen googlen.



\section{Flutter 101}
\subsection{Widgets}
Widgets sind die Elemente in Flutter, die genutzt werden um die Applikation aufzubauen.
Es gibt zwei Arten von Widgets:
1. Statefull Widget
2. Stateless Widget
Der Unterschied ist, dass bei Statefull widgets, gibt es noch eine State Klasse, die den aktuellen Status des Elements speichert und bei einer Änderung die neuen Werte übernimmt und damit eine Änderung in dem Element ausführt. Am besten ist dies etwa mit einem Favoriten Button vorstellbar. Drückt man ihn, fügt man ihn zu den eigenen Favoriten hinzu und das Icon des Buttons ändert sich bspw. von einem Herzicon, das nur den Rahmen hat, zu einem was ausgefüllt ist.

Bei der Erstellung gibt es einiges zu beachten. 
Etwa kann man auch ein Widget erstellen wollen, dass in ein eigenes File auslagern, wie es mit tobBar.dart ist. Hier ist jedoch dann die übergabe von Parametern wieder wie bei normalen Klassen übergeben werden. Man kann sie aber auch erstellen wie bei topBar2.dart, wo man die einzelnen Parameter bennenen kann und auch besser nullabillity handln kann.
\TODO{genauer untersuchen. Vorallem Statless Widgets, nullability wie bei topBar, und vieles mehr}

Man kann aber auf jeden Fall sehr einfach ein eigenes Widget schreiben. Dies ist vorallem sinvoll, wenn man eine Komponente hat, die man wiederverwenden möchte. Man erstellt einfach die Widgetclasse mit den entsprechenden 
\subsection{Plugins}

\subsection{Layout}
-Button
-Row
-Wrap
-Container: Dies ist das Wichtigste Widget von Flutter. Sobald man irgendwelche Platzhalter oder innere Abordnung oder sonst was nötig ist, braucht man oftmals das Widget Container. Man kann ganz einfach andere Elemente in ein Container hineinstecken und alle benötigten Sachen hinzufügen. Von Margin Paddings feste Größen und vieles mehr. Einfach Einen Container erstelle und die innenliegenden Elemente in einem ChildWidget hinzufügen.
-Scaffold
\subsection{Aufbau}
-Widgets
-classes
-Stateclasses
-Navigator / Router
-Assets

\subsection{Plugins / pubspec.yaml}
\subsubsection{benutzte Plugins}
\subsubsection{Lehre aus Unterhaltung mit Entwicklern von Plugins}
Bei erstellen erster Seiten und hinzufügen ersten Packages Fehler aufgetreten betreffend Not implemented. Fehlermeldung und Fehler-stack nicht sehr aussagekräftig. Wurde evaluiert dass es an simple gradient text lag. Laut Dokumentationseite des Packages ist es kompatibel für alle Plattformen. Jedoch funtkionierte es nach hinzufügen der Vorgeschlagenen Lösung nur noch auf Android/mobile. Beim Nachforschen tut man sich schwer genaue Gründe herraus zu finden, da die Fehlermeldung und die Dokumentation hierfür leider keinerlei Hinweise gibt. 
Es kam im Verlaufe der Fehlererforschung auch zum Kontakt mit dem Entwickler der sehr hilfsbereit die Fehlerforschung unterstütze.
\TODO{Ergebniss der Issueunterhaltung mit Entwickler}
Bei genauer untersuchung mit dem Entwickler zusammen kam herraus, dass meine Flutterversion eine ältere war, als die, die benötigt wurde, um das Plugin auf Web laufen zu lassen. Nachdem die Flutter version angepasst war, funktionierte es einwandfrei. Da es jedoch bei vielen Packages und auch bei diesem keine Angaben zur mindestens zu nutzenden Flutterversion gab, empfohl ich dem Entwickler doch dies in der Read.me hinzuzufügen, was auch prompt geschah. 
An diesem Beispiel sieht man sehr gut, dass diese Community sehr aktiv und offen ist für Vorschläge und Veränderungen.
Es hat mir jedoch auch gezeigt, dass einige Sachen hier Fehlen:
1. Eine Angabe welche Flutterversion genutzt werden muss, damit alles gut funktioniert. Es gibt zwar eine Angabe unter welcher Version das Plugin getestet wurde, aber manchmal will man vlt. gar nicht sofort auf die neueste Version wechseln.
2. Eine Anzeige welche Packete und Flutterversionen geupgraded werden können, fehlen. Es muss irgendein Anhaltspunkt geben um schnell zu sehen, welche Plugins bzw. Dependencies veraltet sind und was hier geupdated werden kann.


\section{Layouting in Flutter}
SpacerFürTODO
\TODO{Eigentlich hat ja android das auch. Hier sind halt die Layout Files wirklich nur } 
\TODO{reine Layout Files aber sie sind ja auch xml. Siehe vergleich mit HTML+CSS} 
Flutter ist ersteinmal etwas anderes als die native Entwicklung. 
\TODO{Mit iOS abgleichen}
Containerisierter Aufbau Abgleich miut Kapitel zu Layout aus der Flutter - Doku

Flutter ist ähnlich zu html+css Aufbau Man verschachtelt die verschiedenen Elemente ineinander um dann einen Container-Baum zu erhalten.
Dies ist sehr ähnlich zum html baum. Der entsteht wenn man eine Website entwirft und die verschiedenen Elemente verschachtelt. 
Allerdings kommt hier auch gleich noch css mit rein, da man die verschiedenen Attribute der Elemeente direkt beschreibt.
Was anders ist, ist dass man für Layout spezifische Sachen manchmal noch einen Container braucht, während man ja etwa margin auf jedes Element drauf machen kann. Es ist aber dementsprechend auch wieder ähnlich, da etwa flex boxen manchmal in sogenannte divs eingebaut werden müssen, um zu funtktionieren. Was außerdem besonders ist, ist dass es schon sehr vordefinierte Plugins gibt, die das Layout vorgeben können. Einerseits gibt es rows und Columns, die ansonsten oft über divs mit bestimmten Klassen und Frontend UI Framewortks erreichrt werden können und es gibt oft auch schon spezifizierte Sachen für Buttons etwa. Hier kann man dann OutlinedButton, Button, TextButton oder auch IconButton nehmen um das Design hierfür nicht erst selber bauen zu müssen.

\begin{figure}[ht]
  \centering
  \includegraphics[height=7cm,keepaspectratio]{images/sample-flutter-layout.png} 
  \caption{Hierachie einer Menüleiste Quelle: Flutter Doku}
  \label{fig:flutter_layout_tree}
\end{figure}

Abbildung \ref{fig:flutter_layout_tree}

Um Seiten zu erstellen gibt es verschiedene Methoden. Oft muss man die Seiten komplett selbst aufbauen. Das heißt man fängt beim Startelement an. Dem sogennanten Scaffold. Dieser hat verschiedene Elemente. So gibt es AppBar man kann einen Footer hinzufügen und vorallem gibt es einen Body. Der kann dann ein Widget als Child hinzugefügt werden. Hier kann nun über verschiedene Widgets wieder neue Sachen hinzugefügt werden und so Stück für Stück ein Layout gebaut werden. 
Hierfür gibt es Grundsätzlich zwei verschiedene Arten von Widgets. 
1. Widgets die Helfen die Seite zu strukturieren und Layouts zu verfeinern und aufzubauen.
2. Widgets die Teil der Nutzeroberfläche sind und die unter Umständen noch weitere Childs zur ausgestaltung Besitzen, aber auf der Oberfläche angezeigt werden. So hat etwa ein TextButton noch ein Child Text wo dann der Text hinzugefügt wird, der im Button steht. Der Button ist allerdings ebenfalls ein Teil der UI. Anders als Container. Diese fügen eventuell eine Margin, also Abstand zu anderen Elementen hinzu oder ändern\TODO{Tun sie das wirklich?} unter Umständen auch mal den Hintergrund, sie sind allerdings keine Alleinstehenden Elemente und umgeben eigentlich nur das Child oder die Child Elemente.
\chapter{Entwicklung Hybrider Cross-Platform Application mit WebView und Flutter}

\subsection{Entwicklung einer Hybriden App mit wechsel zwischen Flutter und WebView}
In dem behandelten Beispiel wird Phoenix mit einer sogenannten Live Komponente behandelt. Hier gibt es eine Methode namens Live-redirect. Diese kann zwischen zwei Live komponenten hin und her schalten. Dadurch wird kein erneuter URL Aufruf ausgeführt, der dann von einem URL Request mitbekommen wird. Hier findet jedeglich eine AJAX Abfrage statt, die hier nicht abgefangen werden kann.
Zwei Mögliche Lösungen:
1. Website umbauen, sodass ein richtiger Redirect genutzt wird: Nachteil -> Geringere Performance im Browser
2. App so bauen, dass kompletter Live Bereich abgebildet wird: Nachteil -> Views die Nativ gar nicht gebaut werden sollten, müssen unter umständen gebaut werden, um Funktionalität zu gewährleisten.

Weiteres Problem war die Navigation. Da in diesem Fall zur besseren Performance immer die gleiche WebView wiederverwendet wurde, war es ein konstantes Springen zwischen einer WebView und nativen Teilen. Wenn man nun im WebViewBrowser die zurück Taste aufruft, wird innerhalb der WebView zurück gegangen. Hier kommt allerdings das Problem, dass wenn eine Flutter Page dazwischen geschaltet wurde, 
\chapter{Zusammenfassung und Auswertung}
\chapter{Fazit}
\section{Flutter vs. Native Android}
\section{Nativ vs. Hybrid vs. Cross-Plattform}


\vfill
\pagebreak

\appendix

% ----------------------------------------------------------------------------------------------------------
% Abkürzungsverzeichnis (optional, bitte nur wenn sinnvoll)
% ----------------------------------------------------------------------------------------------------------
%\listoftables
\addchap{Abkürzungsverzeichnis}
\begin{acronym}[KDE]
\acro{BA}[BA]{Bachelorarbeit}
\end{acronym}
\vfill
\pagebreak


% ----------------------------------------------------------------------------------------------------------
% Filter fuer Literatur und Quellen definieren
% ----------------------------------------------------------------------------------------------------------

\defbibheading{Literatur}{\addchap{Literaturverzeichnis}} 
\defbibheading{Quellen}{\addchap{Internetquellenverzeichnis}} 
  
\defbibfilter{Literatur}{\not\keyword{online}} 
\defbibfilter{Quellen}{\keyword{online}} 


% ----------------------------------------------------------------------------------------------------------
% Literatur
% ----------------------------------------------------------------------------------------------------------

\printbibliography[heading=Literatur,filter=Literatur] 
\vfill

\pagebreak


% ---------------------------------------------------------------------------------------------------------- 
% Internetquellen 
% ---------------------------------------------------------------------------------------------------------- 

\printbibliography[title = {Quellenverzeichnis}, heading=Quellen,filter=Quellen] 

\pagebreak 

% ----------------------------------------------------------------------------------------------------------
% Anhang
% ----------------------------------------------------------------------------------------------------------
\appendix


\chapter{Anhang}

% \input{inhalt/suppl}


\pagebreak


% % ----------------------------------------------------------------------------------------------------------
% % Eigenschtändigkeitserklaerung
% % ----------------------------------------------------------------------------------------------------------
\include{inhalt/erklaerung}



\end{document}