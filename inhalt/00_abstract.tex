% \thispagestyle{empty}
\setstretch{1.15} % Zeilenspacing
\chapter*{Abstract}

\bigskip 

In den letzten Jahren nutzen viele Menschen vorrangig nicht mehr ihren Computer, sondern vor allem ihre Smartphones und andere mobile Endgeräte, um auf verschiedene Dienste, Konten und Online-Anwendungen zuzugreifen. Um eine Anwendung auf verschiedenen Plattformen veröffentlichen zu können, haben Entwickler unterschiedliche Möglichkeiten. Diese reichen dabei von der Entwicklung einer eigenen Anwendung für jede einzelne Plattform, bis hin zu einem gemeinsamen Code, der für die verschiedenen Plattformen genutzt werden kann. Allerdings ist die Wahl des richtigen Entwicklungsansatzes und der verwendeten Technologie schwer, da jeder Ansatz seine Vor- und Nachteile hat. 

In dieser Arbeit werden vier verschiedene Applikationen, die mit unterschiedlichen Ansätzen implementiert wurden, beschrieben und analysiert. Danach werden die Ergebnisse der verschiedenen Untersuchungen vorgestellt und verglichen. Dabei wird darauf eingegangen, welche Vor- und Nachteile die einzelnen Implementierungen für die betrachteten Kriterien haben. Der Schwerpunkt der Arbeit liegt dabei auf der Android Plattform und speziell den Programmiersprachen Kotlin und Dart. Die vier Implementierungen sind eine native, hybride, cross-kompilierte und eine gemischte Applikation. Letztere besteht aus der Kombination des hybriden und des cross-kompilierten Ansatzes.