% \thispagestyle{empty}
\setstretch{1.15} % Zeilenspacing
\chapter*{Abstract}

\bigskip 

In den letzten Jahren nutzen immer mehr Leute nicht mehr nur noch Computer sondern vor allem auch ihre Smartphones, um auf die verschiedenen Plattformen, Konten und Online-Anwendungen zuzugreifen. Um eine Anwendung auf den verschiedenen Mobilen-Plattformen veröffentlichen zu können haben Entwickler daher oft nur die Möglichkeit eine App für jeden einzelne Plattform zu schreiben oder eine Website zu bauen. In den letzten Jahren hat sich jedoch hier eine dritte Möglichkeit augetan. Hier wird versucht mit einer einzigen Programmierung eine Applikation für möglichst viele Plattformen zu erzeugen. So genannte Cross-Plattform-Applikationen

In dieser Arbeit soll anhand einer bestehenden Website verschiedene Applikationen programmiert werden und dabei verschiedene Entwicklungsanätze und ihre Vor und Nachteile beleuchtet werden. Dabei wird vor allem die Android Plattform und die Programmiersprachen Kotlin bnzw. Dart beleuchtet werden. Dazu werden verschiedene Android und Flutter Applikationen vorgestellt und anhand ausgewählter Kriterien bewertet werden. 
\TODO{Ende muss noch umgeschrieben werden}
Am Ende soll ein Fazit gezogen werden, bei dem vor allem darauf eingegangen werden soll, welche Probleme, bzw. Herausforderungen bestehen und was ein guter Weg wäre um Applikationen zu entwickeln.
