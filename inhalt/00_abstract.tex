% \thispagestyle{empty}
\setstretch{1.15} % Zeilenspacing
\chapter*{Abstract}

\bigskip 

In den letzten Jahren nutzen immer mehr Leute nicht mehr ihren Computer sondern vor allem ihre Smartphones und andere mobile Endgeräte, um auf die verschiedenen Dienste, Konten und Online-Anwendungen zuzugreifen. Um eine Anwendung auf den verschiedenen Mobilen-Plattformen veröffentlichen zu können haben Entwickler verschiedene Möglichkeiten. Die Möglichkeiten reichen dabei von der Entwicklung einer eigenen Anwendung für jede einzelne Plattform bis hin zu einem Code der für die verschiedenen Plattformen genutzt werden kann. Dadurch ist die Wahl des richtigen Entwicklungsansatzes und der Technologie schwer, da jeder Ansatz seine eigenen Vor- und Nachteile hat. 

In dieser Arbeit werden vier verschiedene Applikationen, die mit unterschiedlichen Ansätzen programmiert wurden, beschrieben und analysiert. Danach werden die Ergebnisse der verschiedenen Untersuchungen vorgestellt und verglichen. Dabei wird darauf eingegangen, welche Vor- und Nachteile die einzelnen Implementierungen für die untersuchten Kriterien haben. Daraus werden anschließend einige Fragen erzeugt, die am Anfang einer Entwicklung gestellt werden sollten, um die Wahl eines passenden Entwicklungsansatzes zu unterstützen. Der Schwerpunkt der Arbeit liegt dabei einerseits auf der Android Plattform und speziell den Programmiersprachen Kotlin und Dart. Die vier Implementierungen sind dabei eine native, hybride Web-, Cross-Compilierte und eine Cross-Compilierte-Web Applikation.\TODO{Die Bennenung der einzelnen Ansätze vlt. nochmal ändern}
