Um den Arbeitsaufwand realistisch zu halten und trotzdem ein generelles Bild abgeben zu können. Werde ich die Implementierung wie folgend beschrieben einschränken.
Allgemein soll eine App gebaut werden, die sich an einer bestehende Webanwendung zum Design orientiert und einen Teil der Funktionalität abbilden soll.
\TODO{Hier irgendwie noch etwas zum Sharingproijekt erklären}
Hierzu kommt zumindest ein Login und Datenspeicherung und Datenabruf um eine Kommunikation mit einem Server zu haben. Außerdem soll es eine grundsätzliche Erklärungsseite mit der möglichkeit zum Login geben. Dazu kommen die Möglichkeit Dinge zu erstellen
\TODO{eventuell wird hier auch noch der Nachtrichtenaustausch hinzugefügt, so könnte man ein Massegesystem auch noch hat}.
Bei den veschiedenen Ansätzen wird immer nur auf eine Plattform näher eingegangen. Dies bedeutet dass bei der nativen und der hybriden Entwicklung lediglich auf die Androidseite eingegangen wird. Dies ist jedoch insofern kein Problem, da die jeweiligen Technologien auch für iOS bestehen und somit nur auf die Entwicklungsumgebung und Programmiersprache angapasst werden müsste. Bei der Cross-Plattform Entwicklung wird lediglich bei der Entwicklung darauf geachtet, dass die App für alle Plattformen kompatibel bleibt. So gesehen kann dann die App auf jeder nötigen Plattform veröffentlicht werden.