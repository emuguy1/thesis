\section{Begriffe}
Im folgenden sollen erst ein paar Begriffe geklärt werden, die in dieser Arbeit öfters vorkommen.
\subsubsection{App/Applikation/Anwendung}
Wenn in dieser Arbeit von einer App oder Applikation geredet wird, so ist hiermit eine Anwendung gemeint, die für mobile Endgeräte, vor allem Smartphones mit den Betriebssystemen Android beziehungsweise iOS gebaut wurden.
\subsubsection{Multi-Plattform-Anwendung}
Außerdem ist in dieser Arbeit oft die Rede von Multi-Plattform-Anwendungen. Darunter versteht man eine Anwendung, die nicht nur für eine Plattform geschrieben wurde, sondern für mehrere. Dies beinhaltet dann beispielsweise die verschiedenen Smartphone Plattformen, oder aber auch die verschiedenen PC Plattformen wie Linux oder MacOS. Eine Anwendung muss dabei auch nicht alle Plattformen beinhalten, sonder kann auch nur zwei abdecken. 
\subsubsection{Cross-Plattform-Anwendung}
Ein weiterer Begriff der in der Arbeit häufig fallen wird ist Cross-Plattform-Anwendungen. Es sind Anwendungen, die nicht nur für eine Plattform geschrieben wurde, sondern für mehrere. Auf den ersten Eindruck also exakt das gleiche wie Multi-Plattform. Jedoch ist die Besonderheit, dass bei Cross-Plattform Anwendungen, der Code nur einmal geschrieben wurde. Beide Begriffe beinhalten also erst mal, dass es eine Anwendung für mehrere Plattformen ist, mit dem Unterschied dass bei Multi-Plattform Anwendungen auch mehrere Programmiersprachen und Quellcodes erzeugt werden können, während bei Cross-Plattform nur ein Quellcode für alle geschrieben wird.