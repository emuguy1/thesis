\section{Begriffe}
\subsubsection{App/Applikation/Anwendung}
Im Rahmen dieser Arbeit wird oft von einer App oder Applikation geredet. Damit ist eine Anwendung gemeint, die für mobile Endgeräte entwickelt wurden. Die am häufigsten genutzten mobilen Geräte sind dabei Smartphones mit einem Marktanteil von etwa 62,6\%\footnote{https://gs.statcounter.com/os-market-share}.

\subsubsection{Plattform}
Der Begriff Plattform kann auf unterschiedliche Betriebssysteme, Prozessortypen, Kommunikationsprotokolle oder Hardwaresysteme bezogen werden \cite{2014Mulit_plattform_definition}. In dieser Arbeit werden damit die unterschiedlichen Betriebssysteme bezeichnet. Beispiele für diese sind Android, iOS oder auch MacOS.

\subsubsection{Multi-Plattform-Anwendung}
Als Multi-Plattform-Anwendung werden jene Anwendung bezeichnet, die auf mehreren Plattformen ausgeführt werden kann und dabei eine gleiche oder ähnliche Funktionalität hat \cite{2014Mulit_plattform_definition}. Eine Anwendung muss dabei nicht für alle Plattformen implementiert sein. Ist eine Anwendung auf mehr als einer Plattform verfügbar so kann diese als Multi-Plattform-Anwendung bezeichnet werden.

\subsubsection{Cross-Plattform-Anwendung}
Cross-Plattform-Anwendungen sind ebenfalls, wie Multi-Plattform Anwendungen, auf mehr als einer Plattform ausführbar. Der Unterschied besteht darin, dass Cross-Plattform-Anwendungen eine gemeinsame Codebasis besitzen, die es ermöglicht, die Applikation nur einmalig implementieren zu müssen, jedoch mehrere Plattformen damit abdecken zu können \cite{2014_Cross_plattform}.

\subsubsection{GraphQL}
GraphQL\footnote{https://graphql.org/} ist eine von Facebook vorgestellte Schnittstellen-Technologie für Web-Server. Im Gegensatz zur oft genutzten REST-API kann der Client dabei die Struktur und den Aufbau der Antwort festlegen, wodurch das Senden von unnötigen Daten verhindert wird. Die Kommunikation findet dabei mit Hilfe von JSON statt. Eine Untersuchung von Brito et al \cite{IEEE_GraphQL} ergab, dass durch die Verwendung von GraphQL eine Reduktion der gesendeten Datenmenge um bis zu 99\% erreicht werden kann. 