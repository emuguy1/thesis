\section{Begriffe}
\subsubsection{App/Applikation/Anwendung}
Wenn in dieser Arbeit von einer App oder Applikation geredet wird, so ist hiermit eine Anwendung gemeint, die für mobile Endgeräte, vor allem Smartphones entwickelt wurde.
\subsubsection{Plattform}
Der Begriff Plattform kann auf unterschiedliche Betriebssysteme, Prozessortypen, Kommunikationsprotokolle oder Hardwaresysteme bezogen werden \cite{2014Mulit_plattform_definition}. In dieser Arbeit werden damit jedoch vor allem die unterschiedlichen Betriebssysteme gemeint. Beispiele hierfür sind Android, iOS oder auch MacOS. 
\subsubsection{Multi-Plattform-Anwendung}
Unter Multi-Plattform-Anwendungen versteht man eine Anwendung, die auf mehreren Plattformen ausgeführt werden kann und dabei eine gleiche oder ähnliche Funktionalität hat \cite{2014Mulit_plattform_definition}. Eine Anwendung muss dabei nicht alle Plattformen abdecken, sonder kann auch auf zwei beschränkt werden.
\subsubsection{Cross-Plattform-Anwendung}
Cross-Plattform-Anwendungen sind ebenfalls wie Multi-Plattform Anwendungen, die auf mehr als einer Plattform ausgeführt werden soll. Der Unterschied besteht darin, dass Cross-Plattform-Anwendungen einen gemeinsamen Code besitzen, der verhindert, die gleiche Applikation für die verschiedenen Plattformen in unterschiedlichen Programmiersprachen schreiben zu müssen \cite{2014_Cross_plattform}.