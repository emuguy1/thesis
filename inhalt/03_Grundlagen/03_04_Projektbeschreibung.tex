\section{Projektbeschreibung}
Als Basis für diese Arbeit wird eine bereits bestehende Elixir-Web-Anwendung genutzt. Hierbei handelt  es sich um eine Plattform, die das Ziel hat, das Verleihen und Leihen innerhalb von Bekanntschaftskreisen zu vereinfachen/ ermöglichen. Hierfür kann jeder Nutzer seine eigenen, verleihbaren Gegenstände auf der Plattform eintragen. Zusätzlich können Nutzer sogenannte Kreise erstellen und zusammen mit Freunden bzw. Familie beitreten. Jeder kann dann die Gegenstände sehen, die in den verschiedenen Kreisen vorhanden sind, in denen er Mitglied ist. Zur Kontaktaufnahme gibt es ein Chatsystem, bei dem sich Leute Nachrichten hin und her schicken können, um den Austausch zu organisieren. Zur Kommunikation mit einer Applikation besitzt das System eine GraphQL-Schnittstelle, über die alle benötigten Daten abgefragt werden können. Auch kann der Nutzer über die Schnittstelle authentifiziert werden und neue Nachrichten versendet.