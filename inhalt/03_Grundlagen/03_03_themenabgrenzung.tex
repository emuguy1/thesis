\section{Themenabgrenzung}
\label{cha:3_3abgrenzung}
In dieser Arbeit sollen nicht die verschiedenen Programmiersprachen oder Frameworks innerhalb der einzelnen Ansätze verglichen werden, sondern viel mehr die Ansätze gegenseitig. Daher wird für zwei Implementierung das Framework Flutter genutzt, da dies das aktuell am meisten genutzte Framework\cite{statist_CP_Framework} unter den Cross-Plattform Framworks ist und somit einen guten Repräsentanten für diese Gruppe darstellt. Für die anderen beiden Implementierungen wurde die Android typische Programmiersprache Kotlin\footnote{https://kotlinlang.org/} genutzt. 

Die in dieser Arbeit vorgestellten Implementierungen werden auf die Android Plattform beschränkt. 
Dies hat verschiedene Gründe. Einerseits ist die Programmierung von Android und iOS vergleichbar.
So sagen Goadrich et al \cite{iOSvsAndroid} in einer Untersuchung, welche Plattform für einen Universitätskurs passend wäre, dass beide Umgebungen die für den Kurs benötigten Funktionalität haben.
Weiter sagen sie, dass Studenten dadurch praktische Erfahrung in der Applikationsentwicklung bekommen würden. Sie kommen am Ende auf den Schluss, dass es kein Unterschied macht welche Plattform dabei behandelt werden würde und beide Plattformen helfen, eine Grundlage zur Lehre der Grundideen für mobile Endgeräte Programmierung bilden würden.
Auch in Sachen Performance gibt es zwischen den Plattformen keine größere Unterschiede. So zeigt eine Untersuchung von Győrödi et al \cite{Android_IOS_Performance_comparison}, dass es zwar kleinere Unterschiede in der Performance gibt, es aber keine Plattform gibt, die insgesamt gesehen besser ist, als die andere.
Des weiteren stellt Android die deutlich größere Nutzergruppe dar. So besagen aktuelle Statistiken, dass Android einen Marktanteil von etwa 70\% hat, während iOS nur auf 30\% kommt \cite{statist_OS_worldwide}. Deswegen führt bei der Entwicklung einer Multi-Plattform-Anwendung in den meisten Fällen kein Weg an einer Android Implementierung vorbei.

Die betrachteten Implementierungen umfassen ebenfalls keine Spieleimplementierung. Diese stellen zwar einen signifikanten Teil der in den Appstores vorhandenen Anwendungen dar, jedoch sind dies keine typischen Apps die von Appagenturen oder privaten Entwicklern produziert werden, sondern eher von Unternehmen mit Grafik- oder Spieleentwicklungshintergrund. Außerdem werden die meisten Spiele nicht in Kotlin oder Swift direkt entwickelt, sondern mit Spiele- und Grafikprogrammen wie Unity oder ähnlichem erstellt.

a\TODO{Hier rein, warum die Ansätze gewählt und andere außen vor gelassen}
