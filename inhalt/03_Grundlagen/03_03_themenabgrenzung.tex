\section{Themenabgrenzung}
\label{cha:3_3abgrenzung}
In dieser Arbeit sollen nicht die konkreten Umsetzungen einzelner Ansätze verglichen werden, sondern die Ansätze untereinander. Hierfür wird für zwei der Implementierung das Framework Flutter genutzt, da dies das aktuell am meisten genutzte Framework unter den Cross-Plattform-Frameworks ist und somit einen guten Repräsentanten für diese Gruppe darstellt \cite{statist_CP_Framework}. Für die anderen beiden Implementierungen wurde die Android typische Programmiersprache Kotlin\footnote{\url{https://kotlinlang.org/}} genutzt. 

Die in dieser Arbeit vorgestellten Implementierungen werden auf die Android Plattform beschränkt. 
So untersuchen Goadrich et al \cite{iOSvsAndroid} die verfügbaren Plattformen auf Nutzbarkeit für einen Universitätskurs. 
Dabei stellten sie fest, dass sowohl Android als auch iOS die für den Kurs benötigten Funktionalität bereitstellen und beide Plattformen die Grundlagen der mobilen Anwendungsentwicklung vermitteln. Sie halten daher fest, dass keine der beiden Plattformen zu präferieren ist und die Programmierung vergleichbar ist.
Im Bezug auf die Performance gibt es zwischen den Plattformen ebenfalls keine größeren Unterschiede. So zeigt eine Messung von Győrödi et al \cite{Android_IOS_Performance_comparison}, dass es zwar kleinere Unterschiede in der Performance gibt, sich jedoch keine Plattform von der anderen stark unterscheidet.
Anzumerken sei jedoch, dass Android die deutlich größere Nutzergruppe darstellt. So zeigt eine aktuelle Statistik \cite{statist_OS_worldwide}, dass Android einen Marktanteil von etwa 70\% hat, während iOS nur auf 30\% kommt. Aufgrund dieser Faktoren ist bei den meisten Multi-Plattform-Applikationen die Android Plattform unterstützt.

Die im Rahmen der Arbeit betrachteten Implementierungen sind außerdem keine Spiele. Diese stellen zwar einen signifikanten Anteil der in den App-Stores vorhandenen Anwendungen dar \cite{statist_games_appstore}. Sie sind allerdings keine typischen Apps, die von App-Agenturen oder privaten Entwicklern produziert werden, sondern vorrangig von Unternehmen mit Grafik- oder Spieleentwicklungshintergrund. Ebenfalls werden die meisten Spiele nicht in Kotlin oder Swift direkt entwickelt, sondern mit Spiele- und Grafikengines wie Unity\footnote{\url{https://unity.com/}} oder ähnlichem erstellt. So nutzte laut einer Umfrage \cite{unity_percantage_game_enginge} 61\% Unity und nur 15\% native Grafiken zur Entwicklung von Mobile-Games.

In dieser Arbeit werden vier der sechs verschiedenen in Kapitel \ref{cha:3_2} vorgestellten Ansätze implementiert und verglichen. Dabei wird eine native, eine hybride, eine cross-kompilierte und eine gemischte Applikation erstellt. Eine Implementierung einer Web-Applikation wird hier nicht erklärt und betrachtet, da diese einerseits bereits existierte und andererseits, wie in Kapitel \ref{cha:3_2_web} bereits erwähnt, der Fokus bei der Nutzung mobiler Endgeräte auf installierbaren Anwendungen liegt \cite{report_webusage}. Außerdem wird die Web-Applikation in dem hybriden und dem gemischten Ansatz in die Applikation integriert. Der interpretierte Ansatz wird nicht weiter betrachtet, da er ähnlich zu dem cross-kompilierten Ansatz ist. Beide werden in einer anderen Programmiersprache geschrieben als in der sie ausgeführt werden. Jedoch sind cross-kompilierte Programmierungen in der Ausführung schneller, da der Übersetzungsprozess nur einmal vor der Installation ausgeführt werden muss \cite{interpreted_vs_compiled}.
