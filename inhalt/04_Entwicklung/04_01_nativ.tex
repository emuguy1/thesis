\section{Entwicklung einer nativen Android Applikation}
Die native Entwicklung ist die ursprüngliche der verschiedenen Entwicklungsklassen. Sie nutzt eine von der Plattform unterstützte Programmiersprache. Als Android 2008 vorgestellt wurde, war dies Java.
Jedoch änderte Google die offiziell bevorzugte Programmiersprache 2019 zu Kotlin \cite{android_standard}. 
Kotlin wurde von Jetbrains entwickelt, um einen Ersatz für Java zu finden. 
Sie entwickelten eine Sprache die alle benötigten Funktionen für eine effektive App-Entwicklung hat, jedoch genauso schnell wie Java kompiliert werden kann\cite{medium_Swift_Kotlin}. 
Eine ähnliche Entwicklung fand auch bei Apple statt, bei der von Objectiv-C zu Swift gewechselt wurde. 
Durch die Entwicklung eigener Programmiersprachen, haben die Plattformentwickler die Kontrolle, welche Funktionalitäten hinzugefügt oder entfernt werden und können diese perfekt für ihre Bedürfnisse anpassen \cite{medium_Swift_Kotlin}.

\subsection{Grundlagen}
Eine native Android App besteht aus zwei separaten Teilen.
Der erste Teil ist das Layout. Dabei wird das Design für die UI und deren Elemente mit Hilfe der XML-Notation erstellt.
In der XML Datei wird ähnlich zu einer HTML Datei, die Oberfläche aufgebaut. Als Wurzelelement hat eine Seite dabei ein Layoutelement. Neben einem linearem Layout oder einem Tabellenlayout, wird vor allem das so genannte Constraint-Layout\footnote{\url{https://developer.android.com/guide/topics/large-screens/support-different-screen-sizes}} genutzt. 
Dieses Layout definiert die Positionen der Elemente abhängig von anderen Elementen. Durch diese Eigenschaft eignet es sich gut, um Designs für unterschiedliche Bildschirmgrößen zu erstellen \cite{ConstraintLayout_Android}. Es bietet dabei auch die Möglichkeit, neben der Position auch die Größe, abhängig von anderen Elementen zu definieren.
Die eigentliche Benutzeroberfläche wird durch Verschachtelung und Anordnung von UI-Elementen innerhalb des Grundlayouts gebaut. Diese können dabei sowohl von Android vordefiniert oder von externen Bibliotheken stammen. Die erstellten Dateien beinhaltet lediglich das Layout und keinerlei Funktionalität.

Der zweite Teil der Implementierung ist die Logik und Funktionalität der Anwendung. Dieser Teil kann in drei Haupt-Klassen eingeteilt werden. 
Die erste Klasse, die sogenannte Activity\footnote{\url{https://developer.android.com/reference/kotlin/android/app/Activity}}, ist die Hauptklasse für eine Seite. In ihr wird das Layout aufgerufen und der aktuellen Seite hinzugefügt, den Elemente der UI ihre Funktionalität zugeordnet, in der Oberfläche die benötigten Daten ergänzt und vorhandenen Listen ein Controller zugeordnet \cite{sarkar_android}.
Dieser Controller, genannt ListAdapter\footnote{\url{https://developer.android.com/reference/androidx/recyclerview/widget/ListAdapter}}, ist die zweite Klasse. Sie werden bei dynamischen Listen in Android benötigt, da diese so konzipiert sind, dass Listenobjekte, die außerhalb der sichtbaren Bereiches sind, wiederverwendet werden \cite{recyclerview_android}. Dafür benötigt jede Liste einen eigenen Adapter, der das benötigte Design, Daten und Funktionalität dem Listenelement hinzufügt.
Die dritte Klasse sind ViewModels\footnote{\url{https://developer.android.com/topic/libraries/architecture/viewmodel}}. Diese sind die Schnittstelle zwischen der Datenhaltung und der Activity und verringern damit die Koppelung von Benutzeroberfläche und der Datenhaltung \cite{viewModel_android}. Neben dem asynchronen Sammeln von Daten aus ihren Quellen, kann ein ViewModel außerdem Daten speichern, die eine Konfigurationsänderung überstehen sollen. Konfigurationsänderungen sind dabei Änderungen, die einen Neustart der Activity nach sich ziehen\footnote{\url{https://developer.android.com/guide/topics/resources/runtime-changes}}. Dazu zählt etwa das Ändern der Bildschirmausrichtung. ViewModels können nach dem erneuten Bauen der Activity, dieser wieder hinzugefügt werden. Sie werden erst gelöscht, wenn die dazugehörige Activity geschlossen wird \cite{Android_Room}. Dadurch können sowohl Daten zwischengespeichert werden, als auch verhindert werden, dass diese unnötig oft abgefragt werden.


\subsection{Genutzte Bibliotheken}
\label{cha:4_1_bibliothek}
Zur Kommunikation mit der GraphQL-Schnittstelle der Web-Anwendung, wird Apollo Kotlin\footnote{\url{https://www.apollographql.com/docs/kotlin/}} genutzt. 
Es ist ein Bibliothek, die einen Client zur Kommunikation mit einem GraphQL-Server anbietet und die geschriebenen GraphQL-Queries in Kotlin und Java Modelle umwandelt, die daraufhin wie normale Objekte in der Implementierung aufgerufen werden können.

Anfänglich stellt die GraphQL Erweiterung eine Verbindung mit der GraphQL-API des Servers her, um die Schnittstellendefinition abzufragen und zu speichern. Dadurch kann der Client die Antworten der Schnittstelle, die in einem JSON-Format gesendet werden, automatisch den jeweiligen Datentypen zuordnen. Daher ist der einfache Zugriff auf die Antwort des Servers möglich, da Schreibfehler in der Abfrage der Daten verhindert werden.
Damit der Client in der kompletten Applikation genutzt werden kann, wird er der Applikation als globales Attribut hinzugefügt.

Die zweite genutzte Bibliothek ist Room\footnote{\url{https://developer.android.com/training/data-storage/room}}. Sie ist Teil von Android Jetpack, einem Set an Bibliotheken, dass Entwicklern helfen soll, Apps für verschiedene Android-Versionen zu entwickeln\footnote{\url{https://developer.android.com/jetpack}}. Room\footnote{\url{https://developer.android.com/training/data-storage/room}} ist dabei der Teil, der einen effizienten Datenbank Zugriff ermöglicht.
Die Besonderheit ist, dass Room eine Zwischenebene darstellt, die die geschriebenen SQL Befehle überprüft und validiert \cite{Android_Room}. Des weiteren wird durch die Nutzung von Annotationen redundanter Code verhindert und eine Asynchronität der Datenbankoperationen vereinfacht.

Zum Speichern eines Authentifizierungsstrings für die Kommunikation mit der GraphQL-\break Schnittstelle werden sogenannte Shared-Preferences\footnote{\url{https://developer.android.com/reference/android/content/SharedPreferences}} genutzt. 
Sie sind ein weiterer Teil von Android Jetpack und dienen zur Speicherung von Key-Value Werten innerhalb der Applikation. 
Die Key-Value Werte können aus jeder Klasse der Applikation abgefragt, hinzugefügt, geändert oder gelöscht werden und sind in einfachen XML-Dateien gespeichert \cite{Shared_Preferences_Android}. 
So kann die Erstellung einer eigenen Entität in der Datenbank verhindert werden und der Zugriff auf den Inhalt vereinfacht. 
Sie sind nur für einfache Daten geeignet, da lediglich primitive Datentypen genutzt werden können. 
Daher werden sie für Werte wie Authentifizierung oder Benutzereinstellungen genutzt \cite{Shared_Preferences_Android}.


\subsection{Fazit native Implementierung}
Der native Ansatz ist sehr gut dokumentiert. Für fast jedes Problem, wie etwa der GraphQL Kommunikation, kann eine erprobte und gut funktionierende Lösung gefunden werden. Dazu kommt, dass es viele Bibliotheken gibt, die Google beziehungsweise JetBrains selber anbieten, um die Entwicklung von Apps zu vereinfachen und zu standardisieren.

Ein Vorteil der Layoutentwicklung in diesem Bereich ist, dass das Layout in einer sofortigen Anzeige zusammengebaut werden kann und somit das Aussehen sofort ersichtlich ist. Dies ist dabei sowohl bei der nativen Android als auch der nativen iOS Entwicklung möglich. 
Jedoch ist diese Anzeige nur auf die Oberfläche ohne Funktionalität oder von der Activity hinzugefügte Daten beschränkt. 
Damit also Änderungen an der App betrachtet werden können, muss die App auf einem Emulator oder physischen Gerät gestartet werden.
Da Android keine zuverlässiges Austauschen von Codeteilen im laufenden Programm unterstützt, wird die App beim laden von Änderungen an den Start zurückgesetzt und es muss zu der geänderten Stelle navigiert werden. 
Dadurch wird der Entwicklungsprozess verlangsamt und erschwert.
