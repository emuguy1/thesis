Wie bereits erwähnt wird im nativen Bereich lediglich die Android Seite implementiert. Die zugrundeliegenden oft verwendeten Technologien sind in den meisten Fällen auf beiden Seiten vorhanden. Sie haben ihre einzelnen Feinheiten, können aber im Grunde soweit als gleich angesehen werden.
\TODO{hier brauchen wir eine Quelle}

Wenn man eine Android App entwickelt, so muss man wie bei jeder Technologie erstmal ein paar Entscheidungen treffen. Hier gibt es natürlich erstmal die größte Entscheidung der Programmiersprache. Denn selbst wenn man native Applikationen entwickelt so ist dennoch nicht klar definiert bzw. vorgegeben mit welcher Programmiersprache dies geschehen muss. Für iOS sind diese Swift bzw. Objectiv-C. Bei Android gibt es da die Unterscheidung zwischen Java und Kotlin. Diese Entscheidung ist jedoch eher eine kleine. Denn von der Sache her sollte man Kotlin und Swift bevorzugen, da diese die neuen Sprachen sind die von Google bzw. Apple extra hierfür entwickelt und veröffentlicht wurden. Man kann immer noch die alten Programmiersprachen nutzen und diese sind auch immer noch aktiv in Nutzung und haben viele Forumseinträge und Anleitung für die verschiedensten Problemstellungen, jedoch haben sie oft keine Weiterentwicklung und  Außerdem bauen die neuen Programmiersprachen grundsätzlich auf den alten auf. So ist Kotlin eine auf Java basierende Programmiersprache die bidirektional übersetzt werden kann. So kann man alten Java Code in die Kotlin App einbinden und automatisch übersetzen lassen. Außerdem wird der Kotlin Code zum bauen der App in Java übersetzt und dann in Java ausgeführt.
Nachdem dies aus dem Weg geschafft ist und wir uns also für Kotlin in diesem Fall entschieden haben, gibt es nun weitere Entscheidungen zu treffen. Bei Android Apps kann man grundsätzlich zwischen zwei verschiedenen Arten des Seitenaufbaus und der Grundarchitektur. 
Diese nennen sich Activitys bzw. Fragments. Sie sind die Entscheidung für eine Art Grundstruktur. Bei Activitys sind die einzelnen Seiten unterschiedliche Klassen und eigenständige Systeme. Jede Activity hat ihren eigenen Context und wird selbständig auf dem Bisldschirm aufgebaut. Danach wird neben ein paar ausnahmen nur zwischen den eigenen Activitys hin und her navigiert um den Ablauf der App nachzubilden.
Bei Fragments haben diese einen geteilten Kontext. Sie werden alle gleichzeitig gebaut und werden dann nur darübergelegt bzw. vom Bildschirm entfernt. Ein großer Vorteil dieser Methode ist, dass man wenn man genügend Platz hat, zwei Bildschirme nebeneinander angezeigt werden können und diese beide normal funktionieren. Bei kleinen Bildschirmen oder unter Umständen kann auch dann nur ein Fragment angezeigt werden. Somit zeigt sich, dass vorallem für Anwendungen die sowohl auf Tablet und Handy Problemlos genutzt werden sollen, Fragments sich gut eignen.
Beide Ansätze haben ihre Vor- und Nachteile.
Für diese Arbeit wurde entschieden auf die Activitys Architektur zurück zu greifen, da sie anfänglich einfacher ist und Schwierigkeiten mit dem Kontextmanagment und dem App Backstack zum navigieren durch die History reibungsloser funktioniert. Außerdem sind Activities aktuell besser dokumentiert und viele Problemlösungen sind für Activities beschrieben.

Nach dem man sich nun für eine Grundarchitektur entschieden hat, müssen ein paar Entscheidungen anhand der Projektarchitektur getroffen werden. Ersteinmal muss man entscheiden ob man eine lokale Offline Datenbank braucht oder ob man eine Onlinedatenhaltung mit eventuell angebundener Serverlogik hat. Natürlich kann auch beides gemacht werden. Jedoch in diesem Fall werden wir nur eine Onlinedatenhaltung benutzen die über eine API angebunden ist. 


Ärgerlich bei der Nativ Entwicklung und Graphql war, dass neu erzeugte Querys erst durch ein Build durchlauf erzeugt wurden, und so etwas längert braucht zur Entwicklung.
\section{Layout}
\section{externe Bibliotheken}
\section{Fazit Nativ}