Ein zweiter Ansatz auf den wir im Folgenden eingehen wollen sind hybride Applikationen. Dabei wird eine Website in einer nativen App angezeigt. So muss man lediglich eine Website schreiben die auch gut auf verschiedenen Bildschirmgrößen angezeigt werden kann. Danach muss man nur noch ein paar Konfigurationen an einer nativen App vornehmen, um eine reibungslose Funktionalität zu gewährleisten und schon hat man eine Anwendung die sowohl als Website im Browser als auch als App genutzt werden kann. Jedoch gibt es hier unterschiedliche Abstufungen die eine Hybride Applikation haben kann.

Alle besprochenen Abstufungen sind ebenfalls auf iOS implementierbar, haben jedoch andere Details oder Implementierungen. 

\subsection{Stufe 1: Reine WebView}
Bei dieser Art von Applikation besteht die Applikation lediglich aus einem Container in dem eine Webseite aufgerufen wird. Hierfür fügt man der Applikation in dem Hauptlayout ein WebView Element hinzu. Im Anschluss wird diese in der Activity konfiguriert. Neben Designkonfigurationen wie der Platzierung der Scrollbar, werden aber vor allem der Browser konfiguriert. Dabei muss etwa JavaScript aktiviert werden, wenn die Webseite JavaScript benutzt. Um weitere Konfigurationen zum Verhalten während der Nutzung machen zu können, muss außerdem ein \textsl{webChromeClient} gesetzt werden. Außerdem muss noch ein \textsl{userAgentString} gesetzt werden ohne den einige Sicherheitsfunktionen wie Google Login nicht funktioniert. Darüber wird definiert mit welchem Browser auf die Webseite zugegriffen wird. Man kann dadurch auch Funktionalität hinzufügen, etwa bestimmte Sachen ein oder aus zu blenden oder eine andere Menüleiste anzuzeigen, wenn ein spezifischer UserAgent genutzt wird.

Ein weiterer wichtiger Punkt ist es, externe Links abzufangen und zu einem externen Tab weiterzuleiten. Dazu wird ein eigener ChromeClient erzeugt werden.  In diesem überschreibt man die Funktion \textsl{shouldOverrideUrlLoading}. In dieser kann man die aufgerufene URL analysieren und entscheiden was passieren soll. Als Rückgabewert wird hier ein Booleanwert erwartet. Wenn false zurückgegeben wird, dann wird die URL nicht überschrieben und die URL wird in der WebView geladen.  Wird true zurück gegeben, wird die URL nicht geladen. In diesem Fall wird ein \textsl{CustomTabsIntent} gestartet und die URL in diesem geladen. Es ist also ein Externes Browsertab ohne dabei die App zu verlassen. Es öffnet dabei eine von Android bereitgestellte Activity, die eindeutig als extern identifiziert werden kann und von der man mit einem Schließen Knopf in der linken oberen Ecke wieder zu der vorherigen Activity zurückkehren kann.

Eine letzte Konfiguration die noch vorgenommen werden muss ist das Behandeln des Zurück-Buttons. 
Denn in der Standardkonfiguration wird mit dessen Betätigung die aktuelle Aktivität beendet und zu der vorherigen zurückgekehrt. 
Wenn keine weitere Aktivität in der Vergangenheitshistorie vorhanden ist, wird die Anwendung beendet. 
Da es in diesem Fall nur eine einzige Aktivität gibt, würde die App sofort geschlossen werden, sobald die Zurücktaste gedrückt wird. 
Daher muss dies abgefangen werden und je nach dem ob innerhalb der WebView noch zurückgegangen werden kann die Funktion \verb|goBack| des Webviews aufgerufen werden. 
Hierfür muss die Funktion \textsl{onBackPressed} überschrieben werden. Wenn die Webview keine weiteren Seiten in ihrem Verlauf hat, dann wird die Methode der Elternklasse aufgerufen und daraufhin die App geschlossen.

\subsection{Stufe 2: Webview gemischt mit Nativer Funktionalität}
Um die Applikation noch mehr wie eine App aussehen zu lassen kann man als nächste Stufe einige native Steuerelemente hinzufügen.So kann man etwa ein Menü über oder unter der WebView eine gibt weitere Möglichkeiten um vorher Erläuterte Implementierung noch etwas mehr anzupassen. 
So kann man das Design so anpassen, dass etwa eine Menüleiste ober beziehungsweise unterhalb der WebView angezeigt wird. Auch kann man Buttons hinzufügen die über dem WebView anzeigen, die dann etwa eine bestimmte Webseite laden oder zur Startseite zurückkehren. Um eine andere URL zu öffnen braucht man lediglich auf dem WebView Objekt die Methode \verb|loadURL| aufrufen, die dann die entsprechend spezifizierte URL aufruft. 

\subsection{Stufe 3: Einbau eines Interfaces in Website um native Funktionalität aufzurufen}
Neben Design und Steuerelementen kann auch noch eine weitere Stufe eingearbeitet werden. Hierzu ist jedoch eine spezielle JavaScript Implementierung des Webservers notwendig. So kann man einen sogenannten \verb|Android Adapter| in den JavaScript Code der Webseite einbauen. Dazu muss man jedoch die passende Webseiten Technologie nutzen, weshalb dies in dieser Arbeit außen vor gelassen wird. Jedoch soll dennoch ein paar Sachen dazu erklärt werden.

Hierzu muss zunächst ein \verb|WebAppInterface| erzeugt werden. In diesem kann man dann Funktionen definieren, die dann von einem Javascript aufgerufen werden können. In diesen Funktionen kann man dann native Funktionalitäten wie etwa eine Bildschirmbenachrichtigung einbauen. Auf der Appseite muss man außerdem das erzeugte Interface der WebView mit der Funtkion \verb|addJavascriptInterface| als Parameter übergibt man dazu das Interface und den Namen des Adapters in der Webanwendung \cite{JavaScript_android_webview}.

In der Webandwendung muss man nun lediglich die Funktion über dem in der App spezifizierten Interface aufrufen. Dabei muss man nicht mal das Interface initialisieren, da das WebView dies zur Verfügung stellt.

Wie man sieht, kann man dadurch native Funktionalität zu sowohl einer Android oder auch iOS App mit den entsprechenden Implementierungen hinzufügen, jedoch hat dieser auch seine Probleme. Einen ersten gibt die Android Dokumentation selber. Denn durch diese Schnittstelle erzeugt man auch eine mögliche Schwachstelle die ausgenutzt werden kann um Zugriff zu der App zu erhalten. Um dies zu verhindern, empfiehlt Google, dass man lediglich vertrauensvolle Quellen für externe Bibliotheken nutzt, um dies zu verhindern. Weiter empfiehlt es nur die eigene Webseite in der WebView anzuzeigen um zu verhindern, dass andere Webseiten eine Schwachstelle ausnutzen.\cite{JavaScript_android_webview}  Ein weiteres Problem ist, dass man in diesem Fall die Webanwendung so programmieren muss, dass es eine Unterscheidung je nach Plattform in der Funktionsweise gibt. Dadurch entsteht doppelter Code innerhalb der Webseite. Das größte Problem ist jedoch, dass nicht jede Web-Technologie zur Entwicklung geeignet ist. Aus diesem Grund wurde diese Stufe nicht implementiert. Es gibt jedoch noch eine weitere Stufe um native Funktionalität zu erreichen.

\subsection{Stufe 4: Mischen zwischen Nativer und Webansicht}
Diese Stufe ist eine Mischung aus WebView und nativen Seiten. Webseiten sind einfach und schnell zu entwickeln und eignen sich hervorragend um statische Informationen und einfache Listen anzuzeigen. Jedoch funktionieren einige Komponenten einer Anwendung schneller beziehungsweise besser, wenn rein native Funktionalität genutzt wird. 
Diese sind dabei nicht immer nur Funktionalität wie etwa eine Overlaybenachrichtung, sondern auch ganze Nutzeroberflächen. Für diesen Fall wäre es gut, wenn man für bestimmte Seiten eine native Seite hat und anhand von bestimmten Entscheidungsparametern entweder eine Seite auf der WebView oder eine nativ programmierte Seite angezeigt.

Hierfür braucht man natürlich im Gegensatzt zu den vorherigen Stufen eine Art um mit dem Backend der Webseite zu kommunizieren und braucht eine Weise, wie die App entscheidet was als nächstes Angezeigt werden soll.
Da in Kapitel 4.4 noch genauer auf eine derartige Implementierung eingegangen wird, wird nun lediglich eine Für diese Art typisches Framework etwas vorgestellt und erläutert werden.

\subsubsection{Exkurs: Hotwire}
Ein Framework was typisch für diese Entwicklungsklasse und Abstufung ist, ist Hotwire in Verbindung mit ihrem selbst entwickelten Turbo Adapter, urch den eine Applikation auf Android und iOS gebracht werden soll.

Hier schreibt man die Website bereits in einer passenden Technologie/ Sprache, um im Anschluss eine App zu entwickeln, die native und Web Ansichten mischt, um einen möglichst effizienten aber trotzdem nativ aussehenden Look für eine App zu erhalten. Hotwire und Turbo. als Beispiel
Diese Technologie wurde von der Firma Basecamp entwickelt, die schon lange für ihre Plattform eine schnelle Lösung für ihre Websiten gesucht haben. Daher haben sie Hotwire erfunden. Hotwire steht für HTML over Wire. Was das bedeuted ist, dass die Website nicht andauernd komplett neugeladen wird, sondern nur die teile, die auch erneuert werden müssen. Dazu kommt noch einige technologie die das caching und die verarbeitung der benötigten javascript Files handelt und raus kommt eine Lösung um schnell und performant Websiten zu laden.
Um das ganze nun auf ein Smartphone zu bringen war die Idee entsanden, dass man hier nicht eine komplette native App schreiben wollte sondern eben eine WebView herzunehmen, in die aber Native Teile gemischt weren, um die Erfahrung der Nutzer möglichst intuitiv zu gestalten. 



\subsection{Fazit(muss wsl. wo anders hin) zu hybrid}
Auch wenn es auf dem ersten Moment super einfach scheint und es so wirkt, als könnte man so jede Website einfach in eine Applikation stecken, ohne dass man groß wiederholenden Code hat, bzw. dass man mehrmals die komplette Anwendung schreiben muss, sondern ja nur einmal und je nach grad noch einmal Aufwand hineinstecken muss, aber nie in dem Umfang wie für eine komplette native App. 
Es hat auch Nachteile und Gründe, warum dies nicht gern genutzt wird.
Ein aller erster großer Nachteil ist eine reine online Funktionalität. 