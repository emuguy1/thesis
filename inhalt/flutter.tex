Notizen zur Entwicklung mit Flutter:


Durch voreinstellung und Erstellen eines Basic screen schon nach wenigen Minuten eine erste laufende "Version" zu sehen.

Hot Reload zeigt sofort sichtbare Änderungen, so dass man gut UI debuggen umbauen und anpassen kann.

Schwer herauszufinden welche Elemente es gibt, wie man sachen konfigurieren kann. Am Anfang nicht sehr intuitiv. Man muss sich auf jeden Fall gut in die Doku einlesen und vlt. auch ein kleines Tutorial machen, bzw. die genauen Sachen googlen.