Notizen zur Entwicklung mit Flutter:


Durch voreinstellung und Erstellen eines Basic screen schon nach wenigen Minuten eine erste laufende "Version" zu sehen.

Hot Reload zeigt sofort sichtbare Änderungen, so dass man gut UI debuggen umbauen und anpassen kann.

Schwer herauszufinden welche Elemente es gibt, wie man sachen konfigurieren kann. Am Anfang nicht sehr intuitiv. Man muss sich auf jeden Fall gut in die Doku einlesen und vlt. auch ein kleines Tutorial machen, bzw. die genauen Sachen googlen.

Bei erstellen erster Seiten und hinzufügen ersten Packages Fehler aufgetreten betreffend Not implemented. Fehlermeldung und Fehler-stack nicht sehr aussagekräftig. Wurde evaluiert dass es an simple gradient text lag. Laut Dokumentationseite des Packages ist es kompatibel für alle Plattformen. Jedoch funtkionierte es nach hinzufügen der Vorgeschlagenen Lösung nur noch auf Android/mobile. Beim Nachforschen tut man sich schwer genaue Gründe herraus zu finden, da die Fehlermeldung und die Dokumentation hierfür leider keinerlei Hinweise gibt. 
Es kam im Verlaufe der Fehlererforschung auch zum Kontakt mit dem Entwickler der sehr hilfsbereit die Fehlerforschung unterstütze.
\TODO{Ergebniss der Issueunterhaltung mit Entwickler}