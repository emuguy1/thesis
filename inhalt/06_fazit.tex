\section{Flutter vs. Native Android}
\section{Nativ vs. Hybrid vs. Cross-Plattform}

\section{Entscheidung zu unterschiedlichen Ansätzen}
Existiert eine Webseite

\section{Future Work}
Die Evaluation von Cross-Plattform Frameworks wurde in diesem Fall mit dem aktuell beliebtesten Framework Flutter gemacht. In einer zukünftigen Arbeit wäre eine Untersuchung weiterer Technologien sehr interessant. Dazu kommt, dass die untersuchten Technologien weiterentwickeln. So ist allein in der Zeit der Erstellung der Arbeit bei Flutter ein großer und 3 kleine Release herausgekommen. Dabei sind unter anderem Performance Verbesserungen oder auch neue Plattformen hinzugefügt wurden. Deswegen ist eine wiederholte Betrachtung nach einiger Zeit durchaus interessant zu sehen, wie die Ergebnisse dann aussehen würden.

------

1. Eine auf mobile angepasste UI. - Auch wenn es heutzutage in fast jedem Framework und vor allem in den gängigen UI-Frameworks verschiedene Ansätze gibt, die eine recht nutzerfreundliche Version für Mobilgeräte anbieten, oder manchmal auch sogar komplett eigenständige Oberflächen für mobil angezeigt werden, so kann es doch sinnvoll sein, nochmal extra angepasste UI in Form einer Applikation nativ für die Geräte zu entwickeln. Dadurch kann man gezielt Oberflächen für die Plattformen bauen und dabei auf Plattform eigene Design Unterstützungen zugreifen. Die eine Bedienung um einiges besser machen.
2. Nutzung von Hardwarefunktionalität. -  Ein noch viel wichtigerer Punkt ist es, Funktionalität die mobile Endgeräte anbieten, zu nutzen, die etwa auf einem PC nicht nutzbar sind. Dazu zählen unter anderem GPS-Nutzung, Kamerafunktionalitäten, Bluetooth-Verbindungen,.... Diese können zwar manchmal auch durch einen PC geboten sein, jedoch ist es hier nicht gegeben, während man bei einem Smartphone sicher davon ausgehen kann, dass die Kamera genutzt werden kann. So können einige Geschäftsprozesse der Applikationen vereinfacht oder umgestaltet werden, so dass eine Nutzung der Applikation für den Nutzer einfacher wird, bzw. es können auch neue Funktionalitäten daraus ergeben, die es eventuell davor nicht gab.


Eine der ersten Fragen die dabei im Raum steht. 
Gibt es bereits eine Webversion der Applikation.
Wenn nicht, so kann es von Anfang an spannend sein, ein Framework zu wählen, dass wie Flutter eine Cross-Plattform-Applikation erzeugt, wo auch eine Webversion mit gehostet werden kann.
Falls es bereits eine Webversion geben, so geht der Weg eher in Richtung von Nativen- bzw. Hybriden Applikationen, da meißt nur noch einzelne Teile der Applikation entwickelt werden müssen. Dies ist jedoch auch kein Grund eine Cross-Plattform-Entwicklung auszuschließen, da dadurch nur ein Code geschrieben werden muss um etwa beide vorherschenden mobilen Plattformen abzudecken: iOS und Android.

'''Hier könnte man so ein Art Diagramm machen mit Fragen und dann Entscheidungswegen. So nach dem Motto finde dein Framework zur entwicklung einer mobilen Applikation.'''
\section{Ausblick und Zukunft}