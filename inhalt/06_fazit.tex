Im Folgenden werden die Erkenntnisse der Arbeit noch einmal zusammengefasst und einige Erkenntnisse erwähnt werden. Im Anschluss soll ein Ausblick auf weitere Forschung und mögliche Verbesserungen gegeben werden.

\section{Zusammenfassung}
In dieser Arbeit wurden verschiedene Methoden genauer vorgestellt, um eine Multi-Plattform Anwendung für mobile Endgeräte zu Entwickeln. Dabei wurde die Implementierung auf eine Smartphone Android Implementierung beschränkt und vier verschiedene Ansätze programmiert. Dabei wurde einerseits eine native Android Implementierung mit Kotlin, eine hybride Applikation mit Kotlin, eine Cross-Compilierte Flutter Applikation und ein gemischter Ansatz einer hybriden Cross-Compilierten Flutter Applikation. Nach der Implementierung dieser Ansätze wurden diese in verschiedenen Kriterien untersucht. So etwa die Performance. Dabei konnte festgestellt werden, dass in dem betrachteten Fall die native Implementierung die am schnellsten startende und am Ressourcenschonendste Implementierung ist, während die Mix Implementierung hierbei am schlechtesten abschnitt. Lediglich beim Rendern der Oberfläche war nativ hier nicht der schnellste Ansatz. Dafür ist er im Bereich der Entwicklergemeinschaft durch die längere Existenz und größere Verbreitung im Vorteil. Dabei gab es mehr beantwortete Fragen und teilweise besser ausgearbeitete Lösungen für Probleme wie der Serverkommunikation über GraphQL. Dabei zeigte sich jedoch auch das große Interesse das an Flutter existiert und der diesen Unterschied über die Zeit verringern könnte.

Der Cross-Kompilierte Ansatz mit Flutter zeigte zwar eine etwas verringerte Performance, diese war jedoch in dem hier untersuchten Beispiel nicht groß. Dafür konnte es in den meisten anderen Kriterien entweder mit der nativen Implementation mithalten oder sogar besser abschneiden und bietet somit eine echte Alternative zur nativen Entwicklung. Lediglich in der Qualität einiger Plugins können die recht neuen Bibliotheken noch nicht mit denen für Kotlin oder andere Programmiersprachen mithalten. Dies wird sich jedoch verbessern, wenn Flutter weiterhin seine Community aufbauen kann, wie bisher. Außerdem ist die Wiederverwendbarkeit des Codes im Vergleich zur hybriden Applikation hier nicht gegeben. Dadurch ist bei einem Wechsel der Technologie zu einer anderen, eine umfangreiche Neuprogrammierung nötig.
Jedoch kann durch eine Entwicklung mit Flutter oder anderen Cross-Compilierten Applikationen bereits während der Entwicklung viel Geld gesparrt werden, da fast alles nur ein einziges mal implementiert werden muss.

Dies ist bei den hybriden Applikationen deutlich besser, da der Großteil mit Web-Technologie geschrieben ist. Dadurch kann schnell und mit begrenzten Aufwand auf eine andere Technologie gewechselt werden. Dazu kommt durch die Nutzung der Web Technologie eine deutlich höhere Entwicklergemeinschaft in diesem Bereich. Hier muss lediglich bei der umgebenden Technologie auf diesen Faktor geschaut werden. Dieser ist aber, wie bereits erwähnt, deutlich einfacher auszutauschen. Jedoch muss hier für jede Plattform eine eigene Version des Plattformcodes geschrieben werden. Dadurch erhöht sich der Programmieraufwand und dadurch auch die Kosten.

Wie durch die verschiedenen untersuchten Kriterien außerdem klar geworden ist, konnte der hier vorgestellte gemischte Ansatz aus hybrider und Cross-Compilierter Applikation in den meisten Kriterien nicht überzeugen. So hatte er eine spürbar schlechtere Performance, höheren Programmieraufwand, eine deutliche Einschränkung in der Plattformabdeckung oder auch in der Dokumentation. Jedoch kann dieser Ansatz von Vorteil sein, wenn eine eine bestehende Webseite in eine Cross-Plattform App mit beispielsweise Flutter umgebaut werden soll. Dabei kann ein langsamer Übergang von Webseite zu einer eigenen App erreicht werden. Ein zweiter Ansatz ist, wenn statische Seite, die keine Plattformfunktionalität angezeigt werden sollen, kann eine Webseiten Version genutzt werden, um eine Reimplementierung zu umgehen.

Wie also im laufe der Arbeit klar geworden ist, gibt es verschiedene Ansätze und noch viel mehr unterschiedliche Frameworks, die innerhalb der einzelnen Cross-Plattform Ansätzen ausgewählt werden können. Dabei haben jede unterschiedliche Vor und Nachteile. Jedoch kann festgestellt werden, dass Flutter als eine mögliche Alternative zu einer nativen Entwicklung in Betracht kommt und eine native Entwicklung nicht mehr als einziger Standard in der Entwicklung von Multi-Plattform-Anwendung gesehen werden darf.

\section{Ausblick}
Natürlich deckt diese Arbeit nicht alle Arten der möglichen Entwicklung ab. So wurde etwa für die Klasse der hybriden Applikation eine Entwicklung mit Hilfe eines WebContainers genutzt, auf dem lediglich eine im Internet verfügbare Webseite angezeigt wird. Jedoch gibt es hier viele andere und mit sicher auch größere Frameworks wie etwa Cordova oder Ionic. Auch wurde auf drei Klassen und eine Mix-Implementierung begrenzt. Deswegen wäre eine umfangreichere Testgruppe interessant um ein vollständiges Bild zu erhalten.

Auch waren die in dieser Arbeit implementierten Applikationen stark in ihrem Umfang eingeschränkt. So wäre beispielsweise der Vergleich von Animationen interessant, da einige Arbeiten hier einen großen Vorteil der nativen Applikationen sehen, aber auch Flutter hier in neuen Updates viel ändert um diese Aufgabe gut zu meistern. Daher wäre es auch interessant, die Kriterien in einigen Jahren noch einmal neu auszuwerten, da hier jedes Jahr größere und kleinere Änderungen herauskommen, die die Ergebnisse wieder stark ändern könnten. So kam allein während der Erstellung dieser Arbeit etwa Flutter 3 heraus, das neben der Unterstützung weiterer Plattformen auch Performanceverbesserungen in unterschiedlichen Bereichen enthält \cite{flutter3}. 
