Im Folgenden soll ein Fazit zur Arbeit gezogen und ein Ausblick auf weitere Forschung und mögliche Verbesserung gegeben werden.

\section{Fazit}
In dieser Arbeit wurden die verschiedenen Ansätze zur Entwicklung einer Multi-Plattform-Anwendung für mobile Endgeräte erklärt. Danach wurde eine Implementierung für vier Ansätzen vorgestellt. Die vier Implementierungen sind eine native Implementierung mit Kotlin, eine hybride Implementierung mit Kotlin, eine cross-kompilierte Implementierung mit Flutter und eine gemischte Implementierung mit Flutter. Letztere ist eine Kombination des hybriden und des cross-kompilierten Ansatzes. Bei der Vorstellung der vier Implementierungen wurde auf die Grundlagen der Implementierung und die genutzten Bibliotheken eingegangen. Zusätzlich wurde jede Implementierung abschließend bewertet. Danach wurden einige Kriterien analysiert, die bei der Entscheidung für einen Ansatz beachtet werden sollten. Dabei wurde anhand der getroffenen Implementierung und anderer Quellen ein Vergleich zwischen den vier Implementierungen gezogen.

So wurde etwa die Performance der vier Implementierungen untersucht. Dabei konnte festgestellt werden, dass die native Implementierung tatsächlich die beste Performance hat. Jedoch wurde auch gezeigt, dass das Flutter Framework durch die Nutzung einer eigenen Render-Engine beim Anzeigen der Nutzeroberfläche schneller war als die native Implementierung. Außerdem stellte sich heraus, dass die Nutzung eines Web-Containers zu einer Performanceverschlechterung führt. 

Als weiteres Kriterium wurde die Entwicklergemeinschaft betrachtet.
Hier konnte festgestellt werden, dass Flutter als Cross-Compiler Framework eine teilweise höhere Beachtung auf GitHub erhält als die nativen Programmiersprachen. 
Sowohl bei den gestellten und beantworteten Fragen auf Stackoverflow, als auch bei der verfügbaren Dokumentation konnte Flutter mit Kotlin und Swift mithalten.

Bei der Untersuchung des Programmieraufwandes sind die Vorteile der Cross-Plattform-Lö-sungen deutlich zu erkennen. Denn anstatt einer vollständigen Implementierung auf jeder einzelnen Plattform erstellen zu müssen, wie dies bei der nativen Entwicklung der Fall ist, können bei den anderen Implementierungen Teile des Codes für die anderen Plattformen wiederverwendet werden. Bei Nutzung des cross-kompilierten Ansatz musste sogar lediglich eine Codebasis geschaffen werden, um eine Applikation für die verschiedenen Plattformen erstellen zu können. So kann der Programmieraufwand und die damit verbundenen Kosten stark verringert werden. Dieser Unterschied wächst dabei mit jeder weiteren unterstützten Plattform.
Zusätzlich konnte festgestellt werden, dass eine hybride Implementierung mit Einbindung einer Webseite, das Verteilen von Updates an Kunden stark vereinfacht. Dabei kann der zeitintensive Review-Prozess der verschiedenen App-Stores umgangen werden. Dies ist jedoch nicht der Fall, wenn der Web-Code lokal gespeichert werden würde.

Bei der Betrachtung der Benutzeroberflächen konnte festgestellt werden, dass die hybride Applikation eine für Webseiten optimierte Bedienung hat und somit beispielsweise das Potenzial der Touchbedienung eines Smartphones nicht ausgeschöpft wird. Es konnte jedoch auch festgestellt werden, dass mit genügend Zeit das gewollte Aussehen und Bedienung auf jeder Plattform und mit jedem Ansatz implementiert werden kann.

Als letztes Kriterium wurde die Funktionalität betrachtet. Hierbei wurde festgestellt, dass bei der Wahl einer Erweiterung, auf die unterstützten Plattformen geachtet werden müssen, da diese die Plattformabdeckung stark einschränken kann. Zum Beispiel musste der gemischte Ansatz aufgrund der Nutzung eines WebView-Containers auf zwei Plattformen beschränkt werden. Grundsätzlich besitzen jedoch alle untersuchten Ansätze die Möglichkeit, eine Implementierung für die verschiedenen vorhandenen Plattformen anzubieten. Außerdem können mit genügend Zeitaufwand alle Ansätze auf Plattformfunktionalität zugreifen. Bei der Offlinefunktionalität haben die Implementierungen mit Web-Anteil den Nachteil, dass dieser nur online verfügbar ist. Jedoch gibt es für den hybriden Ansatz auch Frameworks, die dies ermöglichen. Hierbei würde jedoch die Web-Anwendung lokal auf dem Gerät laufen müssen.

Für die native Implementierung lässt sich somit festhalten, dass sie zwar die beste Performance bietet, jedoch den höchsten Entwicklungsaufwand besitzt.
Die Implementierung des cross-kompilierten Ansatzes mit Flutter zeichnet sich durch die insgesamt kürzeste Implementierung aus, sie kann jedoch bei der Nutzung von externen Erweiterungen in der Unterstützung der Plattformen eingeschränkt werden. Die hybride Implementierung hat bei Nutzung einer Webseite nur sehr kurze Updatezeiten, jedoch in diesem Fall auch ein Problem bei einer benötigten Offlinefunktionalität. Der vorgestellte gemischte Ansatz bietet die Möglichkeit, Seiten einer Web-Anwendung wiederzuverwenden, während Seiten, die eine Anpassung benötigen zu ersetzen. Jedoch kann durch die Kombination von unterschiedlichen Technologien unerwartete Effekte auftreten, die eine Implementierung erschweren.

Es konnte somit aufgezeigt werden, dass alle Ansätze ihre eigenen Vor- und Nachteile besitzen. Deswegen muss zur Entscheidung für einen bestimmten Ansatz zunächst die geplante Applikation analysiert werden, um eine fundierte Entscheidung zu treffen.


\section{Ausblick}
Diese Arbeit kann nicht alle möglichen Frameworks und Ansätze abdecken. So existieren für die hybride App-Entwicklung Frameworks wie etwa Cordova oder Ionic, die die Funktionalität, der hier vorgestellten hybriden-Implementierung übernehmen und somit eine interessante Alternative zu der hier vorgestellten Implementierung bieten. Ähnlich gibt es auch bei den Cross-kompilierten Frameworks mehr Optionen neben Flutter. Daher wäre eine umfangreiche Betrachtung mit mehreren Implementierungen innerhalb der Ansätze interessant, um ein möglichst vollständiges Bild zu erhalten.

Ebenfalls wäre eine umfangreichere Untersuchung mit Hilfe von weiteren Kriterien sinnvoll. Dabei kann es auch interessant sein, verschiedene Grundvoraussetzung zu schaffen oder unterschiedliche Projekte zu betrachten. Beispielsweise könnte eine Implementierung betrachtet werden, die lediglich Offlinefunktionalität oder Animationen enthält, um ein breiteres Spektrum an verschiedenen Apps abdecken zu können.

Diese Arbeit ist lediglich eine Momentaufnahme und die Applikationsentwicklung kann sich in den nächsten Jahren stark verändern. Einerseits werden neue Cross-Plattform-Frameworks erstellt und Updates für bestehende veröffentlicht. Andererseits kann sich auch die Verteilung der verwendeten Technologie innerhalb weniger Jahre stark verändern. Zum Beispiel ist Flutter in den fünf Jahren seit Veröffentlichung, auf den ersten Platz der meistgenutzten Cross-Plattform-Frameworks geklettert und hat dabei ReactNative 2021 von der Spitze abgelöst. Doch auch die Veränderungen in den bestehenden Frameworks und Programmiersprachen kann die Ergebnisse dieser Arbeit stark verändern. So veröffentlichte das Framework Flutter, während der Erstellung dieser Arbeit, ein großes Update, das neben der Unterstützung weiterer Plattformen, auch Performanceverbesserungen in unterschiedlichen Bereichen enthielt \cite{flutter3}. Doch auch die nativen Programmiersprachen entwickeln sich weiter und erhalten Updates, die eine Entwicklung vereinfachen. Daher ist eine erneute Untersuchung der Kriterien in einigen Jahren wünschenswert, um ein aktualisiertes Bild für den Stand der Technik zu erhalten.


%Bei der Betrachtung der typischen Entwicklungsoperationen konnte festgestellt werden, dass Flutter beim Nachladen von Änderung in die laufende Applikation deutlich schneller war als die native Applikation. So war das Laden einer Änderung in eine Seite bei der Flutter Implementierung in gerade einmal 0,6 Sekunden möglich, während die Implementierungen mit Kotlin circa 2 Sekunden länger brauchten. Dazu kommt, dass bei der cross-kompilierten Applikation, der Entwickler nicht zur Startzeite der Applikation zurückgesetzt wurde, wie es der Fall bei der nativen und der hybriden Implementierung war.

%Insgesamt lässt sich so festhalten, dass zwar native die beste Performance bieten, jedoch die höchsten Kosten in der Entwicklung haben, da eine Implementierung für jede Plattform erstellt werden muss. Der cross-kompilierte Ansatz kann durch die Nutzung von Flutter sowohl bei der Performance als auch den anderen Kriterien mithalten. Außerdem benötigte die Implementierung die geringsten Zeilen an Code und durch die Nutzung einer eigenen Render-Engine kann es die Oberfläche schneller bauen, als die anderen Ansätze. Die hybride Implementierung hat gezeigt, dass  bei einer bereits bestehenden Webseite, die benötigte Implementierung je Plattform gering gehalten werden kann und Updates schnell an die Nutzer verteilt werden könne. Jedoch ist die Performance stark von der Internetverbindung abhängig und eine Offline Funktionalität ist nicht möglich. Der hier vorgestellte gemischte Ansatz hat sowohl eine schlechtere Performance als auch eine stark eingeschränkte Plattformabdeckung, jedoch ermöglicht er es, Seiten die lediglich 

%Der cross-kompilierte Ansatz mit Flutter zeigte eine etwas verringerte Performance. Dafür konnte es in den meisten anderen Kriterien entweder mit der nativen Implementation mithalten oder sogar besser abschneiden und bietet somit eine echte Alternative zur nativen Entwicklung. Lediglich in der Qualität einiger Plugins können die recht neuen Bibliotheken noch nicht mit denen für Kotlin oder andere Programmiersprachen mithalten. Dies wird sich jedoch verbessern, wenn Flutter weiterhin seine Community aufbauen kann, wie bisher. Außerdem ist die Wiederverwendbarkeit des Codes im Vergleich zur hybriden Applikation hier nicht gegeben. Dadurch ist bei einem Wechsel der Technologie zu einer anderen, eine umfangreiche Neuprogrammierung nötig. Jedoch kann durch eine Entwicklung mit Flutter oder anderen Cross-Compilierten Applikationen bereits während der Entwicklung viel Geld gesparrt werden, da fast alles nur ein einziges mal implementiert werden muss.

%Dies ist bei den hybriden Applikationen deutlich besser, da der Großteil mit Web-Technologie geschrieben ist. Dadurch kann schnell und mit begrenzten Aufwand auf eine andere Technologie gewechselt werden. Dazu kommt durch die Nutzung der Web Technologie eine deutlich höhere Entwicklergemeinschaft in diesem Bereich. Hier muss lediglich bei der umgebenden Technologie auf diesen Faktor geschaut werden. Dieser ist aber, wie bereits erwähnt, deutlich einfacher auszutauschen. Jedoch muss hier für jede Plattform eine eigene Version des Plattformcodes geschrieben werden. Dadurch erhöht sich der Programmieraufwand und dadurch auch die Kosten.

%Wie durch die verschiedenen untersuchten Kriterien außerdem klar geworden ist, konnte der hier vorgestellte gemischte Ansatz aus hybrider und Cross-Compilierter Applikation in den meisten Kriterien nicht überzeugen. So hatte er eine spürbar schlechtere Performance, höheren Programmieraufwand, eine deutliche Einschränkung in der Plattformabdeckung oder auch in der Dokumentation. Jedoch kann dieser Ansatz von Vorteil sein, wenn eine eine bestehende Webseite in eine Cross-Plattform App mit beispielsweise Flutter umgebaut werden soll. Dabei kann ein langsamer Übergang von Webseite zu einer eigenen App erreicht werden. Ein zweiter Ansatz ist, wenn statische Seite, die keine Plattformfunktionalität angezeigt werden sollen, kann eine Webseiten Version genutzt werden, um eine Reimplementierung zu umgehen.

%Wie also im laufe der Arbeit klar geworden ist, gibt es verschiedene Ansätze und noch viel mehr unterschiedliche Frameworks, die innerhalb der einzelnen Cross-Plattform Ansätzen ausgewählt werden können. Dabei haben jede unterschiedliche Vor und Nachteile. Jedoch kann festgestellt werden, dass Flutter als eine mögliche Alternative zu einer nativen Entwicklung in Betracht kommt und eine native Entwicklung nicht mehr als einziger Standard in der Entwicklung von Multi-Plattform-Anwendung gesehen werden darf.
