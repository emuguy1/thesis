Wenn eine Applikation entwickelt werden soll, ist die Wahl der richtigen Methode, Frameworks oder auch der Programmiersprache eine wichtige Entscheidung für das Projekt. Vieles kann geändert werden, aber um eine derartige Entscheidung zu ändern, müssen Teile der Anwendung oder auch der komplette Quellcode neu geschrieben werden. Deswegen muss hier eine 

In Zeiten der Heimcomputern und des immer mehr wachsenden Internets waren hier oft Webseiten die Lösung um eine Anwendung für die verschiedenen Plattformen anzubieten.Da die meisten Leute oft nur einen Computer besaßen, entwickelte man meist nur PC-Anwendungen oder eben Webapplikationen, die jeder von überall aus aufrufen konnte, ohne die Applikation auf seinem eigenen Computer installieren zu müssen.

Mit der Ära der Smartphones jedoch änderte sich das etwas, denn auch hier kann man natürlich Webapplikationen nutzen, jedoch sind diese oft nicht angepasst gewesen.

Daraus entstand eine Zeit wo mobile Applikationen speziell für das eigene Smartphone entwickelt wurden. Die Nutzung von Smartphones öffnete auch die Tür, andere Funktionalitäten wie die Kamera zu nutzen. Deshalb wurden dann Applikationen mit Objectiv-C/Swift für iOS und Java bzw. mittlerweile Kotlin für Android entwickelt. Mit deren Hilfe wurden native Anwendungen für das Smartphone entwickelt, die diese neuen mobilen Plattformen optimal ausnutzen konnten.

Wenn man jedoch nun auf mehreren Plattformen seine Applikationen veröffentlichen wollte, so musste man für jede Plattform eine eigene Applikationen in der jeweiligen Programmiersprachen schreiben. Dadurch entstand ein hoher Aufwand. Deswegen wurde bereits früh mit der Entwicklung von sogenannten Cross-Platform Frameworks gestartet, die es ermöglichen sollten, mit nur einem Code möglichst viele Plattformen abzudecken. So kam bereits 2008 etwa PhoneGap heraus. Es war ein Open-Source Framework zur Entwicklung von hybriden mobilen Applikationen. PhoneGap und sein Nachfolger Cordova zusammen waren einige Zeit auch sehr beliebt in diesem Bereich. 2019 hatten beide zusammen einen Marktanteil von knapp 40\% unter den Cross-platform Frameworks \cite{statist_CP_Framework}.

Trotzdem war die Entwicklung von nativen Apps bisher der Standard. Auch konnte man bereits den ein oder anderen Blogpost von größeren Unternehmen lesen, in dem sie erklären, wie sie versuchten eine  Cross-Plattform-Entwicklungen einzuführen, jedoch aus verschiedenen Gründen einstellten. So etwa auch Airbnb. Sie nutzten das von Facebook mitentwickelte Framework React Native. 2019 hatte dieses Framework einen Marktanteil von 42\%. 
\break
Die Ziele waren einfach:
\begin{enumerate}
    \item Schnelleres entwickeln
    \item Die gleiche Codequalität beibehalten
    \item Nur noch eine Codebasis
    \item Die Entwicklererfahrung verbessern\cite{Airbnb_react_goals}
\end{enumerate}

Jedoch traten während der Entwicklung Probleme auf, die dazu führten, dass sie 2018 wieder zu einem nativen Ansatz zurückkehrten.
\TODO{Hier eventuell noch etwas mehr dazu warum, aber nicht zwingend}

Durch Beispiele wie dieses, waren App-Entwickler lange Zeit skeptisch gegenüber derartigen Lösungen, da dies auch immer mit großen Änderungen und hohen Investitionen verbunden waren.

\begin{figure}[ht]
  \centering
  \includegraphics[height=7cm,keepaspectratio]{images/cross-platform-mobile-frameworks.png} 
  \caption[Statistik Cross-Plattform-Frameworks]{Cross-Plattform-Frameworks 2019-2021 \cite{statist_CP_Framework}}
  \label{fig:statista_cross_plattform}
\end{figure}
Abbildung \ref{fig:statista_cross_plattform} zeigt eine Statistik von JetBrains, die die Verteilung von verschiedenen Cross-Plattform-Entwicklungen zeigt. Sie zeigt eindrucksvoll wie schnell sich die Verteilung von Cross-Platform Frameworks ändern kann.
Ein Framework, das hier besonders allerdings besonders heraus sticht ist Flutter. Es ist ein Framework das erst 2017 auf den Markt gekommen ist und innerhalb von gerade einmal 4 Jahren auf einen Marktanteil von 42\% gekommen ist. Von einigen wird es schon als der neue Standard angesehen und Unternehmen wie msg. entwickeln neue Applikationen, wenn nicht offiziell anders gewünscht nur noch in Flutter.
\TODO{Fragen wie man Aussagen aus Vortrag benutzen kann}

Wegen den oben genannten Unsicherheiten und einigen anderen Gründen gibt es allerdings auch viele Entwickler die immer noch nativ entwickeln. So entwickelt die Number42 alle ihre betreuten mobilen Applikationen mit den nativen Programmiersprachen. Auch die nativen Programmiersprachen entwickeln sich stetig weiter und bekommen Änderungen, die eine Entwicklung vereinfachen und beschleunigen.



Deswegen soll im folgenden verschiedene Ansätze zur Entwicklung von mobilen Anwendungen untersucht werden und dabei darauf eingegangen werden, was die Vor- und Nachteile der verschiedenen Ansätze sind und anhand von verschiedenen Kriterien eine Einordnung liefern
\TODO{umschreiben}
Deswegen soll in dieser Arbeit drei unterschiedliche Ansätze mit beispielhaften Frameworks und Programmiersprachen betrachtet werden um am Ende vielleicht eine bessere Einschätzung geben zu können, wie eine solche Entscheidung ausfallen könnte und Gründe für und gegen bestimmte Ansätze geben.
\TODO{umschreiben}
Daher ist der Fokus dieser Arbeit einmal an einigen konkreten Beispielen zu untersuchen, wie der aktuelle Stand der Technologie hier ist und zu erforschen, welche Einschränkungen die Frameworks besitzen um hier auch eine gewisse Bewertung zu Benutzbarkeit als Appagentur zu untersuchen.


\section{Einordnung der Arbeit v}
Der Hauptteil der Arbeit ist es, eine Applikation anhand verschiedener Ansätze zu implementieren, um so Aussagen über die Nutzbarkeit zu treffen. Es ist dabei nicht Ziel der Arbeit, zu erklären welche Programmiersprache oder Entwicklungsvariante zur Programmierung einer App genutzt werden sollen. Vielmehr ist es Ziel die unterschiedlichen Ansätze auszuprobieren, zu analysieren und am Ende einen Entscheidungskompass zu geben, wann welche Technologien gut eingesetzt werden kann. 

Dabei werden als Implementierungssprachen Kotlin und Dart verwendet. Hierbei soll vor allem darauf eingegangen werden, wie gut sich das neue Framework Flutter, das mithilfe der Programmiersprache Dart entwickelt wird, für eine Entwicklung von mobilen Apps anbietet, da dies eine noch recht neue Technologie auf dem Markt ist, allerdings bereits einen großen Einfluss im Bereich der Multi-Plattform Entwicklung hat.

\section{Aufbau der Arbeit v}
In Kapitel 1 der Arbeit ist eine Einleitung und Motivation zu finden, weshalb diese Arbeit so sinnvoll ist, erstellt zu werden. Danach wird in Kapitel 2 die verwandten Arbeiten vorgestellt um aufzuzeigen, was hier bereits existiert. In Kapitel 3 wird daraufhin das Projekt vorgestellt, an denen Implementierungen unterschiedlicher Ansätze ausprobiert wurden, außerdem werden die Arbeit abgegrenzt und einige Grundlagen dieser Arbeit erklärt.
Kapitel 4 besteht aus einer Beschreibung der Implementierung, Auffälligkeiten, ein paar Besonderheiten der einzelnen Ansätze und deren Implementierungen hingewiesen. Auch werden ein paar Grundlagen der Entwicklung in den einzelnen Unterkapiteln erklärt um ein Verständnis der Programmiersprachen zu erhalten.
In Kapitel 5 soll anschließend eine Auswertung der einzelnen Entwicklungsansätze stattfinden und anhand einiger Kriterien verglichen werden. Abschließend soll mit Kapitel 6 ein Fazit gezogen werden und eine Einordnung der Ergebnisse der Arbeit passieren.